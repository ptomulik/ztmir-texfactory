\documentclass[paper=a4,DIV=12]{lpas}

\usepackage[polish]{babel}
\usepackage[T1]{fontenc}
\usepackage[utf8]{inputenc}
\usepackage[scaled=.9,otfmath]{XCharter}
\usepackage[scaled=.9,varqu,varl,mono,hyphenate]{inconsolata}
\usepackage[uprightscript,xcharter,vvarbb,scaled=.9]{newtxmath}
%\usepackage[scaled=.9,type1]{sourcesanspro}
\usepackage{mathrsfs}
\usepackage[]{hyperref}
\usepackage{natbib}

%%\everymath{\displaystyle}

\usepackage{bm}
\usepackage{amsmath,amsfonts}
%%\usepackage{mathtools}
\usepackage{graphicx}
\usepackage[titletoc,title]{appendix}
%%\usepackage{subcaption}
%\usepackage{listings}
\usepackage{float}
%%\usepackage{tabularx}
\usepackage{tikz}
%%\usepackage{circuitikz}
%%\usetikzlibrary{arrows.meta,3d,patterns,calc,shapes.geometric}
\usetikzlibrary{arrows.meta}

\newcommand{\brm}[1]{\bm{\mathrm{#1}}}
\renewcommand{\arraystretch}{1.2}
%%\newcolumntype{L}[1]{>{\raggedright\arraybackslash}p{#1}}
%%\newcolumntype{C}[1]{>{\centering\arraybackslash}p{#1}}
%%\newcolumntype{R}[1]{>{\raggedleft\arraybackslash}p{#1}}

% commath provides \od, but the package is not available on my travis-ci setup
\newcommand*{\od}[2]{\frac{\mathrm{d}#1}{\mathrm{d}#2}}
\newcommand*{\odn}[3]{\frac{\mathrm{d}^{#1}#2}{\mathrm{d}{#3}^{#1}}}
\newcommand*{\tod}[2]{\tfrac{\mathrm{d}#1}{\mathrm{d}#2}}
\newcommand*{\todn}[3]{\tfrac{\mathrm{d}^{#1}#2}{\mathrm{d}{#3}^{#1}}}
% gensymb provides \degree command, but whole package for just one symbol?
%%\newcommand{\degree}{^{\circ}}

\setcitestyle{numbers,square,comma}

\begin{document}


\subject{Laboratorium pomiarów, automatyki i~sterowania~I}
\supertitle{Ćwiczenie 10}
\title{Wyznaczanie charakterystyki amplitudowo-fazowej na podstawie odpowiedzi skokowej}

\author{Zakład Teorii Maszyn i Robotów\\ ITLiMS, MEIL, PW}

\date{}
\maketitle

\pagebreak
\tableofcontents
\pagebreak

%%\begin{abstract}
%%\noindent
%%\end{abstract}

\section{Wstęp}
\label{sec:LBU9I}

Właściwości dynamiczne liniowych układów o stałych skupionych można określić
m.in. na podstawie odpowiednich charakterystyk zarówno w dziedzinie czasu, jak
też dziedzinie częstotliwości. W pierwszym przypadku często wykorzystuje się
charakterystyki impulsową i skokową natomiast w drugim, charakterystyki
częstotliwościowe (np. wykresy Nyquista lub Bodego). Znajomość transmitancji
operatorowej układu, a co za tym idzie -- transmitancji widmowej, pozwala na
wyznaczenie obu typów charakterystyk. W związku z tym zachodzą również
możliwości:
%%%%%%%%%%%%%%%%%%%%%%%%%%%%%%%%%%%%%%%%%%%%%%%%%%%%%%%%%%%%%%%%%%%%%%%%%%%%%
\begin{itemize}
  \item wyznaczania transmitancji widmowej układu na podstawie charakterystyki
    skokowej,
  \item wyznaczania charakterystyki skokowej na podstawie transmitancji
    widmowej układu.
\end{itemize}
%%%%%%%%%%%%%%%%%%%%%%%%%%%%%%%%%%%%%%%%%%%%%%%%%%%%%%%%%%%%%%%%%%%%%%%%%%%%%


\subsection{Cel ćwiczenia}
\label{sec:ZWO2V}

Celem ćwiczenia jest zapoznanie studentów z~metodą wyznaczania transmitancji
widmowej układu na podstawie zdjętej doświadczalnie charakterystyki skokowej.

\section{Przebieg ćwiczenia}
\label{sec:2USO1}

To ćwiczenie laboratoryjne stanowi uzupełnienie i kontynuację ćwiczenia
,,Badanie charakterystyk częstotliwościowych i przebiegów nieustalonych
podstawowych członów automatyki''. Zakłada się, że studenci wykonali wcześniej
ćwiczenie i znają sposób pomiaru charakterystyk A-F czwórników RLC przy użyciu
analizatora transmitancji.

\section{Opracowanie wyników pomiarów}
\label{sec:II99S}

%%%%%%%%%%%%%%%%%%%%%%%%%%%%%%%%%%%%%%%%%%%%%%%%%%%%%%%%%%%%%%%%%%%%%%%%%%%%%
\begin{figure}[H]
  \centering
  \input{lpas10/measurements.tex}
  \caption{Przykładowe dane pomiarowe dla członu oscylacyjnego}
  \label{fig:RKDNU}
\end{figure}
%%%%%%%%%%%%%%%%%%%%%%%%%%%%%%%%%%%%%%%%%%%%%%%%%%%%%%%%%%%%%%%%%%%%%%%%%%%%%


\section{Przygotowanie sprawozdania}
\label{sec:ZEUBF}


\clearpage

\begin{appendices}
  \section{Podstawy teoretyczne}
  \label{sec:K5NDO}

  Badany układ traktujemy jako ,,czarną skrzynkę'' posiadającą wejście i~wyjście.
  Jeśli na wejście podamy zmien\-ny w~czasie sygnał $u(t)$, to na wyjściu
  zaobserwujemy sygnał $y(t)$.
  %%%%%%%%%%%%%%%%%%%%%%%%%%%%%%%%%%%%%%%%%%%%%%%%%%%%%%%%%%%%%%%%%%%%%%%%%%%%%
  \begin{figure}[htbp]
    \centering
    \begin{tikzpicture}[scale=1.0]
      \node[draw,inner sep=1em] (G) {$G\left(s\right)$};
      \draw[{Stealth}-] (G.west) -- ++(-1cm, 0cm) node[left] {$U\left(s\right)$};
      \draw[-{Stealth}] (G.east) -- ++( 1cm, 0cm) node[right] {$Y\left(s\right)$};
    \end{tikzpicture}
    \caption{Element badany, jego transmitancja $G(s)$, sygnały}
    \label{fig:6TEOB}
  \end{figure}
  %%%%%%%%%%%%%%%%%%%%%%%%%%%%%%%%%%%%%%%%%%%%%%%%%%%%%%%%%%%%%%%%%%%%%%%%%%%%%
  W~rozważaniach teoretycznych posługujemy się zwyczajowo reprezentacją
  $U = \mathscr{L}\{u\}$ i~$Y = \mathscr{L}\{y\}$ powyższych sygnałów
  w~dziedzinie operatorowej (transformata Laplace'a sygnałów $u$ i $y$).
  W~ogólnych rozważaniach używa się też pojęcia transmitancji operatorowej
  (funkcji przejścia) zdefiniowanej jako
  %%%%%%%%%%%%%%%%%%%%%%%%%%%%%%%%%%%%%%%%%%%%%%%%%%%%%%%%%%%%%%%%%%%%%%%%%%%%%
  \begin{equation}
    G(s) = \frac{Y(s)}{U(s)}.
    \label{eq:FBOCS}
  \end{equation}
  %%%%%%%%%%%%%%%%%%%%%%%%%%%%%%%%%%%%%%%%%%%%%%%%%%%%%%%%%%%%%%%%%%%%%%%%%%%%%

  Niech $h(t)$ będzie odpowiedzią układu na skok jednostkowy $\mathbf{1}(t)$.
  Z~tablicy transformat: $\mathscr{L}\{\mathbf{1}(t)\} = 1/s$. Wstawiając $U(s)
  = 1/s$ oraz $Y(s) = H(s) = \mathscr{L}\{h(t)\}$ do~\eqref{eq:FBOCS} otrzymujemy:
  %%%%%%%%%%%%%%%%%%%%%%%%%%%%%%%%%%%%%%%%%%%%%%%%%%%%%%%%%%%%%%%%%%%%%%%%%%%%%
  \begin{equation}
    \begin{aligned}
      & G(s) = s H(s)
             = \mathscr{L}\left\{h'(t)\right\}
             = \int_0^{\infty} {h'(t) \cdot e^{-st} dt}, &
      & \text{przy } h(0) = 0, &
    \end{aligned}
    \label{eq:SQUJF}
  \end{equation}
  %%%%%%%%%%%%%%%%%%%%%%%%%%%%%%%%%%%%%%%%%%%%%%%%%%%%%%%%%%%%%%%%%%%%%%%%%%%%%
  gdzie $h'(t)$ oznacza pochodną funkcji $h(t)$.

  \subsection{Identyfikacja transmitancji przez rozwinięcie w~szereg}
  \label{eq:9XWOO}

  Istnieje grupa metod identyfikacji $G(s)$ bazujących na zastąpieniu całki
  \eqref{eq:SQUJF} szeregiem
  %%%%%%%%%%%%%%%%%%%%%%%%%%%%%%%%%%%%%%%%%%%%%%%%%%%%%%%%%%%%%%%%%%%%%%%%%%%%%
  \begin{equation}
    G(s) = \sum_{k=0}^{\infty} w_k s^k,
    \label{eq:2PAK7}
  \end{equation}
  %%%%%%%%%%%%%%%%%%%%%%%%%%%%%%%%%%%%%%%%%%%%%%%%%%%%%%%%%%%%%%%%%%%%%%%%%%%%%
  z~odpowiednio dobranymi wagami $w_k$. Metody wyznaczania wag będą treścią
  dalszych podrozdziałów, tutaj zaznaczymy jedynie, że wyrażają one informację
  o~zmierzonej odpowiedzi $h(t)$ na skok jednostkowy.

  Jeśli znany jest (maksymalny) rząd $n$ badanego układu liniowego, to
  przyjmuje się, że jego transmitancja $G(s)$ ma postać funkcji wymiernej
  o~rzeczywistych współczynnikach $v_i$, $i=1,\dots,2n+1$:
  %%%%%%%%%%%%%%%%%%%%%%%%%%%%%%%%%%%%%%%%%%%%%%%%%%%%%%%%%%%%%%%%%%%%%%%%%%%%%
  \begin{equation}
    G(s) = \frac{v_{2n+1} s^n + v_{2n} s^{n-1} + \cdots + v_{n+2} s + v_{n+1}}
                {v_n s^n + v_{n-1} s^{n-1} + \cdots + v_1 s + 1}.
    \label{eq:Z0EFM}
  \end{equation}
  %%%%%%%%%%%%%%%%%%%%%%%%%%%%%%%%%%%%%%%%%%%%%%%%%%%%%%%%%%%%%%%%%%%%%%%%%%%%%
  Zestawiając \eqref{eq:2PAK7} z~\eqref{eq:Z0EFM} uzyskuje się równanie
  %%%%%%%%%%%%%%%%%%%%%%%%%%%%%%%%%%%%%%%%%%%%%%%%%%%%%%%%%%%%%%%%%%%%%%%%%%%%%
  \begin{equation}
      v_{2n+1} s^n + v_{2n} s^{n-1} + \cdots + v_{n+2} s + v_{n+1}
      = \left(
          v_n s^n + v_{n-1} s^{n-1} + \cdots + v_1 s + 1
        \right) \sum_{k=0}^{\infty} w_k s^k.
    \label{eq:6EIFN}
  \end{equation}
  %%%%%%%%%%%%%%%%%%%%%%%%%%%%%%%%%%%%%%%%%%%%%%%%%%%%%%%%%%%%%%%%%%%%%%%%%%%%%
  Przez przyrównanie współczynników stojących przy kolejnych potęgach $s^i$,
  $i=0,1,\dots,2n$ po lewej i~prawej stronie równania~\eqref{eq:6EIFN}
  otrzymuje się układ równań liniowych $(2n+1) \times (2n+1)$ względem~$v_i$:
  %%%%%%%%%%%%%%%%%%%%%%%%%%%%%%%%%%%%%%%%%%%%%%%%%%%%%%%%%%%%%%%%%%%%%%%%%%%%%
  \begin{equation}
    \begin{aligned}
      v_{n+1} & = w_0 \\
    - w_0 v_1 + v_{n+2} & = w_1 \\
      \vdots  & \\
   -  w_{n-1} v_1 - w_{n-2} v_2 - \cdots - w_0 v_n + v_{2n+1} &= w_n \\
      - w_n v_1 - w_{n-1} v_2 - \cdots - w_1 v_n &=  w_{n+1} \\
      - w_{n+1} v_1 - w_n v_2 - \cdots - w_2 v_n & = w_{n+2} \\
      \vdots & \\
      - w_{2n-1} v_1 - w_{2n-2} v_2 - \cdots - w_n v_n & = w_{2n} \\
    \end{aligned}
    \label{eq:4LX5X}
  \end{equation}
  %%%%%%%%%%%%%%%%%%%%%%%%%%%%%%%%%%%%%%%%%%%%%%%%%%%%%%%%%%%%%%%%%%%%%%%%%%%%%
  Równania \eqref{eq:4LX5X} da się zapisać w~postaci macierzowej
  %%%%%%%%%%%%%%%%%%%%%%%%%%%%%%%%%%%%%%%%%%%%%%%%%%%%%%%%%%%%%%%%%%%%%%%%%%%%%
  \begin{equation}
    \mathbf{A}_n \cdot \mathbf{v}_n = \mathbf{w}_n,
    \label{eq:7CL5I}
  \end{equation}
  %%%%%%%%%%%%%%%%%%%%%%%%%%%%%%%%%%%%%%%%%%%%%%%%%%%%%%%%%%%%%%%%%%%%%%%%%%%%%
  gdzie
  %%%%%%%%%%%%%%%%%%%%%%%%%%%%%%%%%%%%%%%%%%%%%%%%%%%%%%%%%%%%%%%%%%%%%%%%%%%%%
  \begin{equation}
    \begin{aligned}
      \mathbf{A}_n = \begin{bmatrix}
         0         & 0         & 0       & 0      & 1 &  0  & 0 & \cdots & 0 \\
         -w_0      & 0         & 0       & 0      & 0 &  1  & 0 & \cdots & 0 \\
         -w_1      & -w_0      & 0       & 0      & 0 &  0  & 1 &        & 0 \\
         \vdots    &           & \ddots  &        &   &     &   & \ddots &   \\
         -w_{n-1}  & -w_{n-2}  & \cdots  & -w_0   & 0 &  0  & 0 &        & 1 \\
         -w_{n}    & -w_{n-1}  & \cdots  & -w_1   & 0 &  0  & 0 & \cdots & 0 \\
         -w_{n+1}  & -w_{n}    & \cdots  & -w_2   & 0 &  0  & 0 & \cdots & 0 \\
         \vdots    &           &         & \vdots & \vdots & &  &        &   \\
         -w_{2n-1} & -w_{2n-2} & \cdots  & -w_n   & 0 &  0  & 0 & \cdots & 0
      \end{bmatrix}, &
      & \mathbf{v}_n = \begin{bmatrix}
        v_1 \\ v_2 \\ \vdots \\ v_n \\
        v_{n+1} \\ v_{n+2} \\ \vdots \\ v_{2n+1}
      \end{bmatrix}, &
      & \mathbf{w}_n = \begin{bmatrix}
        w_0 \\ w_1 \\ \vdots \\ w_n \\ w_{n+1} \\ \vdots \\ w_{2n}
      \end{bmatrix}. &
    \end{aligned}
    \label{eq:VR42H}
  \end{equation}
  %%%%%%%%%%%%%%%%%%%%%%%%%%%%%%%%%%%%%%%%%%%%%%%%%%%%%%%%%%%%%%%%%%%%%%%%%%%%%

  \subsubsection{Metoda dla układów drugiego rzędu}
  \label{eq:7RDBZ}

  W~kontekście przedmiotowego eksperymentu zakładamy, że badane układy mają
  rząd nie wyższy niż $n=2$, w~związku z czym, macierze układu równań
  \eqref{eq:7CL5I} mają postać
  %%%%%%%%%%%%%%%%%%%%%%%%%%%%%%%%%%%%%%%%%%%%%%%%%%%%%%%%%%%%%%%%%%%%%%%%%%%%%
  \begin{equation}
    \begin{aligned}
      \mathbf{A}_2 = \begin{bmatrix}
         0    &    0 & 1 & 0 & 0 \\
         -w_0 &    0 & 0 & 1 & 0 \\
         -w_1 & -w_0 & 0 & 0 & 1 \\
         -w_2 & -w_1 & 0 & 0 & 0 \\
         -w_3 & -w_2 & 0 & 0 & 0
      \end{bmatrix}, &
      & \mathbf{v}_2 = \begin{bmatrix}
        v_1 \\ v_2 \\ v_3 \\ v_4 \\ v_5
      \end{bmatrix}, &
      & \mathbf{w}_2 = \begin{bmatrix}
        w_0 \\ w_1 \\  w_2 \\ w_3 \\ w_4
      \end{bmatrix}, &
    \end{aligned}
  \end{equation}
  %%%%%%%%%%%%%%%%%%%%%%%%%%%%%%%%%%%%%%%%%%%%%%%%%%%%%%%%%%%%%%%%%%%%%%%%%%%%%
  a~transmitancja operatorowa
  %%%%%%%%%%%%%%%%%%%%%%%%%%%%%%%%%%%%%%%%%%%%%%%%%%%%%%%%%%%%%%%%%%%%%%%%%%%%%
  \begin{equation}
    G(s) = \frac{v_5 s^2 + v_4 s + v_3}{v_2 s^2 + v_1 s + 1}.
  \end{equation}
  %%%%%%%%%%%%%%%%%%%%%%%%%%%%%%%%%%%%%%%%%%%%%%%%%%%%%%%%%%%%%%%%%%%%%%%%%%%%%

  \subsubsection{Metoda dla układów pierwszego rzędu}
  \label{eq:R5YS1}

  Jeśli badany układ jest rzędu $n=1$, to macierze układu równań
  \eqref{eq:7CL5I} mają postać
  %%%%%%%%%%%%%%%%%%%%%%%%%%%%%%%%%%%%%%%%%%%%%%%%%%%%%%%%%%%%%%%%%%%%%%%%%%%%%
  \begin{equation}
    \begin{aligned}
      \mathbf{A}_1 = \begin{bmatrix}
         0    &    1 &    0 \\
         -w_0 &    0 &    1 \\
         -w_1 &    0 &    0
      \end{bmatrix}, &
      & \mathbf{v}_1 = \begin{bmatrix}
        v_1 \\ v_2 \\ v_3
      \end{bmatrix}, &
      & \mathbf{w}_1 = \begin{bmatrix}
        w_0 \\ w_1 \\  w_2
      \end{bmatrix}, &
    \end{aligned}
  \end{equation}
  %%%%%%%%%%%%%%%%%%%%%%%%%%%%%%%%%%%%%%%%%%%%%%%%%%%%%%%%%%%%%%%%%%%%%%%%%%%%%
  a~transmitancja operatorowa
  %%%%%%%%%%%%%%%%%%%%%%%%%%%%%%%%%%%%%%%%%%%%%%%%%%%%%%%%%%%%%%%%%%%%%%%%%%%%%
  \begin{equation}
    G(s) = \frac{v_3 s + v_2}{v_1 s + 1}.
  \end{equation}
  %%%%%%%%%%%%%%%%%%%%%%%%%%%%%%%%%%%%%%%%%%%%%%%%%%%%%%%%%%%%%%%%%%%%%%%%%%%%%

  \subsection{Metoda bazująca na rozwinięciu funkcji wykładniczej w~szereg potęgowy}
  \label{eq:Q1UWQ}

  Jest to jedna z~możliwych metod ustalenia współczynników $w_k$ występujących
  we wzorze~\eqref{eq:2PAK7}. Bazuje ona na rozwinięciu w~szereg potęgowy
  (Taylora) funkcji wykładniczej $e^x$. Rozwinięcie to, zastosowane w~tym
  przypadku do czynnika $e^{-st}$ występującego w~całce~\eqref{eq:SQUJF},
  daje poniższą formułę
  %%%%%%%%%%%%%%%%%%%%%%%%%%%%%%%%%%%%%%%%%%%%%%%%%%%%%%%%%%%%%%%%%%%%%%%%%%%%%
  \begin{equation}
    e^{-st} = \sum_{k=0}^{\infty}{\frac{(-st)^k}{k!}}
            = \sum_{k=0}^{\infty}{\frac{(-1)^k}{k!} s^k t^k }.
    \label{eq:BPUN9}
  \end{equation}
  %%%%%%%%%%%%%%%%%%%%%%%%%%%%%%%%%%%%%%%%%%%%%%%%%%%%%%%%%%%%%%%%%%%%%%%%%%%%%
  Po uwzględnieniu \eqref{eq:BPUN9} w~całce \eqref{eq:SQUJF} otrzymuje się
  %%%%%%%%%%%%%%%%%%%%%%%%%%%%%%%%%%%%%%%%%%%%%%%%%%%%%%%%%%%%%%%%%%%%%%%%%%%%%
  \begin{equation}
    G(s) = \int_0^{\infty}{h'(t) \cdot e^{-s t} dt}
    = \sum_{k=0}^{\infty}{\frac{(-1)^k}{k!}s^k \int_0^{\infty}{t^k h'(t) dt}}
    \label{eq:FEPSU}
  \end{equation}
  %%%%%%%%%%%%%%%%%%%%%%%%%%%%%%%%%%%%%%%%%%%%%%%%%%%%%%%%%%%%%%%%%%%%%%%%%%%%%
  Oznaczmy jako $m_k$ poniższe wyrażenie całkowe występujące w~\eqref{eq:FEPSU}
  %%%%%%%%%%%%%%%%%%%%%%%%%%%%%%%%%%%%%%%%%%%%%%%%%%%%%%%%%%%%%%%%%%%%%%%%%%%%%
  \begin{equation}
    m_k = \int_0^{\infty}{t^k h'(t) dt},
    \label{eq:6658Y}
  \end{equation}
  %%%%%%%%%%%%%%%%%%%%%%%%%%%%%%%%%%%%%%%%%%%%%%%%%%%%%%%%%%%%%%%%%%%%%%%%%%%%%
  zwane momentem rzędu $k$. Wtedy, współczynniki $w_k$ dla metody obliczeniowej
  będą dane jako
  %%%%%%%%%%%%%%%%%%%%%%%%%%%%%%%%%%%%%%%%%%%%%%%%%%%%%%%%%%%%%%%%%%%%%%%%%%%%%
  \begin{equation}
    w_k = \frac{(-1)^k}{k!} m_k.
    \label{eq:W0DJC}
  \end{equation}
  %%%%%%%%%%%%%%%%%%%%%%%%%%%%%%%%%%%%%%%%%%%%%%%%%%%%%%%%%%%%%%%%%%%%%%%%%%%%%

  \subsubsection{Analityczne wyznaczanie momentów \texorpdfstring{$m_k$}{mk}}

  Szczególnej uwagi wymagają momenty $m_k$. Dla $k=0$ mamy (przy oznaczeniu
  $h(\infty) = \lim_{T\to\infty}h(T)$):
  %%%%%%%%%%%%%%%%%%%%%%%%%%%%%%%%%%%%%%%%%%%%%%%%%%%%%%%%%%%%%%%%%%%%%%%%%%%%%
  \begin{equation}
    m_0 = \int_0^{\infty} 1 \cdot h'(t) dt = h(\infty) - h(0).
    \label{eq:6V264}
  \end{equation}
  %%%%%%%%%%%%%%%%%%%%%%%%%%%%%%%%%%%%%%%%%%%%%%%%%%%%%%%%%%%%%%%%%%%%%%%%%%%%%
  Aby uniknąć różniczkowania $h(t)$, dla $k \ge 1$ całkę \eqref{eq:6658Y}
  przekształca się stosując całkowanie przez części
  %%%%%%%%%%%%%%%%%%%%%%%%%%%%%%%%%%%%%%%%%%%%%%%%%%%%%%%%%%%%%%%%%%%%%%%%%%%%%
  \begin{equation}
    \begin{aligned}
      & m_k = \int_0^{\infty} {t^k h'(t) dt}
            = \left[t^k h(t)\right]_0^{\infty} - k \int_0^{\infty}{t^{k-1} h(t) dt}, &
      & \text{dla } k=1, \dots, \infty &
    \end{aligned}
    \label{eq:MRW8Q}
  \end{equation}
  %%%%%%%%%%%%%%%%%%%%%%%%%%%%%%%%%%%%%%%%%%%%%%%%%%%%%%%%%%%%%%%%%%%%%%%%%%%%%
  Zakładając, że $h(0)$ i~$h(\infty)$ przyjmują wartości ustalone i $|h(0)| <
  \infty$ oraz $|h(\infty)| < \infty$, możemy zapisać
  %%%%%%%%%%%%%%%%%%%%%%%%%%%%%%%%%%%%%%%%%%%%%%%%%%%%%%%%%%%%%%%%%%%%%%%%%%%%%
  \begin{equation}
    \left[t^k h(t)\right]_0^{\infty}
      = \lim_{T\to\infty} T^k h(T) - 0
      = \lim_{T\to\infty}{\left(k \int_0^T t^{k-1} dt\right) h(T)}
      = k \int_0^{\infty}{t^{k-1} h(\infty) dt}.
    \label{eq:XUVTX}
  \end{equation}
  %%%%%%%%%%%%%%%%%%%%%%%%%%%%%%%%%%%%%%%%%%%%%%%%%%%%%%%%%%%%%%%%%%%%%%%%%%%%%
  Po wstawieniu \eqref{eq:XUVTX} z~powrotem do \eqref{eq:MRW8Q} otrzymuje się
  %%%%%%%%%%%%%%%%%%%%%%%%%%%%%%%%%%%%%%%%%%%%%%%%%%%%%%%%%%%%%%%%%%%%%%%%%%%%%
  \begin{equation}
    \begin{aligned}
      & m_k = k \int_0^{\infty}{t^{k-1} (h(\infty) - h(t)) dt}, &
      & \text{dla } k=1,\dots, \infty &
    \end{aligned}
    \label{eq:CNY34}
  \end{equation}
  %%%%%%%%%%%%%%%%%%%%%%%%%%%%%%%%%%%%%%%%%%%%%%%%%%%%%%%%%%%%%%%%%%%%%%%%%%%%%

  Ponieważ posługujemy się całkami niewłaściwymi, istotną kwestią jest
  zapewnienie ich zbieżności. W~tym przypadku zbieżność zależy od $h(t)$.

  \paragraph{Przykład} Dla przykładowej funkcji $h(t)=1/(1+t)$ całka $m_2$
  byłaby rozbieżna:
  %%%%%%%%%%%%%%%%%%%%%%%%%%%%%%%%%%%%%%%%%%%%%%%%%%%%%%%%%%%%%%%%%%%%%%%%%%%%%
  \begin{equation}
    m_2 = 2 \int_0^{\infty}{t\cdot\left(0 - \frac{1}{1+t}\right)dt}
        = 2 \left[\ln{|t+1| - t}\right]_0^{\infty}
        = -\infty
    \label{eq:VBGL6}
  \end{equation}
  %%%%%%%%%%%%%%%%%%%%%%%%%%%%%%%%%%%%%%%%%%%%%%%%%%%%%%%%%%%%%%%%%%%%%%%%%%%%%

  \paragraph{Przykład} Dla przykładu wyznaczmy wartości momentów $m_k$ dla
  członu inercyjnego pierwszego rzędu. Odpowiedź skokowa $h(t)$ członu
  inercyjnego o~stałej czasowej $T$ (wzmocnienie $k=1$), to
  %%%%%%%%%%%%%%%%%%%%%%%%%%%%%%%%%%%%%%%%%%%%%%%%%%%%%%%%%%%%%%%%%%%%%%%%%%%%%
  \begin{equation}
    \begin{aligned}
      & h(t) = 1 - e^{-t/T}, && h(\infty) = 1. &
    \end{aligned}
  \end{equation}
  %%%%%%%%%%%%%%%%%%%%%%%%%%%%%%%%%%%%%%%%%%%%%%%%%%%%%%%%%%%%%%%%%%%%%%%%%%%%%
  Odpowiednio do tego, wartości momentów $m_k$ dla członu inercyjnego I-go
  rzędu, to
  %%%%%%%%%%%%%%%%%%%%%%%%%%%%%%%%%%%%%%%%%%%%%%%%%%%%%%%%%%%%%%%%%%%%%%%%%%%%%
  \begin{equation}
    \begin{aligned}
      & m_0 = 1,   &
      & m_1 = \int_0^{\infty}{e^{-t/T} dt} = T,   &
      & m_2 = 2 \int_0^{\infty} {t e^{-t/T}dt} = 2 T^2. &
    \end{aligned}
  \end{equation}
  %%%%%%%%%%%%%%%%%%%%%%%%%%%%%%%%%%%%%%%%%%%%%%%%%%%%%%%%%%%%%%%%%%%%%%%%%%%%%
  Dla członu różniczkującego rzeczywistego nie ma jakościowej różnicy, ponieważ
  jego odpowiedź $h(t) = e^{-t/T}$, $h(\infty) = 0$, zatem momenty $m_k$ będą
  jedynie różniły się znakiem. Istotnym wnioskiem jest, że dla tych układów,
  całki $m_k$ są zbieżne. Podobne wnioskowanie można przeprowadzić dla badanych
  układów drugiego rzędu.

  \subsubsection{Numeryczne wyznaczanie momentów \texorpdfstring{$m_k$}{mk}}
  \label{eq:MTP2Q}

  Niech $T > 0$ będzie okresem obserwacji układu, zaś
  %%%%%%%%%%%%%%%%%%%%%%%%%%%%%%%%%%%%%%%%%%%%%%%%%%%%%%%%%%%%%%%%%%%%%%%%%%%%%
  \begin{equation}
    \hat{h}(t) = h(\infty) - h(t).
    \label{eq:7S9M3}
  \end{equation}
  %%%%%%%%%%%%%%%%%%%%%%%%%%%%%%%%%%%%%%%%%%%%%%%%%%%%%%%%%%%%%%%%%%%%%%%%%%%%%
  Całkę we wzorze \eqref{eq:CNY34} można rozdzielić na dwie części:
  %%%%%%%%%%%%%%%%%%%%%%%%%%%%%%%%%%%%%%%%%%%%%%%%%%%%%%%%%%%%%%%%%%%%%%%%%%%%%
  \begin{equation}
    \underbrace{\int_0^{\infty} t^{k-1}\hat{h}(t) dt}_{M_k(\infty)}
      = \underbrace{\int_0^{T} t^{k-1}\hat{h}(t) dt}_{M_k(T)}
      + \underbrace{\int_{T}^{\infty} t^{k-1}\hat{h}(t) dt}_{R_k(T)}
    \label{eq:GGFI8}
  \end{equation}
  %%%%%%%%%%%%%%%%%%%%%%%%%%%%%%%%%%%%%%%%%%%%%%%%%%%%%%%%%%%%%%%%%%%%%%%%%%%%%
  a~następnie przyjąć, że w~przypadku zbieżności całki $M_k(\infty)$ zachodzi
  również
  %%%%%%%%%%%%%%%%%%%%%%%%%%%%%%%%%%%%%%%%%%%%%%%%%%%%%%%%%%%%%%%%%%%%%%%%%%%%%
  \begin{equation}
    \lim_{T \to \infty}\int_T^{\infty}{t^{k-1} \hat{h}(t) dt} = 0.
    \label{eq:DITCB}
  \end{equation}
  %%%%%%%%%%%%%%%%%%%%%%%%%%%%%%%%%%%%%%%%%%%%%%%%%%%%%%%%%%%%%%%%%%%%%%%%%%%%%
  W~ten sposób uznajemy, że wyraz $R_k(T)$ w~\eqref{eq:GGFI8} traci na znaczeniu
  ze wzrostem~$T$. Co za tym idzie, dla odpowiednio dużych $T$ możemy przyjąć
  całkę $M_k(T)$ w~skończonym przedziale $[0,T]$ za~przybliżenie całki
  $M_k(\infty)$ w~przedziale nieskończonym $[0,\infty)$:
  %%%%%%%%%%%%%%%%%%%%%%%%%%%%%%%%%%%%%%%%%%%%%%%%%%%%%%%%%%%%%%%%%%%%%%%%%%%%%
 \begin{equation}
   \begin{aligned}
      &M_k(\infty) \approx M_k(T), && \text{dla } T \gg 0.&
   \end{aligned}
   \label{eq:MZ4XJ}
 \end{equation}
  %%%%%%%%%%%%%%%%%%%%%%%%%%%%%%%%%%%%%%%%%%%%%%%%%%%%%%%%%%%%%%%%%%%%%%%%%%%%%

  Niech $m > 0$ będzie liczbą całkowitą oraz $\tau:  T = \tau m$. Przyjmijmy, że
  %%%%%%%%%%%%%%%%%%%%%%%%%%%%%%%%%%%%%%%%%%%%%%%%%%%%%%%%%%%%%%%%%%%%%%%%%%%%%
  \begin{equation}
    \begin{aligned}
      & t = \tau x, && x > 0.&
    \end{aligned}
    \label{eq:J7GYY}
  \end{equation}
  %%%%%%%%%%%%%%%%%%%%%%%%%%%%%%%%%%%%%%%%%%%%%%%%%%%%%%%%%%%%%%%%%%%%%%%%%%%%%
  Dokonując zamiany zmiennych, całkę $M_k(T)$ we~wzorze~\eqref{eq:GGFI8} można
  zapisać jako:
  %%%%%%%%%%%%%%%%%%%%%%%%%%%%%%%%%%%%%%%%%%%%%%%%%%%%%%%%%%%%%%%%%%%%%%%%%%%%%
  \begin{equation}
    M_k(T) = \int_0^m{(\tau x)^{k-1} \hat{h}(\tau x) d(\tau x)}
           = \tau^k \int_0^m{x^{k-1} \hat{h}(\tau x) dx}.
    \label{eq:R2WL9}
  \end{equation}
  %%%%%%%%%%%%%%%%%%%%%%%%%%%%%%%%%%%%%%%%%%%%%%%%%%%%%%%%%%%%%%%%%%%%%%%%%%%%%
  Całkę we wzorze \eqref{eq:R2WL9} w~granicach $[0,m]$ można dalej wyrazić jako
  sumę całek częściowych w~granicach $[i,i+1]$, dla $i=0,\dots,m-1$
  %%%%%%%%%%%%%%%%%%%%%%%%%%%%%%%%%%%%%%%%%%%%%%%%%%%%%%%%%%%%%%%%%%%%%%%%%%%%%
  \begin{equation}
    M_k(T) = \tau^k \sum_{i=0}^{m-1} \int_i^{i+1} x^{k-1}\hat{h}(\tau x) dx,
    \label{eq:NP3NQ}
  \end{equation}
  %%%%%%%%%%%%%%%%%%%%%%%%%%%%%%%%%%%%%%%%%%%%%%%%%%%%%%%%%%%%%%%%%%%%%%%%%%%%%
  a następnie, przyjmując $x = i + \xi$, $\xi \in [0,1]$:
  %%%%%%%%%%%%%%%%%%%%%%%%%%%%%%%%%%%%%%%%%%%%%%%%%%%%%%%%%%%%%%%%%%%%%%%%%%%%%
  \begin{equation}
    M_k(T) = \tau^k \sum_{i=0}^{m-1} \int_0^1 (i + \xi)^{k-1}\hat{h}(\tau (i + \xi)) d\xi.
    \label{eq:O5GB0}
  \end{equation}
  %%%%%%%%%%%%%%%%%%%%%%%%%%%%%%%%%%%%%%%%%%%%%%%%%%%%%%%%%%%%%%%%%%%%%%%%%%%%%

  W~dalszym postępowaniu przyjmujemy, że znane są jedynie dyskretne wartości
  $\hat{h}_i = \hat{h}(\tau i)$ funkcji $\hat{h}(t)$ w~równomiernie
  rozmieszczonych punktach $t_i = \tau i$
  %%%%%%%%%%%%%%%%%%%%%%%%%%%%%%%%%%%%%%%%%%%%%%%%%%%%%%%%%%%%%%%%%%%%%%%%%%%%%
  \begin{equation}
    \begin{aligned}
      & \hat{h}_i = \hat{h}(\tau i), && i = 0, 1, \dots, m.&
    \end{aligned}
    \label{eq:0XLVC}
  \end{equation}
  %%%%%%%%%%%%%%%%%%%%%%%%%%%%%%%%%%%%%%%%%%%%%%%%%%%%%%%%%%%%%%%%%%%%%%%%%%%%%

  \paragraph{Metoda trapezów} W~tej metodzie zakłada się odcinkową postać
  funkcji całkowanej
  %%%%%%%%%%%%%%%%%%%%%%%%%%%%%%%%%%%%%%%%%%%%%%%%%%%%%%%%%%%%%%%%%%%%%%%%%%%%%
  \begin{equation}
    x^{k-1} \hat{h}(\tau x),
    \label{eq:0MSTS}
  \end{equation}
  %%%%%%%%%%%%%%%%%%%%%%%%%%%%%%%%%%%%%%%%%%%%%%%%%%%%%%%%%%%%%%%%%%%%%%%%%%%%%
  zastępując ją funkcją $f(x)$
  %%%%%%%%%%%%%%%%%%%%%%%%%%%%%%%%%%%%%%%%%%%%%%%%%%%%%%%%%%%%%%%%%%%%%%%%%%%%%
  \begin{equation}
    f(x) = \begin{cases}
      f_i & \text{dla } x = i, \\
      f_i \cdot (1 - (x-i)) + f_{i+1} \cdot (x-i) & \text{dla } x \in (i, i+1),
    \end{cases}
  \end{equation}
  %%%%%%%%%%%%%%%%%%%%%%%%%%%%%%%%%%%%%%%%%%%%%%%%%%%%%%%%%%%%%%%%%%%%%%%%%%%%%
  gdzie $f_i = i^{k-1} \hat{h}_i$. Zobrazowano to na rysunku~\ref{fig:Z319K}.
  %%%%%%%%%%%%%%%%%%%%%%%%%%%%%%%%%%%%%%%%%%%%%%%%%%%%%%%%%%%%%%%%%%%%%%%%%%%%%
  \begin{figure}[H]
    \centering
    \input{lpas10/trapezoid.tex}
    \caption{Zasada aproksymacji funkcji $f(t)$ w~metodzie trapezów}
    \label{fig:Z319K}
  \end{figure}
  %%%%%%%%%%%%%%%%%%%%%%%%%%%%%%%%%%%%%%%%%%%%%%%%%%%%%%%%%%%%%%%%%%%%%%%%%%%%%
  Całkę $M_k(T)$ aproksymuje się więc następująco (zakładając $\hat{h}_m = 0$):
  %%%%%%%%%%%%%%%%%%%%%%%%%%%%%%%%%%%%%%%%%%%%%%%%%%%%%%%%%%%%%%%%%%%%%%%%%%%%%
  \begin{equation}
    \begin{aligned}
      & M_k(T) \approx \tau^k \sum_{i=0}^{m-1} {\int_0^1 {\left(
        f_i \cdot (1 - \xi) + f_{i+1}\cdot \xi
      \right) d\xi}} = \tau^k \sum_{i=1}^{m-1} {i^{k-1} \hat{h}_i} &
      & \text{dla } k=1,\dots &
    \end{aligned}
    \label{eq:HNVFS}
  \end{equation}
  %%%%%%%%%%%%%%%%%%%%%%%%%%%%%%%%%%%%%%%%%%%%%%%%%%%%%%%%%%%%%%%%%%%%%%%%%%%%%
  Uwzględniając \eqref{eq:HNVFS} w~\eqref{eq:CNY34} otrzymujemy
  %%%%%%%%%%%%%%%%%%%%%%%%%%%%%%%%%%%%%%%%%%%%%%%%%%%%%%%%%%%%%%%%%%%%%%%%%%%%%
  \begin{equation}
    \begin{aligned}
      & m_k \approx k \tau^k \sum_{i=1}^{m-1} {i^{k-1} \hat{h}_i} &
      & \text{dla } k = 1,\dots &
    \end{aligned}
    \label{eq:YKY27}
  \end{equation}
  %%%%%%%%%%%%%%%%%%%%%%%%%%%%%%%%%%%%%%%%%%%%%%%%%%%%%%%%%%%%%%%%%%%%%%%%%%%%%
  a~następnie
  %%%%%%%%%%%%%%%%%%%%%%%%%%%%%%%%%%%%%%%%%%%%%%%%%%%%%%%%%%%%%%%%%%%%%%%%%%%%%
  \begin{equation}
    \begin{aligned}
      & w_k \approx \frac{(-1)^k}{(k-1)!} \tau^k \sum_{i=1}^{m-1} {i^{k-1} \hat{h}_i} &
      & \text{dla } k = 1,\dots &
    \end{aligned}
  \end{equation}
  %%%%%%%%%%%%%%%%%%%%%%%%%%%%%%%%%%%%%%%%%%%%%%%%%%%%%%%%%%%%%%%%%%%%%%%%%%%%%
\end{appendices}

%% \bibliographystyle{unsrtnat}
%% \bibliography{lpas}

\end{document}

%\label{?:CJQM7}
%\label{?:HU8MI}
%\label{?:AW332}
%\label{?:5VQVG}
%\label{?:PO024}
%\label{?:YGQLA}
%\label{?:XZDX0}
%\label{?:AHW5W}
%\label{?:N5LK3}
%\label{?:R3TWE}
%\label{?:DA2GA}
%\label{?:3DK88}
%\label{?:ZQ8OX}
%\label{?:6U6WZ}
%\label{?:H872F}
%\label{?:DT8PT}
%\label{?:QY1BQ}
%\label{?:RX41L}
%\label{?:W9N0W}
%\label{?:PBHHG}
%\label{?:K949U}
%\label{?:33EHI}
%\label{?:DU5ZO}
%\label{?:CDSBF}
%\label{?:OVAJA}
%\label{?:1KZYI}
%\label{?:5H7AS}
%\label{?:AIERO}
%\label{?:4J0VS}
%\label{?:0CM1T}
%\label{?:ZGCWY}
%\label{?:KY270}
%\label{?:MSTSV}
%\label{?:BGL6F}
%\label{?:EPSUO}
%\label{?:5GB0N}
%\label{?:P3NQ0}
%\label{?:XLVCC}
%\label{?:V1AT7}
%\label{?:S9M3M}
%\label{?:TP2Q6}
%\label{?:V264C}
%\label{?:NY34B}
%\label{?:RN89W}
%\label{?:0DJCX}
%\label{?:UVTXM}
%\label{?:RW8QQ}
%\label{?:1UWQ7}
%\label{?:RDBZR}
%\label{?:5YS19}
%\label{?:XWOO7}
%\label{?:CL5I4}
%\label{?:LX5XV}
%\label{?:R42H6}
%\label{?:EIFNZ}
%\label{?:0EFM2}
%\label{?:PAK7B}
%\label{?:PUN96}
%\label{?:658YT}
%\label{?:YJ2R3}
%\label{?:571J3}
%\label{?:ODA2G}
%\label{?:A3DK8}
%\label{?:8ZQ8O}
%\label{?:X6U6W}
%\label{?:ZH872}
%\label{?:FDT8P}
%\label{?:TQY1B}
%\label{?:QRX41}
%\label{?:LW9N0}
%\label{?:WPBHH}
%\label{?:GK949}
%\label{?:U33EH}
%\label{?:IDU5Z}
%\label{?:OCDSB}
%\label{?:FOVAJ}
%\label{?:A1KZY}
%\label{?:I5H7A}
%\label{?:SAIER}
%\label{?:O4J0V}
%\label{?:S0CM1}
%\label{?:TZGCW}

% vim: set syntax=tex tabstop=2 shiftwidth=2 expandtab spell spelllang=pl:
