\documentclass[paper=a4,DIV=12]{lpas}

\usepackage[polish]{babel}
\usepackage[T1]{fontenc}
\usepackage[utf8]{inputenc}
\usepackage{courier}
\usepackage{tgtermes,newtxtext,newtxmath}
\usepackage[]{hyperref}
\usepackage{natbib}

%%\everymath{\displaystyle}

\usepackage{bm}
\usepackage{amsmath,amsfonts}
\usepackage{mathtools}
\usepackage{graphicx}
\usepackage[titletoc,title]{appendix}
\usepackage{subcaption}
\usepackage{listings}
\usepackage{float}
\usepackage{tabularx}
\usepackage{tikz}
\usetikzlibrary{arrows,3d,patterns,calc}

\newcommand{\brm}[1]{\bm{\mathrm{#1}}}
\renewcommand{\arraystretch}{1.2}
\newcolumntype{L}[1]{>{\raggedright\arraybackslash}p{#1}}
\newcolumntype{C}[1]{>{\centering\arraybackslash}p{#1}}
\newcolumntype{R}[1]{>{\raggedleft\arraybackslash}p{#1}}

% commath provides \od, but the package is not available on my travis-ci setup
\newcommand{\od}[2]{\frac{\mathrm{d}#1}{\mathrm{d}#2}}
\newcommand{\odn}[3]{\frac{\mathrm{d}^{#1}#2}{\mathrm{d}{#3}^{#1}}}
\newcommand{\tod}[2]{\tfrac{\mathrm{d}#1}{\mathrm{d}#2}}
\newcommand{\todn}[3]{\tfrac{\mathrm{d}^{#1}#2}{\mathrm{d}{#3}^{#1}}}
% gensymb provides \degree command, but whole package for just one symbol?
\newcommand{\degree}{^{\circ}}

\lstset{%
  basicstyle=\footnotesize\ttfamily\selectfont,
  language=Matlab,
  inputencoding=utf8,
  extendedchars=true,
  frame=trBL
}

\newfloat{lstfloat}{htbp}{lop}
\floatname{lstfloat}{Listing}

\setcitestyle{numbers,square,comma}

\begin{document}


\serietitle{\\{\ }\\{\ }\\Laboratorium pomiarów, automatyki i~sterowania~I}

\title{\Large{Ćwiczenie nr 4}}

\subtitle{\huge{Analiza kinematyczna układu o dwóch stopniach swobody}}

\author{\\Paweł Tomulik\\ Zakład Teorii Maszyn i Robotów\\ ITLiMS, MEIL, PW}
\date{}
\maketitle
\thispagestyle{empty}

\pagebreak
\tableofcontents
\pagebreak

\begin{abstract}
\section{Streszczenie}
\noindent Wyznaczanie współrzędnych układu mechanicznego z~użyciem czujników znajduje
  wiele zastosowań. Jednym z~nich są systemy odometrii używane m.in. w~robotach
  mobilnych. Typowe zadanie odometrii polega na estymacji położenia
  i~orientacji kątowej robota poprzez analizę danych zebranych z~różnych
  czujników (np. czujniki przyspieszenia, enkodery, czujniki odległości, układy
  wizyjne). W~niniejszym ćwiczeniu podejmuje się próbę wyznaczenia przebiegów
  przyspieszeń, położeń i~prędkości układu o~dwu stopniach swobody na podstawie
  danych zebranych z~czujników przyspieszenia i~położenia. Przedstawiony
  przykład jest pouczający, pozwala zaobserwować typowe problemy wynikające
  z~nakładania się na siebie błędów metody pomiarowej oraz użytych metod
  obliczeniowych (całkowanie czy różniczkowanie numeryczne).
\end{abstract}

\section{Wprowadzenie}
\label{sec:BP9CM}

Obiektem badanym jest wahadło umocowane w~jednym końcu za pomocą przegubu
Cardana. Schemat kinematyczny układu pokazano na rysunku~\ref{fig:7YP7N}.
%%%%%%%%%%%%%%%%%%%%%%%%%%%%%%%%%%%%%%%%%%%%%%%%%%%%%%%%%%%%%%%%%%%%%%%%%%%%%%
\begin{figure}[htbp]
  \centering
  \begin{tikzpicture}[%
      scale=1.0,%
      axis/.style={thin,->},%
      zxplane/.style={canvas is zx plane at y=#1},%
      yxplane/.style={canvas is yx plane at z=#1},%
      yzplane/.style={canvas is yz plane at x=#1},%
  ]
    \coordinate (O) at (0,0,0);

    \begin{scope}[x={(-135:1cm)}, y={( 0:1cm)}, z={(90:1cm)}]
      \draw[axis] (0.0, 0.0, 0.0) -- (2,0,0) node [left] {$x$};
      \draw[axis] (0.0, 0.0, 0,0) -- (0,2,0) node [right] {$y$};
      \draw[axis] (0.0, 0.0, 0.0) -- (0,0,2) node [right] {$z$};
      \draw[axis,yzplane=1.8] (0,0) ++ (-120:.3) arc(-120:120:.3) node [above] {$\varphi_x$};
      \begin{scope}[rotate around x=10]
        % axis "1" (rotation alpha_x)
        \draw[thick] (-1.05, 0, 0) -- (1.05, 0, 0);

        % details "1"
        \draw[thick] (-1, 0, 0.1) -- (-0.7, 0, 0.1);
        \draw[thick] (-1, 0,-0.1) -- (-0.7, 0,-0.1);
        \draw[thick] (-1.05, 0,-0.2) -- (-1.05, 0, 0.2);
        \draw[thick] (-0.65, 0,-0.2) -- (-0.65, 0, 0.2);

        \draw[thick] ( 1, 0, 0.1) -- ( 0.7, 0, 0.1);
        \draw[thick] ( 1, 0,-0.1) -- ( 0.7, 0,-0.1);
        \draw[thick] ( 1.05, 0,-0.2) -- ( 1.05, 0, 0.2);
        \draw[thick] ( 0.65, 0,-0.2) -- ( 0.65, 0, 0.2);

        \draw[draw=none,fill=black,zxplane=0] ( 0.1,-1) rectangle ( 0.2,-0.7);
        \draw[draw=none,fill=black,zxplane=0] (-0.1,-1) rectangle (-0.2,-0.7);
        \draw[draw=none,fill=black,zxplane=0] ( 0.1, 1) rectangle ( 0.2, 0.7);
        \draw[draw=none,fill=black,zxplane=0] (-0.1, 1) rectangle (-0.2, 0.7);

        \draw[axis,zxplane=1.8] (0,0) ++ (-120:.3) arc(-120:120:.3) node [midway, above] {$\varphi_y$};

        \begin{scope}[rotate around y=20]
          % axis "2" (rotation alpha_y)
          \draw[thick](0,-1.05, 0) -- (0, 1.05, 0);

          % details "2"
          \draw[thick](0,-1,-0.1) -- (0,-0.7,-0.1);
          \draw[thick](0,-1, 0.1) -- (0,-0.7, 0.1);
          \draw[thick](0,-1.05,-0.2) -- (0,-1.05, 0.2);
          \draw[thick](0,-0.65,-0.2) -- (0,-0.65, 0.2);

          \draw[thick](0, 1,-0.1) -- (0, 0.7,-0.1);
          \draw[thick](0, 1, 0.1) -- (0, 0.7, 0.1);
          \draw[thick](0, 1.05,-0.2) -- (0, 1.05, 0.2);
          \draw[thick](0, 0.65,-0.2) -- (0, 0.65, 0.2);

          \draw[thick](0,-0.85,-0.1) -- (0,-0.85,-0.3) -- (0,0.85,-0.3) -- (0,0.85,-0.1);

          \draw[very thick,-*](0, 0, -0.3) -- (0, 0, -6.0) node [right] {$r$};

          \draw[axis] ( 0.0, 0.0, 0.0) -- (2,0,0) node [left] {$x^{\prime}$};
          \draw[axis] ( 0.0, 0.0, 0,0) -- (0,2,0) node [right] {$y^{\prime}$};
          \draw[axis] ( 0.0, 0.0, 0.0) -- (0,0,2) node [right] {$z^{\prime}$};
        \end{scope}
      \end{scope}
    \end{scope}
  \end{tikzpicture}
  \caption{Schemat kinematyczny wahadła.}
  \label{fig:7YP7N}
\end{figure}
%%%%%%%%%%%%%%%%%%%%%%%%%%%%%%%%%%%%%%%%%%%%%%%%%%%%%%%%%%%%%%%%%%%%%%%%%%%%%%
Na schemacie zaznaczono dwa układy współrzędnych:
\begin{itemize}
  \item nieruchomy układ $x,y,z$ ze środkiem w~centrum przegubu Cardana, osie
        $x$ i~$y$ biegną poziomo, przy czym oś $x$ jest równoległa do ściany
        laboratorium a~oś $y$ jest prostopadła do ściany (grot skierowany
        w~kierunku sali), oś $z$ skierowana jest pionowo w~górę,
  \item ruchomy układ $x^{\prime}, y^{\prime}, z^{\prime}$, ze środkiem w~centrum przegubu
        Cardana, układ jest przytwierdzony do drążka wahadła (ruchoma część)
        tak, że oś $z^{\prime}$ pokrywa się z~linią drążka wahadła, zaś oś $y^{\prime}$
        biegnie wzdłuż osi pary obrotowej krzyżak-drążek.
\end{itemize}
W~położeniu swobodnym (nieodchylonym, $\varphi_x = \varphi_y=0$) układ
ruchomy pokrywa się z~nieruchomym.

Punkt na końcówce drążka wahadła oznaczono jako $r$. Jednocześnie
$r$ będzie oznaczać długość wahadła ($r = 0.66m$).


\section{Układ pomiarowy}
\label{sec:Z2CJY}

W~osiach $x$ i~$y^{\prime}$ zamocowane są potencjometry umożliwiające pomiar
(pośredni) wychyleń kątowych $\varphi_x$ i~$\varphi_y$ zaznaczonych na
rysunku~\ref{fig:7YP7N}. Zmierzone potencjometrami wartości kątowe będziemy
oznaczali symbolami $\bar{\varphi}_x$, $\bar{\varphi}_y$ Na~końcówce drążka
wahadła zamocowany jest akcelerometr, który umożliwia pomiar (pośredni)
chwilowych przyspieszeń liniowych końcówki wahadła (we wszystkich trzech
osiach, $x^{\prime},y^{\prime},z^{\prime}$). Przyspieszenia wskazywane przez
akcelerometr oznaczymy symbolem $\hat{\brm{a}}^{\prime} = \begin{bmatrix}
\hat{a}_x^{\prime} & \hat{a}_y^{\prime} & \hat{a}_z^{\prime} \end{bmatrix}^T$.
Dedykowany program komputerowy, wraz z~kartą pomiarową, umożliwiają zebranie
pomiarów w~krótkich odstępach czasowych i~zapisanie przebiegów do pliku.

Wahadło jest też wyposażone w~kątomierze umożliwiające odczyt wizualny wychyleń
kątowych $\varphi_x$ i $\varphi_y$ w~warunkach statycznych. Zakres kątomierzy
$\mp 30^{\circ}$.

\section{Skalowanie czujników}
\label{eq:NC01F}

{\em Przed przystąpieniem do skalowania czujników należy włączyć układ
pomiarowy i~uruchomić program komputerowy (program do skalowania czujników)}.

Dane zebrane z~czujników są dostarczone w~postaci napięć wyrażonych w Voltach.
Przed przy\-stą\-pie\-niem do~obliczeń należy je przeliczyć na wielkości
w~odpowiednich jednostkach -- radianach dla odchyleń kątowych oraz
$\tfrac{m}{s^2}$ dla przyspieszeń. Zakłada się, że czujniki mają
charakterystyki liniowe. Dla określenia charakterystyki danego czujnika
powinno więc wystarczyć znalezienie dwóch jej punktów\footnote{W~rzeczywistości
należałoby jednak zrobić pomiar w~wielu punktach i~zastosować regresję
liniową. W~przypadku akcelerometru, wymaga to zastosowania specjalnego
stanowiska do kalibracji.}.

\subsection{Skalowanie czujników wychylenia (potencjometry)}
\label{eq:460IS}

Dla każdego potencjometru należy zmierzyć
napięcie wyjściowe w~położeniach wahadła $-30^{\circ}$, $0^{\circ}$
i~$+30^{\circ}$, oznaczone dalej jako $U_{\bullet}^{-30^{\circ}}$,
$U_{\bullet}^{0^{\circ}}$, $U_{\bullet}^{+30^{\circ}}$ ($\bullet=x$ dla kanału
$\varphi_x$ oraz $y$ dla $\varphi_y$). Odczytane wartości należy zebrać
w~tabeli, np. takiej jak tablica~\ref{tab:KM4WU}.
%%%%%%%%%%%%%%%%%%%%%%%%%%%%%%%%%%%%%%%%%%%%%%%%%%%%%%%%%%%%%%%%%%%%%%%%%%%%%%
\begin{table}[htbp]
  \caption{Skalowanie czujników położenia (potencjometrów).}
  \label{tab:KM4WU}
  \centering
  \begin{tabular}{|l|c|c|c|}
    \hline
                                  & Kanał $x$ & Kanał $y$ \\ \hline
      $U_{\bullet}^{0^{\circ}}$   &           &           \\ \hline
      $U_{\bullet}^{-30^{\circ}}$ &           &           \\ \hline
      $U_{\bullet}^{+30^{\circ}}$ &           &           \\ \hline
      $C_{\bullet}$               &           &           \\ \hline
  \end{tabular}
\end{table}
%%%%%%%%%%%%%%%%%%%%%%%%%%%%%%%%%%%%%%%%%%%%%%%%%%%%%%%%%%%%%%%%%%%%%%%%%%%%%%

Nachylenia charakterystyk potencjometrów (oś $x$ i~$y^{\prime}$) wyrazimy poprzez
współczynniki
%%%%%%%%%%%%%%%%%%%%%%%%%%%%%%%%%%%%%%%%%%%%%%%%%%%%%%%%%%%%%%%%%%%%%%%%%%%%%%
\begin{align}
  & C_x = \frac{\pi/3}{U_x^{+30^{\circ}} - U_x^{-30^{\circ}}}, &
  & C_y = \frac{\pi/3}{U_y^{+30^{\circ}} - U_y^{-30^{\circ}}}. &
  \label{eq:XIRR2}
\end{align}
%%%%%%%%%%%%%%%%%%%%%%%%%%%%%%%%%%%%%%%%%%%%%%%%%%%%%%%%%%%%%%%%%%%%%%%%%%%%%%
Zmierzone wartości wychyleń kątowych wyrażone w~radianach będziemy zatem liczyli jako
%%%%%%%%%%%%%%%%%%%%%%%%%%%%%%%%%%%%%%%%%%%%%%%%%%%%%%%%%%%%%%%%%%%%%%%%%%%%%%
\begin{align}
  &\bar{\varphi}_x = C_x \left(U_x - U_x^{0^{\circ}}\right),&
  &\bar{\varphi}_y = C_y \left(U_y - U_y^{0^{\circ}}\right),&
  \label{eq:547XS}
\end{align}
%%%%%%%%%%%%%%%%%%%%%%%%%%%%%%%%%%%%%%%%%%%%%%%%%%%%%%%%%%%%%%%%%%%%%%%%%%%%%%
gdzie $U_x$ i~$U_y$ są chwilowymi wartościami napięć zmierzonych
na~potencjometrach. Można też próbować używać użyć w~charakterze
$U_{\bullet}^{0^{\circ}}$ wartości średniej $U_{\bullet}^{0^{\circ}} =
\tfrac{1}{2}\left(U_{\bullet}^{-30^{\circ}} + U_{\bullet}^{+30^{\circ}}\right)$
jak również zastosować regresję liniową wg trzech punktów $-30^{\circ},
0^{\circ}, +30^{\circ}$ do wyznaczenia współczynników charakterystyki.

\subsection{Skalowanie akcelerometru}
\label{sec:DHEMO}

Skalowanie akcelerometru można przeprowadzić na kilka sposobów. Dwa z~nich
przedstawiono w~podrozdziałach \ref{sec:CT4YN} i~\ref{sec:BDRBE}. W~pierwszej
metodzie odpowiednie czynności wykonuje się poza wahadłem. Wymontowany
akcelerometr ustawia się w~położeniach~$\mp 1g$, tzn.  ustawia się kolejne jego
osie pionowo raz w~dół raz w~górę mierząc odpowiadające tym położeniom wartości
napięć generowanych przez akcelerometr.  W~drugiej metodzie akcelerometr jest
zamontowany na końcówce wahadła. Odczytuje się wartości wskazywane przez
akcelerometr w~dwóch różnych położeniach wahadła (w~bezruchu). Skalowanie można
przeprowadzić przed właściwym eksperymentem, jak również można wykonać
,,post-factum'' na podstawie pomiarów z~eksperymentu, przy spełnieniu
odpowiednich warunków.

\subsubsection{Skalowanie metodą $\mp 1g$}
\label{sec:CT4YN}

W~celu wyskalowania akcelerometru należy go zdemontować z~końcówki wahadła.
Swobodnym, aczkolwiek podłączonym do~układu pomiarowego akcelerometrem
obracamy ,,w~powietrzu'' starając się ustawić go w~położeniach przedstawionych
na rysunku~\ref{fig:ZDELS}.
%%%%%%%%%%%%%%%%%%%%%%%%%%%%%%%%%%%%%%%%%%%%%%%%%%%%%%%%%%%%%%%%%%%%%%%%%%%%%%
\begin{figure}[htbp]
  \centering
  \begin{tikzpicture}[%
      scale=1.0,%
      axis/.style={thin,->},%
      zxplane/.style={canvas is zx plane at y=#1},%
      yxplane/.style={canvas is yx plane at z=#1},%
      yzplane/.style={canvas is yz plane at x=#1},%
  ]
    % gravity
    \draw[very thick,->] (-2.8,-2) -- + (0, -1) node [left] {$\brm{g}$};
    \draw[very thick,->] (-2.6,-2) -- + (0, -1);
    \draw[very thick,->] (-2.4,-2) -- + (0, -1);

    \node[draw=none] at (0,1.75) {$V_x^{0g}, V_y^{0g}, V_z^{+1g}$};

    \begin{scope}[x={(-135:1cm)}, y={( 0:1cm)}, z={(90:1cm)}]
      \draw[very thick,fill=gray,yxplane=0] (-1,-1) rectangle (1,1);
      \draw[axis] (0,0,0) -- +(2,0,0) node [left] {$x^{\prime}$};
      \draw[axis] (0,0,0) -- +(0,2,0) node [above] {$y^{\prime}$};
      \draw[axis] (0,0,0) -- +(0,0,1) node [right] {$z^{\prime}$};
    \end{scope}

    \begin{scope}[shift={(5,0)}]
      \node[draw=none] at (0,1.75) {$V_x^{-1g}, V_y^{0g}, V_z^{0g}$};
      \begin{scope}[x={(-135:1cm)}, y={( 0:1cm)}, z={(90:1cm)}]
        \begin{scope}[rotate around y=90]
          \draw[very thick,fill=gray,yxplane=0] (-1,-1) rectangle (1,1);
          \draw[axis] (0,0,0) -- +(2,0,0) node [left] {$x^{\prime}$};
          \draw[axis] (0,0,0) -- +(0,2,0) node [above] {$y^{\prime}$};
          \draw[axis] (0,0,0) -- +(0,0,1) node [above] {$z^{\prime}$};
        \end{scope}
      \end{scope}
    \end{scope}

    \begin{scope}[shift={(10,0)}]
      \node[draw=none] at (1.25,1.75) {$V_x^{+1g}, V_y^{0g}, V_z^{0g}$};
      \begin{scope}[x={(-135:1cm)}, y={( 0:1cm)}, z={(90:1cm)}]
        \begin{scope}[rotate around y=-90]
          \draw[very thick,fill=gray,yxplane=0] (-1,-1) rectangle (1,1);
          \draw[axis] (0,0,0) -- +(2,0,0) node [left] {$x^{\prime}$};
          \draw[axis] (0,0,0) -- +(0,2,0) node [above] {$y^{\prime}$};
          \draw[axis,dashed] (0,0,0) -- +(0,0,1) node [above] {$z^{\prime}$};
        \end{scope}
      \end{scope}
    \end{scope}

    \begin{scope}[shift={(0,-5)}]
      \node[draw=none] at (-0.75,1.5) {$V_x^{0g}, V_y^{-1g}, V_z^{0g}$};
      \begin{scope}[x={(-135:1cm)}, y={( 0:1cm)}, z={(90:1cm)}]
        \begin{scope}[rotate around x=-90]
          \draw[very thick,fill=gray,yxplane=0] (-1,-1) rectangle (1,1);
          \draw[axis] (0,0,0) -- +(2,0,0) node [left] {$x^{\prime}$};
          \draw[axis] (0,0,0) -- +(0,2,0) node [right] {$y^{\prime}$};
          \draw[axis] (0,0,0) -- +(0,0,2) node [above] {$z^{\prime}$};
        \end{scope}
      \end{scope}
    \end{scope}

    \begin{scope}[shift={(5,-5)}]
      \node[draw=none] at (-1.25,1.5) {$V_x^{0g}, V_y^{+1g}, V_z^{0g}$};
      \begin{scope}[x={(-135:1cm)}, y={( 0:1cm)}, z={(90:1cm)}]
        \begin{scope}[rotate around x=90]
          \draw[axis] (0,0,0) -- +(0,0,2) node [above] {$z^{\prime}$};
          \draw[very thick,fill=gray,yxplane=0] (-1,-1) rectangle (1,1);
          \draw[axis] (0,0,0) -- +(2,0,0) node [left] {$x^{\prime}$};
          \draw[axis] (0,0,0) -- +(0,2,0) node [right] {$y^{\prime}$};
        \end{scope}
      \end{scope}
    \end{scope}

    \begin{scope}[shift={(10,-5)}]
      \node[draw=none] at (-0.5,1.5) {$V_x^{0g}, V_y^{0g}, V_z^{-1g}$};
      \begin{scope}[x={(-135:1cm)}, y={( 0:1cm)}, z={(90:1cm)}]
        \begin{scope}[rotate around y=180]
          \draw[axis] (0,0,0) -- +(0,0,2) node [right] {$z^{\prime}$};
          \draw[very thick, fill=gray, yxplane=0] (-1,-1) rectangle (1,1);
          \draw[axis] (0,0,0) -- +(2,0,0) node [left] {$x^{\prime}$};
          \draw[axis] (0,0,0) -- +(0,2,0) node [right] {$y^{\prime}$};
        \end{scope}
      \end{scope}
    \end{scope}

  \end{tikzpicture}
  \caption{Pozycjonowanie akcelerometru przy skalowaniu.}
  \label{fig:ZDELS}
\end{figure}
%%%%%%%%%%%%%%%%%%%%%%%%%%%%%%%%%%%%%%%%%%%%%%%%%%%%%%%%%%%%%%%%%%%%%%%%%%%%%%
Podczas pozycjonowania należy zebrać wartości napięć $V_x$, $V_y$ i~$V_z$
zmierzone w~przedstawionych na rysunku~\ref{fig:ZDELS} położeniach
ekstremalnych. Zebrane wartości można zestawić w~tabeli, np. takiej jak
tablica~\ref{tab:Q31HW}.
%%%%%%%%%%%%%%%%%%%%%%%%%%%%%%%%%%%%%%%%%%%%%%%%%%%%%%%%%%%%%%%%%%%%%%%%%%%%%%
\begin{table}[htbp]
  \centering
  \caption{Tabela pomiarowa do skalowania akcelerometru}
  \label{tab:Q31HW}
  \begin{tabular}{|l|c|c|c|c|}
    \hline
                        & Kanał $x$ & Kanał $y$ & Kanał $z$ \\ \hline
    $V_{\bullet}^{0g}$  &           &           &           \\ \hline
    $V_{\bullet}^{-1g}$ &           &           &           \\ \hline
    $V_{\bullet}^{+1g}$ &           &           &           \\ \hline
    $D_{\bullet}$       &           &           &           \\ \hline
  \end{tabular}
\end{table}
%%%%%%%%%%%%%%%%%%%%%%%%%%%%%%%%%%%%%%%%%%%%%%%%%%%%%%%%%%%%%%%%%%%%%%%%%%%%%%

Nachylenia charakterystyk akcelerometru w~poszczególnych kanałach (oś $x^{\prime}$,
$y^{\prime}$ i~$z^{\prime}$) wyrazimy poprzez współczynniki
%%%%%%%%%%%%%%%%%%%%%%%%%%%%%%%%%%%%%%%%%%%%%%%%%%%%%%%%%%%%%%%%%%%%%%%%%%%%%%
\begin{align}
  & D_x = \frac{2 g}{V_x^{+1g} - V_x^{-1g}}, &
  & D_y = \frac{2 g}{V_y^{+1g} - V_y^{-1g}}, &
  & D_z = \frac{2 g}{V_y^{+1g} - V_y^{-1g}}. &
  \label{eq:TUSWM}
\end{align}
%%%%%%%%%%%%%%%%%%%%%%%%%%%%%%%%%%%%%%%%%%%%%%%%%%%%%%%%%%%%%%%%%%%%%%%%%%%%%%
Wartości zmierzonych przyspieszeń wyrażone w~$\tfrac{m}{s^2}$ będziemy zatem
liczyli jako
%%%%%%%%%%%%%%%%%%%%%%%%%%%%%%%%%%%%%%%%%%%%%%%%%%%%%%%%%%%%%%%%%%%%%%%%%%%%%%
\begin{align}
  &{\hat{a}_x}^{\prime}= D_x \left(V_x - V_x^{0g}\right),&
  &{\hat{a}_y}^{\prime}= D_y \left(V_y - V_y^{0g}\right),&
  &{\hat{a}_z}^{\prime}= D_z \left(V_z - V_z^{0g}\right),&
  \label{eq:HM7PF}
\end{align}
%%%%%%%%%%%%%%%%%%%%%%%%%%%%%%%%%%%%%%%%%%%%%%%%%%%%%%%%%%%%%%%%%%%%%%%%%%%%%%
gdzie $V_x$, $V_y$ i~$V_z$ są chwilowymi wartościami napięć zmierzonych
w~kanałach $x$,$y$,$z$ akcelerometru.

\subsubsection{Skalowanie wg dwóch różnych położeń wahadła}
\label{sec:BDRBE}

W~metodzie tej akcelerometr jest zamontowany na końcówce wahadła tak jak
podczas właściwego eksperymentu. Rejestrujemy wskazania akcelerometru
oraz wartości wychyleń kątowych $\bar{\varphi}_x$, $\bar{\varphi}_y$ w~dwóch
położeniach wahadła -- $A$ i $B$ (za każdym razem wahadło powinno być
nieruchome w~danym położeniu). Ważne jest, aby $\bar{\varphi}_{\bullet}^A$
różniło się istotnie od $\bar{\varphi}_{\bullet}^B$ zarówno dla kanału $x$ jak
i~$y$.

Dane z~obu pomiarów możemy zanotować w~tabeli, np.~takiej jak
Tablica~\ref{tab:22DGI}. Wartości kątowe $\bar{\varphi}_x$, $\bar{\varphi}_y$
mogą pochodzić z~kątomierza (jeśli kalibrację robi się niezależnie od
eksperymentu właściwego), bądź mogą być wyznaczone z~napięć $U_x$, $U_y$
(zmierzonych specjalnie w~celu kalibracji, bądź pochodzących z~właściwego
eksperymentu). W~przypadku użycia $U_x$, $U_y$ trzeba zadbać przed obliczeniem
$\bar{\varphi}_x$, $\bar{\varphi}_y$, aby potencjometry były prawidłowo
wyskalowane.
%%%%%%%%%%%%%%%%%%%%%%%%%%%%%%%%%%%%%%%%%%%%%%%%%%%%%%%%%%%%%%%%%%%%%%%%%%%%%%
\begin{table}[htbp]
  \caption{Skalowanie akcelerometru wg dwóch różnych położeń wahadła}
  \label{tab:22DGI}
  \begin{tabular}{c|c|c|c|c|c|c|c|}
    \cline{2-8}
    & \multicolumn{3}{|c|}{$V_{\bullet}$}
    & \multicolumn{2}{|c|}{$U_{\bullet}$}
    & \multicolumn{2}{|c|}{$\bar{\varphi}_{\bullet}$}
    \\
    \cline{2-8}
    & Kanał $V_{x}$ & Kanał $V_{y}$ & Kanał $V_{z}$
    & Kanał $U_{x}$ & Kanał $U_{y}$
    & Kanał $\bar{\varphi}_{x}$ & Kanał $\bar{\varphi}_{y}$
    \\
    \hline
    \multicolumn{1}{|c|}{A} & & & & & & & \\ \hline
    \multicolumn{1}{|c|}{B} & & & & & & & \\ \hline
    \multicolumn{1}{|c|}{$V_{\bullet}^{0g}$}  & & & & \multicolumn{4}{|c}{} \\ \cline{1-4}
    \multicolumn{1}{|c|}{$D_{\bullet}$}  & & & & \multicolumn{4}{|c}{} \\ \cline{1-4}
  \end{tabular}
\end{table}
%%%%%%%%%%%%%%%%%%%%%%%%%%%%%%%%%%%%%%%%%%%%%%%%%%%%%%%%%%%%%%%%%%%%%%%%%%%%%%

Nieznane współczynniki skali $D_{\bullet}$, $V_{\bullet}^{0g}$ wyznacza się
poprzez zapisanie równań \eqref{eq:EUIQU} dla~$\brm{a}^{\prime} = \brm{0}$
w~punktach $A$ i~$B$ i~rozwiązanie tak powstałego układu równań. W~efekcie
otrzymujemy
%%%%%%%%%%%%%%%%%%%%%%%%%%%%%%%%%%%%%%%%%%%%%%%%%%%%%%%%%%%%%%%%%%%%%%%%%%%%%%
\begin{subequations}
  \label{eq:GN30K}
  \begin{align}
    & D_x = -g \frac{ \cos{\bar{\varphi}_x^A}\sin{\bar{\varphi}_y^A} -\cos{\bar{\varphi}_x^B}\sin{\bar{\varphi}_y^B} }
                    { V_x^A - V_x^B }, &
    & V_x^{0g} = V_x^A +\frac{g}{D_x} \cos{\bar{\varphi}_x^A}\sin{\bar{\varphi}_y^A}, &
    \label{eq:L8MQV}
    \\
    & D_y = g \frac{ \sin{\bar{\varphi}_x^A} - \sin{\bar{\varphi}_x^B} }
                   { V_y^A - V_y^B }, &
    & V_y^{0g} = V_y^A - \frac{g}{D_y}\sin{\bar{\varphi}_x^A}, &
    \label{eq:OAGGF}
    \\
    & D_z = g \frac{ \cos{\bar{\varphi}_x^A}\cos{\bar{\varphi}_y^A} -\cos{\bar{\varphi}_x^B}\cos{\bar{\varphi}_y^B} }
                    { V_z^A - V_z^B }, &
    & V_z^{0g} = V_z^A - \frac{g}{D_z} \cos{\bar{\varphi}_x^A}\cos{\bar{\varphi}_y^A}, &
    \label{eq:JZ38T}
  \end{align}
\end{subequations}
%%%%%%%%%%%%%%%%%%%%%%%%%%%%%%%%%%%%%%%%%%%%%%%%%%%%%%%%%%%%%%%%%%%%%%%%%%%%%%
Tak wyznaczone współczynniki mogą być użyte do przeliczania wartości
napięciowych z~akcelerometru na~przyspieszenia $\hat{\brm{a}}^{\prime}$ 
wg wzoru \eqref{eq:HM7PF}.

\section{Pomiar}
\label{sec:CKUHT}

Do przeprowadzenia pomiaru potrzebne są dwie osoby -- jedna przy komputerze,
druga przy wahadle.
\begin{itemize}
  \item Wahadło ustawić {\em nieruchomo} w~wybranym położeniu początkowym.
  \item Uruchomić pomiar na komputerze.
  \item Po około sekundzie wprawić wahadło w~ruch.
  \item Po upływie $32\dots33$ sekund rozpocząć wyhamowywanie wahadła (jeśli się porusza).
  \item Przed $34$ sekundą wahadło powinno być już nieruchome.
  \item Po upływie $36$-tej sekundy zakończyć rejestrację pomiarów.
\end{itemize}
Wynik zapisany przez program pomiarowy w~postaci arkusza Excel będzie zawierał
5 kolumn
%%%%%%%%%%%%%%%%%%%%%%%%%%%%%%%%%%%%%%%%%%%%%%%%%%%%%%%%%%%%%%%%%%%%%%%%%%%%%%
\begin{itemize}
  \item \textbf{Untitled}, \textbf{Untitled 1}, \textbf{Untitled 2}: $V_x$, $V_y$, $V_z$,
  \item \textbf{Untitled 3}, \textbf{Untitled 4}: $U_x$, $U_y$.
\end{itemize}
%%%%%%%%%%%%%%%%%%%%%%%%%%%%%%%%%%%%%%%%%%%%%%%%%%%%%%%%%%%%%%%%%%%%%%%%%%%%%%


\section{Obliczenia}
\label{sec:ZCIBX}

Podczas pomiaru z~czujników zbierane są wartości napięć $U_{x,k}$, $U_{y,k}$
(potencjometry), oraz $V_{x,k}$, $V_{y,k}$, $V_{z,k}$ (akcelerometr)
w~dyskretnych chwilach czasowych $t_k$, gdzie $k$ jest numerem próbki: $k = 0,
1, \dots, K$. Zapisane w~pliku wynikowym liczby są wyrażone w~Voltach.

\subsection{Konwersja zmierzonych wartości}
\label{sec:O9JWA}

W~pierwszej kolejności należy dokonać konwersji wartości uzyskanych z~układu
pomiarowego (wyrażonych w Voltach) do postaci liczb wyrażonych w~jednostkach
docelowych (radiany, $\tfrac{m}{s^2}$):
%%%%%%%%%%%%%%%%%%%%%%%%%%%%%%%%%%%%%%%%%%%%%%%%%%%%%%%%%%%%%%%%%%%%%%%%%%%%%%
\begin{itemize}
  \item z~napięć $U_{\bullet,k}$, wyznaczyć
        wychylenia kątowe $\bar{\varphi}_{\bullet,k}$ (w~radianach) stosując
        wzory~\eqref{eq:547XS}, $\bullet \in \left\{x,y\right\}$,
  \item z~napięć $V_{\bullet,k}$ wyznaczyć przyspieszenia
        ,,brutto'' ${\hat{a}_{\bullet,k}}^{\prime}$ (w~$\tfrac{m}{s^2}$)
        stosując wzory \eqref{eq:HM7PF}, $\bullet \in \left\{x,y,z\right\}$,
\end{itemize}
%%%%%%%%%%%%%%%%%%%%%%%%%%%%%%%%%%%%%%%%%%%%%%%%%%%%%%%%%%%%%%%%%%%%%%%%%%%%%%
dla $k=0,1,\dots,K$.

\subsection{Całkowanie numeryczne przyspieszeń}
\label{sec:OKCY5}

Przebiegi ${\hat{a}_{x,k}}^{\prime}$, ${\hat{a}_{y,k}}^{\prime}$ zmierzonych
przyspieszeń mogą być użyte do wyznaczenia przybliżonych przebiegów wychylenia
kątowego $\widetilde{\varphi}_{x,k}$, $\widetilde{\varphi}_{y,k}$ wahadła.
W~tym celu należy zastosować procedurę całkującą do rozwiązania odpowiednio
postawionego zagadnienia początkowego dla równań różniczkowych~\eqref{eq:ROGUO}.

\paragraph{Implementacja procedury całkującej.}
Aby móc stosować procedury numerycznego całkowania, równania \eqref{eq:ROGUO}
należy przekształcić do postaci równoważnego układu równań pierwszego rzędu
%%%%%%%%%%%%%%%%%%%%%%%%%%%%%%%%%%%%%%%%%%%%%%%%%%%%%%%%%%%%%%%%%%%%%%%%%%%%%%
\begin{equation}
  \dot{\brm{y}}\left(t\right) = \brm{f}\left(t,\brm{y}\left(t\right)\right),
  \label{eq:DOHJE}
\end{equation}
%%%%%%%%%%%%%%%%%%%%%%%%%%%%%%%%%%%%%%%%%%%%%%%%%%%%%%%%%%%%%%%%%%%%%%%%%%%%%%
jak to zaproponowano w~dodatku~\ref{sec:WKKBB}. Prawą stronę $\brm{f}$
takiego równoważnego układu równań (zob.~\eqref{eq:S8LTL}) zapiszemy w~wersji
dyskretnej następująco
%%%%%%%%%%%%%%%%%%%%%%%%%%%%%%%%%%%%%%%%%%%%%%%%%%%%%%%%%%%%%%%%%%%%%%%%%%%%%%
\begin{equation}
  \brm{f}_k \coloneqq \brm{f}\left(t_k, \brm{y}_k\right) = \begin{bmatrix}
    y_{3,k} \\
    y_{4,k} \\
    r^{-1} \left( {\hat{a}_{y,k}}^{\prime} - g \sin{y_{1,k}} \right)  \\
    r^{-1} \left(-{\hat{a}_{x,k}}^{\prime} - g \cos{y_{1,k}}\sin{y_{2,k}} \right)
  \end{bmatrix},
  \label{eq:ETMGP}
\end{equation}
%%%%%%%%%%%%%%%%%%%%%%%%%%%%%%%%%%%%%%%%%%%%%%%%%%%%%%%%%%%%%%%%%%%%%%%%%%%%%%
gdzie
%%%%%%%%%%%%%%%%%%%%%%%%%%%%%%%%%%%%%%%%%%%%%%%%%%%%%%%%%%%%%%%%%%%%%%%%%%%%%%
\begin{equation}
  \brm{y}_k = \begin{bmatrix}
    y_{1,k} & y_{2,k} & y_{3,k} & y_{4,k}
  \end{bmatrix}^T \coloneqq \begin{bmatrix}
    \widetilde{\varphi}_{x,k} &
    \widetilde{\varphi}_{y,k} &
    \dot{\widetilde{\varphi}}_{x,k} &
    \dot{\widetilde{\varphi}}_{y,k}
  \end{bmatrix}^T
  \label{eq:VF0ZM}
\end{equation}
%%%%%%%%%%%%%%%%%%%%%%%%%%%%%%%%%%%%%%%%%%%%%%%%%%%%%%%%%%%%%%%%%%%%%%%%%%%%%%
jest wektorem algebraicznym niewiadomych zapisanych w~dyskretnej chwili
czasowej $t_k$. Symboli zdefiniowanych wzorami~\eqref{eq:ETMGP}
i~\eqref{eq:VF0ZM} możemy użyć wprost do zaimplementowania dowolnego
(stało-krokowego) schematu całkowania, np. schematu jawnego Eulera (zobacz
dodatek~\ref{sec:VNLHT}). W~ten sposób można dość łatwo zaimplementować metodę
całkującą, np. w~arkuszu kalkulacyjnym Excel.

\paragraph{Warunki początkowe.}
Jako warunków początkowych w~procesie całkowania można użyć
%%%%%%%%%%%%%%%%%%%%%%%%%%%%%%%%%%%%%%%%%%%%%%%%%%%%%%%%%%%%%%%%%%%%%%%%%%%%%%
\begin{equation}
  \brm{y}_0 = \begin{bmatrix}
    \bar{\varphi}_{x,0} & \bar{\varphi}_{y,0} & 0 & 0
  \end{bmatrix}^T,
\end{equation}
%%%%%%%%%%%%%%%%%%%%%%%%%%%%%%%%%%%%%%%%%%%%%%%%%%%%%%%%%%%%%%%%%%%%%%%%%%%%%%
gdzie $\bar{\varphi}_{x,0}$, $\bar{\varphi}_{y,0}$ są odchyleniami kątowymi
odczytanymi z~potencjometrów w~chwili czasowej $t_0$, a~zera w~miejscu
$y_{3,0}$ i~$y_{4,0}$ wynikają ze~sposobu przeprowadzenia eksperymentu
(w~chwili początkowej wahadło miało zerową prędkość). Jeśli w~chwili
początkowej wahadło poruszało się z~niezerową prędkością, prędkość początkową
można spróbować aproksymować wzorami różnicowymi na podstawie kilku
początkowych próbek $\bar{\varphi}_{\bullet,k}$.

\subsection{Opis szablonu arkusza kalkulacyjnego do obliczeń}
\label{sec:CS7OJ}

Aby ułatwić przygotowanie sprawozdania, udostępnia się studentom szablon
arkusza kalkulacyjnego do uzupełnienia. Arkusz kalkulacyjny
\texttt{wahadlo-clean.xlsx} zawiera następujące zakładki
%%%%%%%%%%%%%%%%%%%%%%%%%%%%%%%%%%%%%%%%%%%%%%%%%%%%%%%%%%%%%%%%%%%%%%%%%%%%%%
\begin{itemize}
  \item \texttt{PAR} -- parametry eksperymentu,
  \item \texttt{POM} -- miejsce na wstawienie zebranych danych pomiarowych,
  \item \texttt{EULF} -- arkusz do całkowania numerycznego metodą jawną Eulera,
  \item \texttt{EULB} -- arkusz do całkowania numerycznego metodą niejawną Eulera.
  \item \texttt{TRAP} -- arkusz do całkowania numerycznego metodą trapezów.
\end{itemize}
%%%%%%%%%%%%%%%%%%%%%%%%%%%%%%%%%%%%%%%%%%%%%%%%%%%%%%%%%%%%%%%%%%%%%%%%%%%%%%

Arkusz \texttt{POM} jest przygotowany na przyjęcie $3500$ próbek pomiarowych,
co przy okresie próbkowania $10$~ms daje $35$~sekund. Uzupełnianie arkusza,
krok po kroku, wyjaśniono poniżej.

\paragraph{Uwagi ogólne dotyczące uzupełniania arkusza.} Treść należy
,,wklejać specjalnie'' (\texttt{Ctl+Shift+V}), tak aby nie nadpisać
skonfigurowanego odpowiednio formatowania komórek (format liczb, wyświetlanie
jednostek miar itp.). Komórki z~szarym tłem nie wymagają uzupełniania, zwykle
są tam z~góry znane parametry bądź formuły. Komórki bez tła (defaultowo białe)
wymagają wpisania bądź wklejenia wartości (liczbowych). Komórki w~kolorze
żółtym wymagają wpisania odpowiednich formuł.

\paragraph{Zakładka \texttt{PAR}.} Zakładkę z~parametrami eksperymentu
uzupełnia się następująco
%%%%%%%%%%%%%%%%%%%%%%%%%%%%%%%%%%%%%%%%%%%%%%%%%%%%%%%%%%%%%%%%%%%%%%%%%%%%%%
\begin{itemize}
  \item W~tabeli \texttt{PAR-POT} należy wpisać napięcia zebrane podczas
    skalowania potencjometrów. Opcjonalnie, można też poprawić figurujące tam
    zakresy kątów na kątomierzu ($-\#^{\circ}$, $+\#^{\circ}$), jeśli
    skalowania dokonano w~innym zakresie kątów.
  \item Do~tabeli \texttt{PAR-ACC-1G} wpisać napięcia zmierzone przy
    skalowaniu akcelerometru metodą $\mp 1g$.
  \item Do~tabeli \texttt{PAR-ACC-AB} wpisać napięcia zmierzone przy skalowaniu
    akcelerometru wg~dwóch różnych położeń wahadła
    (podrozdział~\ref{sec:BDRBE}). Jeśli wychylenie wahadła przy skalowaniu
    akcelerometru zmierzono kątomierzami, w~szarych polach należy podać
    zmierzone kąty. W~innym przypadku wpisać wartości napięć na potencjometrach
    w~odpowiednie komórki tabeli.
  \item Do~tabeli \texttt{PAR-ACC} wkleić wyliczone współczynniki skali
    akcelerometru bądź to z~\texttt{PAR-ACC-1G}, bądź z~\texttt{PAR-ACC-AB},
    w~zależności od tego, który zestaw współczynników ma być użyty przy
    wyznaczaniu przyspieszeń w~arkuszu \texttt{POM}.
\end{itemize}
%%%%%%%%%%%%%%%%%%%%%%%%%%%%%%%%%%%%%%%%%%%%%%%%%%%%%%%%%%%%%%%%%%%%%%%%%%%%%%

\paragraph{Zakładka \texttt{POM}.} Zakładkę z~pomiarami uzupełnia się
następująco
%%%%%%%%%%%%%%%%%%%%%%%%%%%%%%%%%%%%%%%%%%%%%%%%%%%%%%%%%%%%%%%%%%%%%%%%%%%%%%
\begin{itemize}
  \item z~pliku źródłowego przygotować zakres danych obejmujący $3500$ próbek,
  \item wkleić odpowiednie serie przygotowanych danych w~puste pola kolumn
    \texttt{Vx}, \texttt{Vy}, \texttt{Vz}, \texttt{Ux}, \texttt{Uz}.
\end{itemize}
%%%%%%%%%%%%%%%%%%%%%%%%%%%%%%%%%%%%%%%%%%%%%%%%%%%%%%%%%%%%%%%%%%%%%%%%%%%%%%
W~zakładce \texttt{POM} automatycznie wyliczane są przyspieszenia
$\hat{\brm{a}}^{\prime}$ (przyspieszenia ,,brutto'' mierzone akcelerometrem),
wychylenia kątowe $\bar{\varphi}_x$, $\bar{\varphi}_x$ mierzone potencjometrami
oraz moduł $\left|\hat{\brm{a}}^{\prime}\right|$ mierzonego przyspieszenia
(do celów diagnostycznych).

\paragraph{Zakładki \texttt{EULF}, \texttt{EULB} i~\texttt{TRAP}.} Zakładki
służą do całkowania przyspieszeń. W~żółtych komórkach kolumn \texttt{f1}, ...,
\texttt{f4} oraz \texttt{y1}, ..., \texttt{y4} należy zaimplementować
odpowiednią metodę całkowania. Kolumny \texttt{y1}, ..., \texttt{y4}
przewidziane są dla kolejnych współrzędnych wektora niewiadomych $\brm{y}$, zaś
\texttt{f1}, ..., \texttt{f4} powinny zawierać kolejne składowe funkcji
$\brm{f}$ (patrz rozdział~\ref{sec:OKCY5}).

\section{Przygotowanie sprawozdania}
\label{sec:GO0AP}

Przygotowane sprawozdanie powinno zawierać
%%%%%%%%%%%%%%%%%%%%%%%%%%%%%%%%%%%%%%%%%%%%%%%%%%%%%%%%%%%%%%%%%%%%%%%%%%%%%%
\begin{itemize}
  \item krótkie wprowadzenie do prezentowanego eksperymentu (nie dłuższe niż
    pół strony),
  \item uzupełnioną tabelę uzyskaną przy skalowaniu potencjometrów
    (Tablica~\ref{tab:KM4WU}, podrozdział~\ref{eq:460IS}) oraz tabele
    ze~skalowania akcelerometru metodą $\mp 1g$ (Tablica~\ref{tab:Q31HW},
    podrozdział~\ref{sec:CT4YN}) i~metodą wg~dwóch różnych położeń wahadła
    (Tablica~\ref{tab:22DGI}, podrozdział \ref{sec:BDRBE}),
  \item przebiegi $\tilde{\varphi}_x$, $\tilde{\varphi}_y$ wynikające
    z~całkowania przyspieszeń akcelerometru metodami (cały eksperyment, oraz
    pierwsze $15$ sekund)
    \begin{itemize}
      \item jawna metoda Eulera,
      \item niejawna metoda Eulera,
      \item metoda trapezów,
    \end{itemize}
    w~wariantach
    \begin{itemize}
      \item akcelerometry wyskalowane metodą $\pm 1g$,
      \item akcelerometry wyskalowane metodą wg~dwu położeń wahadła;
    \end{itemize}
    przebiegi mają być przedstawione w~postaci wykresów (na osi poziomej czas
    $t$, na osi pionowej wychylenie kątowe); na każdym wykresie powinien być
    również naniesiony (dla porównania) odpowiedni przebieg
    $\bar{\varphi}_{\bullet}$ uzyskany z~pomiarów czujnikiem położenia
    (potencjometrem),
  \item wnioski; należy zwrócić uwagę jakie czynniki wpływają istotnie na
    uzyskane wyniki całkowania oraz czy ta metoda nadaje się do estymacji
    położeń na długich przedziałach czasowych.
\end{itemize}
%%%%%%%%%%%%%%%%%%%%%%%%%%%%%%%%%%%%%%%%%%%%%%%%%%%%%%%%%%%%%%%%%%%%%%%%%%%%%%


\begin{appendices}
\section{Kinematyka wahadła z~przegubem Cardana}
\label{sec:4XR18}

W~poniższej sekcji wyprowadzono równania opisujące ruch modelu kinematycznego
badanego wahadła. Celem jest uzyskanie formuł, które wiązałyby ze sobą
przyspieszenia $\hat{\brm{a}}^{\prime}$ zmierzone przy pomocy akcelerometru
z~drugimi pochodnymi $\ddot{\varphi}_x$, $\ddot{\varphi}_y$ wychyleń kątowych
w~przegubie Cardana. Zastosowanie metod numerycznego całkowania, czy
różniczkowania, umożliwi nam następnie porównanie trajektorii uzyskanych
różnymi sposobami.

\subsection{Oddziaływanie przyspieszenia grawitacyjnego na akcelerometry}
\label{sec:LIIHL}

W~obecności pola grawitacyjnego, wartości przyspieszeń odczytanych
z~akcelerometru, oprócz obserwowanego przyspieszenia kinematycznego (ruchu),
będą zawierały jeszcze składowe od przyspieszenia grawitacyjnego. Poniższa
dyskusja objaśnia istotę tego zjawiska i~wprowadza technikę eliminacji
składowych grawitacyjnych.

\subsubsection{Przykłady kanoniczne}
\label{sec:YFBVX}

\paragraph{Nieruchomy akcelerometr.}
{\em Nieruchomy} akcelerometr będzie wskazywał wzdłuż wybranej osi  zerowe
przyspieszenie tylko wtedy, gdy ustawimy tę oś idealnie w~płaszczyźnie
poziomej (prostopadłej do wektora grawitacji). W~skrajnie odmiennej sytuacji,
tj. jeśli skierujemy oś nieruchomego akcelerometru idealnie wzdłuż wektora
grawitacji $\brm{g}$, wskazanie na tej osi powinno wynieść $+1g$ bądź $-1g$.
Przyjmijmy, że jeśli akcelerometr porusza się z~dodatnim przyspieszeniem wzdłuż
dowolnej swojej osi, np. $z^{\prime}$, to odczytana z~tego kanału wartość
przyspieszenia powinna być dodatnia. Przy takiej konwencji, oś {\em
nieruchomego} akcelerometru skierowana pionowo w~górę (przeciwnie do wektora
grawitacji), da odczyt $+1g$ (dodatni). Wynika to z~zależności
\eqref{eq:QDX8V} poniżej, gdzie dla nieruchomego akcelerometru należy
przyjąć $\brm{a} = \brm{0}$. Powyższe obserwacje można wykorzystać np.~do
skalowania akcelerometru, tj. do określenia współczynników jego
charakterystyki.

\paragraph{Swobodne spadanie.}
Aby wskazanie wzdłuż pionowo ustawionej osi akcelerometru było zerowe,
należałoby mu pozwolić swobodnie opadać pod wpływem pola grawitacyjnego.
Jeśli oś (np.~$z^{\prime}$)  akcelerometru ustawimy idealnie pionowo w~górę
(zgodnie z~wektorem grawitacji) i~uwolnimy akcelerometr od więzów,
pozwalając mu swobodnie spadać, wskazanie wzdłuż tej osi będzie zerowe, tzn.
$\hat{a}_z =0$. Wiemy jednak, że względem nieruchomego układu współrzędnych
akcelerometr porusza się z~przyspieszeniem $a_z = -g$. A~więc faktyczne
przyspieszenie kinematyczne jest $a_z = \hat{a}_z - g$. W~ogólności
zapisalibyśmy to w~nieruchomym układzie współrzędnych jako
%%%%%%%%%%%%%%%%%%%%%%%%%%%%%%%%%%%%%%%%%%%%%%%%%%%%%%%%%%%%%%%%%%%%%%%%%%%%%%
\begin{equation}
  \brm{a} = \hat{\brm{a}} + \brm{g},
  \label{eq:QDX8V}
\end{equation}
%%%%%%%%%%%%%%%%%%%%%%%%%%%%%%%%%%%%%%%%%%%%%%%%%%%%%%%%%%%%%%%%%%%%%%%%%%%%%%
gdzie
%%%%%%%%%%%%%%%%%%%%%%%%%%%%%%%%%%%%%%%%%%%%%%%%%%%%%%%%%%%%%%%%%%%%%%%%%%%%%%
\begin{equation}
  \brm{g} = \begin{bmatrix} 0 & 0 & -g \end{bmatrix}^T, \;\; g \approx 9.81 \frac{m}{s^2}
  \label{eq:PWKKB}
\end{equation}
%%%%%%%%%%%%%%%%%%%%%%%%%%%%%%%%%%%%%%%%%%%%%%%%%%%%%%%%%%%%%%%%%%%%%%%%%%%%%%
jest wektorem przyspieszenia ziemskiego, oraz
%%%%%%%%%%%%%%%%%%%%%%%%%%%%%%%%%%%%%%%%%%%%%%%%%%%%%%%%%%%%%%%%%%%%%%%%%%%%%%
\begin{align}
  & \brm{a} = \begin{bmatrix}
    a_x & a_y & a_z
  \end{bmatrix}^T, &
  & \hat{\brm{a}} = \begin{bmatrix}
    \hat{a}_x & \hat{a}_y & \hat{a}_z
  \end{bmatrix}^T &
  \label{eq:0JPXH}
\end{align}
%%%%%%%%%%%%%%%%%%%%%%%%%%%%%%%%%%%%%%%%%%%%%%%%%%%%%%%%%%%%%%%%%%%%%%%%%%%%%%
są wektorami algebraicznymi przyspieszeń ,,netto'' i ,,brutto'' akcelerometru
wyrażonymi w~nieruchomym układzie współrzędnych.

\subsubsection{Oddziaływanie na akcelerometr zamocowany na wahadle}
\label{sec:JO17Z}

Oczywiście, fizyka oddziaływania pola grawitacyjnego na akcelerometr nie zmieni
się w~wyniku jego zamontowaniu na końcówce wahadła. Celem niniejszego
podrozdziału jest jedynie określenie składowych wektora grawitacji
pojawiających się na osiach akcelerometru w~zależności od~wychyleń kątowych
wahadła (a~więc wypracowanie odpowiednich transformacji geometrycznych).
Dotychczasowe nasze rozważania przeprowadzono w~nieruchomym układzie
współrzędnych $x,y,z$ jednak w~eksperymencie mamy do czynienia
z~przyspieszeniami wyrażonymi w~układzie
ruchomym~$x^{\prime},y^{\prime},z^{\prime}$ (osie akcelerometru są w~każdej
chwili równoległe do osi układu ruchomego $x^{\prime},y^{\prime},z^{\prime}$).

Rysunek~\ref{fig:AFKVE} pokazuje w~dwóch rzutach wahadło w~położeniu określonym
przez kąty~$\varphi_x$, $\varphi_y$. W~prawej części rysunku pokazano rzut
na płaszczyznę $y-z$, w~której odbywa się obrót $\varphi_x$. Grot osi $x$
skierowany jest w~stronę obserwatora. W~lewej części rysunku pokazano rzut na
płaszczyznę~$z^{\prime\prime}, x^{\prime\prime}$, w~której odbywa się
obrót~$\varphi_y$ (tak jak to widzi oko obserwatora z~prawego rysunku).
%%%%%%%%%%%%%%%%%%%%%%%%%%%%%%%%%%%%%%%%%%%%%%%%%%%%%%%%%%%%%%%%%%%%%%%%%%%%%%
\begin{figure}[htbp]
  \centering
  \begin{tikzpicture}[%
      scale=0.85,%
      axis/.style={very thin,->},%
      rodr/.style={thick,double distance=0.25cm,cap=round},%
      rodr2/.style={thick,double distance=0.35cm,cap=round},%
      rodh/.style={thick,double distance=0.25cm,cap=rect},%
      rodh2/.style={thick,double distance=0.35cm,cap=rect},%
      zxplane/.style={canvas is zx plane at y=#1},%
      yxplane/.style={canvas is yx plane at z=#1},%
      yzplane/.style={canvas is yz plane at x=#1},%
  ]

    \begin{scope}[x={(0,0)},y={(1cm,0)},z={(0,1cm)}]

      \coordinate (r) at (0,0,-4);

      \draw[dashdotted] (0,0,0) -- ++(0,0,-3.5);
      \begin{scope}[rotate around x=30]
        %eye
        \pgfmathsetmacro{\eyeSize}{1}
        \pgfmathsetmacro{\ex}{4}
        \pgfmathsetmacro{\ey}{-2.5}
        \pgfmathsetmacro{\eRot}{-180}
        \pgfmathsetmacro{\eAp}{-55}
        \draw[yzplane=0,rotate around={\eRot:(\ex,\ey)}] (\ex,\ey) -- ++(-.5*\eAp:\eyeSize)
             (\ex,\ey) -- ++(.5*\eAp:\eyeSize);
        \draw[yzplane=0] (\ex,\ey) ++(\eRot+\eAp:.75*\eyeSize) arc (\eRot+\eAp:\eRot-\eAp:.75*\eyeSize);

        % IRIS
        \draw[yzplane=0,fill=gray] (\ex,\ey) ++(\eRot+\eAp/3:.75*\eyeSize) % start point
          arc (\eRot+180-\eAp:\eRot+180+\eAp:.28*\eyeSize);

        %PUPIL, a filled arc
        \draw[yzplane=0,fill=black] (\ex,\ey) ++(\eRot+\eAp/3:.75*\eyeSize) % start point
          arc (\eRot+\eAp/3:\eRot-\eAp/3:.75*\eyeSize);
        %
        \draw[draw=lightgray,thin,dashed,->] (0,3,-2.5) -- ++ (0,-2.5,0);
        %
        \draw[rodh](0,0,-0.50)-- (0,0,-4) node[above=0.20cm,right=0.20cm] {$r$};
        \draw[rodh](0,-0.85,0) -- (0,0.85,0);
        \draw[rodh2](0,-0.75,0.1) -- (0,-0.75,-0.50) -- (0,0.75,-0.50) -- (0,0.75,0.1);
        \draw[thin,dashdotted] (0,0,0) -- ++ (0,0,-4);
        % x''-y''-z'' coordinate system
        \draw[dashed,axis] (0,0,0) -- ++(0,2,0) node[right] {$y^{\prime\prime},y^{\prime}$};
        \draw[dashed,axis] (0,0,0) -- ++(0,0,2) node[above] {$z^{\prime\prime},z^{\prime}$};

        \begin{scope}[shift={(30:-4cm)}]
          \draw[thin,dashed] (0,0,-5.5) -- (0,0,3);
        \end{scope}

        \begin{scope}[shift={(30:-8cm)}]
          \begin{scope}[rotate around z=-90]
            \draw[rodh] (-0.85,0,0) -- (0.85,0,0);
            % base
            \draw[rodh2](-0.75,0,0.75) -- (-0.75,0,-0.1);
            \draw[rodh2](0.75,0,-0.1) -- (0.75,0,0.75);
            \draw[zxplane=0,fill=white,postaction={pattern=north east lines}] (0.75,-1) rectangle (1.25,1);
            \draw[dashdotted] (0,0,0) -- ++(0,0,-3.5);
            % pendulum
            \begin{scope}[rotate around y=30]
              \draw[rodh](0,0,0) -- ++(0,0,{-4/cos(30)}) node[above=0.30cm,right=0.05cm] {$r$};
              \draw[rodr2](0,0,{-0.35/cos(30)}) -- (0,0,0);
              \draw[zxplane=0] (0,0,0) circle (0.125);
              \draw[thin,dashdotted,cap=round] (0,0,0) -- ++ (0,0,{-4/cos(30)});
              % x'-y'-z' coordinate system
              \draw[axis] (0,0,0) -- ++(2,0,0) node[right] {$x^{\prime}$};
              \draw[axis] (0,0,0) -- ++(0,0,2) node[above] {$z^{\prime}$};
            \end{scope}
            % angle \varphi_y
            \draw[axis,zxplane=0] (0,0) ++ (180:2) arc(180:210:2) node [below=0.25cm] {$\varphi_y$};
            % gravity
            \draw[thick,->,zxplane=0] (-150:{4/cos(30)}) -- ++ (-{2*cos(30)},0) node[below] {$g\cos{\varphi_x}$};
            \draw[thick,->,zxplane=0] (-150:{4/cos(30)}) -- ++ (-150:{2*cos(30)*cos(30)}) node[below=0.25cm,right] {$g\cos{\varphi_x}\cos{\varphi_y}$};
            \draw[thick,->,zxplane=0] (-150:{4/cos(30)}) -- ++ (-240:{2*cos(30)*sin(30)}) node[below=0.25cm,left] {$g\cos{\varphi_x}\sin{\varphi_y}$};
            % x''-y''-z'' coordinate system
            \draw[dashed,axis] (0,0,0) -- ++(2,0,0) node[left] {$x^{\prime\prime}$};
            \draw[dashed,axis] (0,0,0) -- ++(0,0,2) node[above] {$z^{\prime\prime},z$};
          \end{scope}
        \end{scope}

      \end{scope}
      \draw[axis,yzplane=0] (0,0) ++ (-90:2) arc(-90:-60:2) node [midway, below] {$\varphi_x$};

      \draw[rodr2] (0,0,0) -- (0,0,0.75);
      \draw[yzplane=0,fill=white,postaction={pattern=north east lines}] (-1,0.75) rectangle (1,1.25);
      \draw[yzplane=0] (0,0,0) circle (0.125);

      % gravity
      \draw[thick,->,yzplane=0] (-60:4) -- ++ (0,-2) node[below] {$g$};
      \draw[thick,->,yzplane=0] (-60:4) -- ++ ( -60:{2*cos(30)}) node[below right] {$g\cos{\varphi_x}$};
      \draw[thick,->,yzplane=0] (-60:4) -- ++ (-150:{2*sin(30)}) node[left] {$g\sin{\varphi_x}$};
      % gravity helper lines
      \draw[thin,draw=lightgray,dashed,yzplane=0] {(-60:4)+(-150:{2*sin(30)})} -- ++ (-150:6);
      \draw[thin,draw=lightgray,dashed,yzplane=0] {(-60:{4+2*cos(30)})} -- ++ (-150:6);

      % x-y-z coordinate system
      \draw[axis] (0,0,0) -- ++(0,2,0) node[right] {$y$};
      \draw[axis] (0,0,0) -- ++(0,0,2) node[above] {$z$};
    \end{scope}
  \end{tikzpicture}
  \caption{Oddziaływanie przyspieszenia ziemskiego na końcówkę wahadła}
  \label{fig:AFKVE}
\end{figure}
%%%%%%%%%%%%%%%%%%%%%%%%%%%%%%%%%%%%%%%%%%%%%%%%%%%%%%%%%%%%%%%%%%%%%%%%%%%%%%

Zakładając, że osie akcelerometru są zgodne z~osiami układu ruchomego
$x^{\prime},y^{\prime},z^{\prime}$ wnioskujemy na podstawie rysunku oraz
wcześniejszych rozważań, że przyspieszenie kinematyczne $\brm{a}^{\prime}$
(,,netto'') będzie związane ze~wskazywanym przez akcelerometr przyspieszeniem
(,,brutto'')~$\hat{\brm{a}}^{\prime}$ w~następujący sposób
%%%%%%%%%%%%%%%%%%%%%%%%%%%%%%%%%%%%%%%%%%%%%%%%%%%%%%%%%%%%%%%%%%%%%%%%%%%%%%
\begin{equation}
  {\brm{a}^{\prime}} = \hat{\brm{a}}^{\prime}
  -
  g \cdot
  \begin{bmatrix}
              -\cos{\varphi_x} \sin{\varphi_y}  \\
                      \sin{\varphi_x}          \\
    \phantom{-}\cos{\varphi_x} \cos{\varphi_y}
  \end{bmatrix},
  \label{eq:EUIQU}
\end{equation}
%%%%%%%%%%%%%%%%%%%%%%%%%%%%%%%%%%%%%%%%%%%%%%%%%%%%%%%%%%%%%%%%%%%%%%%%%%%%%%
przy czym
%%%%%%%%%%%%%%%%%%%%%%%%%%%%%%%%%%%%%%%%%%%%%%%%%%%%%%%%%%%%%%%%%%%%%%%%%%%%%%
\begin{align}
  & \brm{a}^{\prime} = \begin{bmatrix}
    a_x^{\prime}  & a_y^{\prime}  & a_z^{\prime}
  \end{bmatrix}^T, &
  & \hat{\brm{a}}^{\prime}  = \begin{bmatrix}
    \hat{a}_x^{\prime}  & \hat{a}_y^{\prime}  & \hat{a}_z^{\prime}
  \end{bmatrix}^T &
  \label{eq:NL9I0}
\end{align}
%%%%%%%%%%%%%%%%%%%%%%%%%%%%%%%%%%%%%%%%%%%%%%%%%%%%%%%%%%%%%%%%%%%%%%%%%%%%%%
są przyspieszeniami ,,netto'' i~,,brutto'' akcelerometru wyrażonymi w~układzie
ruchomym $x^{\prime},y^{\prime},z^{\prime}$.

\subsection{Związek między przyspieszeniami liniowymi i~kątowymi}
\label{sec:QTVP6}

Przyspieszenie ,,netto'' $\brm{a}^{\prime}$ odzwierciedla liniowe
przyspieszenie kinematyczne końcówki wahadła wyrażone w~układzie ruchomym.
Jeśliby przyjąć, że wynikające z~pomiarów przyspieszenia $\brm{a}^{\prime}$
odpowiadają dokładnie faktycznym przyspieszeniom kinematycznym końcówki
(perfekcyjny eksperyment) to przyspieszenia kątowe $\ddot{\varphi_x}$
i~$\ddot{\varphi_y}$ będą wynikały wprost z~przyspieszeń $\brm{a}^{\prime}$
%%%%%%%%%%%%%%%%%%%%%%%%%%%%%%%%%%%%%%%%%%%%%%%%%%%%%%%%%%%%%%%%%%%%%%%%%%%%%%
\begin{subequations}
  \label{eq:9HDQE}
  \begin{align}
    \ddot{\varphi}_x &=\phantom{-}r^{-1} a_y^{\prime},
    \label{eq:UIQUR}
    \\
    \ddot{\varphi}_y &=         - r^{-1} a_x^{\prime},
    \label{eq:ZBLRS}
  \end{align}
\end{subequations}
%%%%%%%%%%%%%%%%%%%%%%%%%%%%%%%%%%%%%%%%%%%%%%%%%%%%%%%%%%%%%%%%%%%%%%%%%%%%%%
($r$ jest długością wahadła). Równania \eqref{eq:9HDQE} stanowią
układ równań różniczkowych zwyczajnych drugiego rzędu z~niewiadomymi funkcjami
$\varphi_x(t)$, $\varphi_y(t)$ oraz dobrze zdefiniowanymi prawymi stronami
$r^{-1}a_x^{\prime}(t,\varphi_x,\varphi_y)$, $r^{-1}a_y^{\prime}(t,\varphi_x)$
%%%%%%%%%%%%%%%%%%%%%%%%%%%%%%%%%%%%%%%%%%%%%%%%%%%%%%%%%%%%%%%%%%%%%%%%%%%%%%
\begin{subequations}
  \label{eq:ROGUO}
  \begin{align}
    \ddot{\varphi}_x & = \phantom{-} r^{-1} \left(
      \hat{a}_y^{\prime}\left(t\right) - g \sin{\varphi_x}
    \right),
    \label{eq:HRYMM}
    \\
    \ddot{\varphi}_y & = - r^{-1} \left(
        \hat{a}_x^{\prime}\left(t\right) + g \cos{\varphi_x} \sin{\varphi_y}
    \right).
    \label{eq:D4N2B}
  \end{align}
\end{subequations}
%%%%%%%%%%%%%%%%%%%%%%%%%%%%%%%%%%%%%%%%%%%%%%%%%%%%%%%%%%%%%%%%%%%%%%%%%%%%%%
Równania takie, po opatrzeniu odpowiednimi warunkami początkowymi, można
rozwiązywać np.~przy użyciu procedur całkowania numerycznego. W~naszym
przypadku, użycie tego typu metody do danych pomiarowych umożliwiłoby
wyznaczenie przybliżonych przebiegów wychyleń wahadła $\varphi_x$, $\varphi_y$
na podstawie przyspieszeń zebranych z~akcelerometrów.

\subsection{Redukcja równań różniczkowych II rzędu do postaci równań I rzędu}
\label{sec:WKKBB}

Aby zastosować procedurę całkowania numerycznego, problem powinien być
sformułowany jako zagadnienie początkowe z~układem równań różniczkowych
pierwszego rzędu, czyli w~postaci~\eqref{eq:L8KAD}. W~przypadku
równań~\eqref{eq:ROGUO} można to osiągnąć poprzez zastosowanie następującego
zestawu zmiennych
%%%%%%%%%%%%%%%%%%%%%%%%%%%%%%%%%%%%%%%%%%%%%%%%%%%%%%%%%%%%%%%%%%%%%%%%%%%%%%
\begin{equation}
  \brm{y} = \begin{bmatrix}
    y_1 & y_2 & y_3 & y_4
  \end{bmatrix}^T \coloneqq \begin{bmatrix}
    \varphi_x & \varphi_y & \dot{\varphi}_x & \dot{\varphi}_y
  \end{bmatrix}^T.
  \label{eq:U09S6}
\end{equation}
%%%%%%%%%%%%%%%%%%%%%%%%%%%%%%%%%%%%%%%%%%%%%%%%%%%%%%%%%%%%%%%%%%%%%%%%%%%%%%
Teraz, w~celu wyrażenia równań~\eqref{eq:ROGUO} w~postaci~\eqref{eq:R7M37}
wystarczy zdefiniować funkcję $\brm{f}$ jako
%%%%%%%%%%%%%%%%%%%%%%%%%%%%%%%%%%%%%%%%%%%%%%%%%%%%%%%%%%%%%%%%%%%%%%%%%%%%%%
\begin{equation}
  \brm{f}\left(t,\brm{y}\right) \coloneqq \begin{bmatrix}
    y_3 \\
    y_4 \\
    r^{-1} \left( {\hat{a}_y}^{\prime}\left(t\right) - g \sin{y_1} \right)  \\
    r^{-1} \left(-{\hat{a}_x}^{\prime}\left(t\right) - g \cos{y_1}\sin{y_2} \right)
  \end{bmatrix}.
  \label{eq:S8LTL}
\end{equation}
%%%%%%%%%%%%%%%%%%%%%%%%%%%%%%%%%%%%%%%%%%%%%%%%%%%%%%%%%%%%%%%%%%%%%%%%%%%%%%
Jako $\brm{y}_0$ w~warunkach początkowych \eqref{eq:5DFP4} należy oczywiście
użyć
%%%%%%%%%%%%%%%%%%%%%%%%%%%%%%%%%%%%%%%%%%%%%%%%%%%%%%%%%%%%%%%%%%%%%%%%%%%%%%
\begin{equation}
  \brm{y}_0 = \begin{bmatrix}
    \varphi_{x,0} & \varphi_{y,0} & \dot{\varphi}_{x,0} & \dot{\varphi}_{y,0}
  \end{bmatrix}^T.
\end{equation}
%%%%%%%%%%%%%%%%%%%%%%%%%%%%%%%%%%%%%%%%%%%%%%%%%%%%%%%%%%%%%%%%%%%%%%%%%%%%%%

\section{Metody całkowania numerycznego}
\label{sec:VNLHT}

Współczesna numeryka oferuje cały wachlarz metod obliczeniowych umożliwiających
przybliżone rozwiązywanie zagadnień różniczkowych. Poszczególne metody
numerycznego całkowania równań różniczkowych różnią się między sobą m.in.
takimi właściwościami, jak rząd dokładności, obszar stabilności, tłumienie
numeryczne~\cite{asher&petzold:1998:computer-methods}.

Tradycyjne metody numerycznego całkowania równań różniczkowych zwyczajnych
umożliwiają rozwiązanie zagadnienia początkowego sformułowanego jako
%%%%%%%%%%%%%%%%%%%%%%%%%%%%%%%%%%%%%%%%%%%%%%%%%%%%%%%%%%%%%%%%%%%%%%%%%%%%%%
\begin{subequations}
\label{eq:L8KAD}
\begin{align}
  \dot{\brm{y}}\left(t\right) &= \brm{f}\left(t, \brm{y}(t)\right),
  \label{eq:R7M37}
  \\
  \brm{y}\left(0\right) &= \brm{y}_0 \in \mathbb{R}^n,
  \label{eq:5DFP4}
\end{align}
\end{subequations}
%%%%%%%%%%%%%%%%%%%%%%%%%%%%%%%%%%%%%%%%%%%%%%%%%%%%%%%%%%%%%%%%%%%%%%%%%%%%%%
gdzie $\brm{y}: \mathbb{R} \mapsto \mathbb{R}^n$ jest poszukiwaną funkcją.
Równanie \eqref{eq:R7M37} jest równaniem różniczkowym zwyczajnym (pierwszego
rzędu), zaś \eqref{eq:5DFP4} definiuje warunki początkowe dla zagadnienia.

Klasyfikacji metod numerycznego całkowania dokonuje się wg rozmaitych
kryteriów. Pod względem konstrukcji schematu różnicowego, możemy wyodrębnić
podział na metody jedno i~wielokrokowe, a~w~każdej z~tych kategorii będą
występować schematy jawne i~niejawne. Przy ocenie dokładności rozwiązania
uzyskiwanego daną metodą zwykle mówi się o~rzędzie metody (mówi się, że metoda
jest rzędu $p$ jeśli błąd uzyskiwanego przybliżenia numerycznego jest $d =
O(h^{p+1})$, gdzie $h$ jest krokiem całkowania). Na potrzeby laboratorium
wprowadzimy kilka znanych metod całkowania.

\subsection{Metody jednokrokowe}
\label{sec:5BIFZ}

\subsubsection{Metody Eulera}
\label{sec:AN1EA}

Znane i stosowane są dwa warianty metody Eulera -- jawna i niejawna. Metody
Eulera są metodami pierwszego rzędu. Różnią się między sobą obszarem
stabilności.

\paragraph{Jawna metoda Eulera.}
W~metodzie jawnej Eulera używa się następującego schematu różnicowego do
przybliżenia pochodnej $\dot{\brm{y}}$ (schemat różnicowy ,,w~przód'')
%%%%%%%%%%%%%%%%%%%%%%%%%%%%%%%%%%%%%%%%%%%%%%%%%%%%%%%%%%%%%%%%%%%%%%%%%%%%%%
\begin{equation}
  \dot{\brm{y}}_{k-1} = \frac{\brm{y}_{k} - \brm{y}_{k-1}}{h},
  \label{eq:IHSVX}
\end{equation}
%%%%%%%%%%%%%%%%%%%%%%%%%%%%%%%%%%%%%%%%%%%%%%%%%%%%%%%%%%%%%%%%%%%%%%%%%%%%%%
gdzie $\brm{y}_{l} \approx \brm{y}\left(t_l\right)$ jest przybliżeniem wartości
poszukiwanej funkcji w~punkcie $t_l$ a~$\dot{\brm{y}}_{l} \approx
\dot{\brm{y}}(t_l)$ przybliżeniem pochodnej. W~wyniku prostego przekształcenia,
otrzymuje się jawną metodę Eulera
%%%%%%%%%%%%%%%%%%%%%%%%%%%%%%%%%%%%%%%%%%%%%%%%%%%%%%%%%%%%%%%%%%%%%%%%%%%%%%
\begin{equation}
  \brm{y}_k = \brm{y}_{k-1} + h \cdot \brm{f}_{k-1},\;\;k = 1,\dots,K,
 \label{eq:3PVVU}
\end{equation}
%%%%%%%%%%%%%%%%%%%%%%%%%%%%%%%%%%%%%%%%%%%%%%%%%%%%%%%%%%%%%%%%%%%%%%%%%%%%%%
w~której $\brm{f}_l = \brm{f}\left(t_l, \brm{y}_l\right)$. Zauważmy, że dla
$k=1$ mamy po prawej stronie $\brm{y}_0$ znane z~warunków początkowych, więc
możemy od razu wyliczyć $\brm{y}_1$ i~dalej, krok po kroku, $\brm{y}_2$,
$\brm{y}_3$, itd..

\paragraph{Niejawna metoda Eulera.}
W~niejawnej metodzie Eulera, pochodną aproksymuje się następującym schematem
różnicowym (wsteczna różnica skończona)
%%%%%%%%%%%%%%%%%%%%%%%%%%%%%%%%%%%%%%%%%%%%%%%%%%%%%%%%%%%%%%%%%%%%%%%%%%%%%%
\begin{equation}
  \dot{\brm{y}}_k = \frac{\brm{y}_{k} - \brm{y}_{k-1}}{h}.
  \label{eq:YNJO9}
\end{equation}
%%%%%%%%%%%%%%%%%%%%%%%%%%%%%%%%%%%%%%%%%%%%%%%%%%%%%%%%%%%%%%%%%%%%%%%%%%%%%%
Uzyskana metoda całkowania ma więc postać
%%%%%%%%%%%%%%%%%%%%%%%%%%%%%%%%%%%%%%%%%%%%%%%%%%%%%%%%%%%%%%%%%%%%%%%%%%%%%%
\begin{equation}
  \brm{y}_k = \brm{y}_{k-1} + h \cdot \brm{f}_{k},\;\;k = 1,\dots,K.
  \label{eq:R8JE3}
\end{equation}
%%%%%%%%%%%%%%%%%%%%%%%%%%%%%%%%%%%%%%%%%%%%%%%%%%%%%%%%%%%%%%%%%%%%%%%%%%%%%%
Należy zaznaczyć, że równanie \eqref{eq:R8JE3} jest nieliniowym równaniem
uwikłanym, ponieważ niewiadome~$\brm{y}_k$ są uwikłane w~definicji nieliniowej
funkcji $\brm{f}_k = \brm{f}(t_k,\brm{y}_k)$. W~każdym kroku całkowania należy
więc zastosować odpowiednią (iteracyjną) metodę rozwiązywania równań
nieliniowych (np.~metodę punktu stałego, bądź metodę Newtona).

\subsection{Metody wielokrokowe}
\label{sec:QS6QT}

Ogólny przepis na (liniową) metodę wielokrokową jest następujący
%%%%%%%%%%%%%%%%%%%%%%%%%%%%%%%%%%%%%%%%%%%%%%%%%%%%%%%%%%%%%%%%%%%%%%%%%%%%%%
\begin{equation}
  \sum_{j=0}^J \alpha_j \brm{y}_{k-j} = h \sum_{j=0}^J \beta_j \brm{f}_{k-j}.
  \label{eq:2AOXP}
\end{equation}
%%%%%%%%%%%%%%%%%%%%%%%%%%%%%%%%%%%%%%%%%%%%%%%%%%%%%%%%%%%%%%%%%%%%%%%%%%%%%%
Konkretne metody całkowania konstruuje się poprzez zastosowanie pewnych
(znanych) zestawów współczynników $\alpha_j$ i $\beta_j$. Zakłada się przy tym,
że $\alpha_0 \neq 0$ (w~wyśmienitej większości metod $\alpha_0 = 1$).
Liczba~$J$ definiuje liczbę kroków metody. Mówi ona ile kroków wstecz dana
metoda spogląda aby wyznaczyć bieżącą wartość $\brm{y}_k$. W~grupie metod
wielo-krokowych będą występować {\em metody jawne} ($\beta_0 = 0$) i~{\em
metody niejawne} ($\beta_0 \neq 0$).

Przyjmując, zgodnie z~wcześniejszą uwagą, $\alpha_0 = 1$, wzór \eqref{eq:2AOXP}
można przekształcić do postaci
%%%%%%%%%%%%%%%%%%%%%%%%%%%%%%%%%%%%%%%%%%%%%%%%%%%%%%%%%%%%%%%%%%%%%%%%%%%%%%
\begin{equation}
  \brm{y}_k = \sum_{j=1}^J \alpha_j \brm{y}_{k-j} + h \sum_{j=0}^J \beta_j \brm{f}_{k-j}
  \label{eq:NN23A}
\end{equation}
%%%%%%%%%%%%%%%%%%%%%%%%%%%%%%%%%%%%%%%%%%%%%%%%%%%%%%%%%%%%%%%%%%%%%%%%%%%%%%
Dla $k>=J$ wszystkie wielkości po prawej stronie (oprócz nieznanego
$\brm{f}_k$) są określone, toteż można próbować rozwiązać
równanie~\eqref{eq:NN23A} względem $\brm{y}_k$. W~przypadku $k < J$ (np. $k=1$
dla metody dwu-krokowej) należy zastosować w~danym punkcie $k$ metodę niższego
rzędu (co najwyżej $k$-krokową). Mówi się wtedy potocznie o~użyciu metody
startującej naszą metodę. Zadaniem takiego ,,startera'' jest wyznaczenie $J-1$
wartości początkowych $\brm{y}_1, \dots \brm{y}_{J-1}$ na~początku procesu
całkowania.

Dla metod jawnych ($\beta_0 = 0$) wzór \eqref{eq:NN23A} przybiera szczególną
postać
%%%%%%%%%%%%%%%%%%%%%%%%%%%%%%%%%%%%%%%%%%%%%%%%%%%%%%%%%%%%%%%%%%%%%%%%%%%%%%
\begin{equation}
  \brm{y}_k = \sum_{j=1}^J \alpha_j \brm{y}_{k-j} + h \sum_{j=1}^J \beta_j \brm{f}_{k-j}
  \label{eq:EQ0F7}
\end{equation}
%%%%%%%%%%%%%%%%%%%%%%%%%%%%%%%%%%%%%%%%%%%%%%%%%%%%%%%%%%%%%%%%%%%%%%%%%%%%%%
tzn. znika z~niego wyrażenie $\beta_0 \brm{f}_k$ i~po prawej stronie zostają
tylko wielkości z~poprzednich kroków.

Do popularnych metod wielokrokowych należą metody Adamsa (jawne i~niejawne).
W~metodach tych $\alpha_0=1$, $\alpha_1=-1$ i~$\alpha_j = 0$ dla $j > 1$.

\subsubsection{Jawne metody Adamsa (metody Adamsa-Bashfortha)}
\label{sec:JVLO6}

Wszystkie metody Adamsa-Bashfortha działają wg poniższej formuły liniowej
%%%%%%%%%%%%%%%%%%%%%%%%%%%%%%%%%%%%%%%%%%%%%%%%%%%%%%%%%%%%%%%%%%%%%%%%%%%%%%
\begin{equation}
  \brm{y}_k = \brm{y}_{k-1} + h \sum_{j=1}^J \beta_j \brm{f}_{k-j}.
  \label{eq:DUQAC}
\end{equation}
%%%%%%%%%%%%%%%%%%%%%%%%%%%%%%%%%%%%%%%%%%%%%%%%%%%%%%%%%%%%%%%%%%%%%%%%%%%%%%
Istnieje cała rodzina metod Adamsa-Bashforta różniących się rzędem metody.
Najczęściej stosowane są metody rzędu $p$ od $1$ do $6$, dla których
współczynniki $\beta_j$ przedstawiono w~tablicy~\ref{tab:OWS53}.
%%%%%%%%%%%%%%%%%%%%%%%%%%%%%%%%%%%%%%%%%%%%%%%%%%%%%%%%%%%%%%%%%%%%%%%%%%%%%%
\begin{table}[htbp]
  \caption{Metody Adamsa-Bashfortha dla $p=1,\dots6$}
  \label{tab:OWS53}
  \centering
  \begin{tabular}{|c|c|r|r|r|r|r|r|r|}
    \hline
    $p$ & $J$ & $j \rightarrow$ &    1 &    2 &    3 &    4 & 5    & 6    \\ \hline
     1  &  1  &   $\beta_j$     &    1 &      &      &      &      &      \\
     2  &  2  &   $2\beta_j$    &    3 &   -1 &      &      &      &      \\
     3  &  3  &  $12\beta_j$    &   23 &  -16 &    5 &      &      &      \\
     4  &  4  &  $24\beta_j$    &   55 &  -59 &   37 &   -9 &      &      \\
     5  &  5  & $720\beta_j$    & 1901 &-2774 & 2616 & 1274 &  251 &      \\
     6  &  6  &$1440\beta_j$    & 4277 &-7923 & 9982 &-7298 & 2877 & -475 \\ \hline
  \end{tabular}
\end{table}
%%%%%%%%%%%%%%%%%%%%%%%%%%%%%%%%%%%%%%%%%%%%%%%%%%%%%%%%%%%%%%%%%%%%%%%%%%%%%%
Przykładowo, zastosowanie współczynników dla $p=3$ generuje metodę 3-krokową
($J=3$) jak poniżej
%%%%%%%%%%%%%%%%%%%%%%%%%%%%%%%%%%%%%%%%%%%%%%%%%%%%%%%%%%%%%%%%%%%%%%%%%%%%%%
\begin{equation}
  \brm{y}_{k} =  \brm{y}_{k-1} + \frac{h}{12} \left(23\,\brm{f}_{k-1} - 16\,\brm{f}_{k-2} + 5\,\brm{f}_{k-3} \right).
  \label{eq:IYF96}
\end{equation}
%%%%%%%%%%%%%%%%%%%%%%%%%%%%%%%%%%%%%%%%%%%%%%%%%%%%%%%%%%%%%%%%%%%%%%%%%%%%%%
Należy zauważyć, że metoda Adamsa-Bashfortha pierwszego rzędu jest w~zasadzie
metodą jednokrokową i~jest tożsama z~jawną metodą Eulera.


\subsubsection{Niejawne metody Adamsa (metody Adamsa-Moultona)}
\label{sec:FYDDW}

Niejawne metody Adamsa-Moultona mają postać ogólną
%%%%%%%%%%%%%%%%%%%%%%%%%%%%%%%%%%%%%%%%%%%%%%%%%%%%%%%%%%%%%%%%%%%%%%%%%%%%%%
\begin{equation}
  \brm{y}_k = \brm{y}_{k-1} + h \sum_{j=0}^J \beta_j \brm{f}_{k-j}.
  \label{eq:YCGME}
\end{equation}
%%%%%%%%%%%%%%%%%%%%%%%%%%%%%%%%%%%%%%%%%%%%%%%%%%%%%%%%%%%%%%%%%%%%%%%%%%%%%%
Podobnie jak w~przypadku jawnych metod Adamsa, dla każdego $p=1,2,\dots,6$
opracowano unikalny zestaw współczynników $\beta_j$ definiujący metodę
Adamsa-Moultona rzędu $p$. Współczynniki te zebrano w~tablicy~\ref{tab:06769}.
%%%%%%%%%%%%%%%%%%%%%%%%%%%%%%%%%%%%%%%%%%%%%%%%%%%%%%%%%%%%%%%%%%%%%%%%%%%%%%
\begin{table}[htbp]
  \caption{Metody Adamsa-Moultona dla $p=1,\dots6$}
  \label{tab:06769}
  \centering
  \begin{tabular}{|c|c|r|r|r|r|r|r|r|}
    \hline
    $p$ & $J$ & $j \rightarrow$ &    0 &    1 &    2 &    3 &    4 &    5 \\ \hline
     1  &  1  &   $\beta_j$     &    1 &      &      &      &      &      \\
     2  &  1  &   $2\beta_j$    &    1 &    1 &      &      &      &      \\
     3  &  2  &  $12\beta_j$    &    5 &    8 &   -1 &      &      &      \\
     4  &  3  &  $24\beta_j$    &    9 &   19 &   -5 &    1 &      &      \\
     5  &  4  & $720\beta_j$    &  251 &  646 & -264 &  106 &  -19 &      \\
     6  &  5  &$1440\beta_j$    &  475 & 1427 & -798 &  482 & -173 &   27 \\ \hline
  \end{tabular}
\end{table}
%%%%%%%%%%%%%%%%%%%%%%%%%%%%%%%%%%%%%%%%%%%%%%%%%%%%%%%%%%%%%%%%%%%%%%%%%%%%%%
Przykładowo, metodę Adamsa-Moultona rzędu $p=3$ zapiszemy, zgodnie z~tablicą,
jako
%%%%%%%%%%%%%%%%%%%%%%%%%%%%%%%%%%%%%%%%%%%%%%%%%%%%%%%%%%%%%%%%%%%%%%%%%%%%%%
\begin{equation}
  \brm{y}_k = \brm{y}_{k-1} + \frac{h}{12}\left(
    5 \brm{f}_k + 8 \brm{f}_{k-1} - 1 \brm{f}_{k-2}
  \right)
\end{equation}
%%%%%%%%%%%%%%%%%%%%%%%%%%%%%%%%%%%%%%%%%%%%%%%%%%%%%%%%%%%%%%%%%%%%%%%%%%%%%%

Należy zauważyć, że istnieją dwie różne jedno-krokowe metody Adamsa-Moultona,
jedna rzędu $1$, druga rzędu $2$. Ponadto dostrzeżemy, że
%%%%%%%%%%%%%%%%%%%%%%%%%%%%%%%%%%%%%%%%%%%%%%%%%%%%%%%%%%%%%%%%%%%%%%%%%%%%%%
\begin{itemize}
  \item wariant $J=1$ z~$\beta_1 = 0$ (metoda $p=1$) jest w~istocie niejawną
        metodą Eulera,
  \item wariant $J=1$ z~$\beta_1 \neq 0$ (metoda $p=2$) jest znana skądinąd jako
        niejawna metoda trapezów
        %%%%%%%%%%%%%%%%%%%%%%%%%%%%%%%%%%%%%%%%%%%%%%%%%%%%%%%%%%%%%%%%%%%%%%
        \begin{equation}
          \brm{y}_k = \brm{y}_{k-1} + \frac{h}{2} \left(\brm{f}_{k} + \brm{f}_{k-1}\right).
          \label{eq:GY62E}
        \end{equation}
        %%%%%%%%%%%%%%%%%%%%%%%%%%%%%%%%%%%%%%%%%%%%%%%%%%%%%%%%%%%%%%%%%%%%%%
\end{itemize}
%%%%%%%%%%%%%%%%%%%%%%%%%%%%%%%%%%%%%%%%%%%%%%%%%%%%%%%%%%%%%%%%%%%%%%%%%%%%%%


\end{appendices}

\bibliographystyle{unsrtnat}
\bibliography{lpas}

\end{document}
%% \label{??:PUZKT}
%% \label{??:S4682}
%% \label{??:KUPE1}
%% \label{??:3H14E}
%% \label{??:BZ9QQ}
%% \label{??:9FO7L}
%% \label{??:7N0I0}
%% \label{??:83G4K}
%% \label{??:9GKIQ}
%% \label{??:BLRSY}
%% \label{??:FBVXJ}
%% \label{??:O17ZL}
%% \label{??:IIHLA}
%% \label{??:FKVED}
%% \label{??:OHJEV}
%% \label{??:F0ZME}
%% \label{??:TMGPW}
%% \label{??:KKBBU}
%% \label{??:09S6S}
%% \label{??:8LTLJ}
%% \label{??:FRP5U}
%% \label{??:153BN}
%% \label{??:N23AQ}
%% \label{??:DX8VZ}
%% \label{??:CIBXT}
%% \label{??:USWMH}
%% \label{??:M7PFQ}
%% \label{??:TVP6Q}
%% \label{??:C1YQ4}
%% \label{??:XR18W}
%% \label{??:WNDWC}
%% \label{??:KUHTK}
%% \label{??:M4WUQ}
%% \label{??:31HWD}
%% \label{??:HEMOZ}
%% \label{??:DELSO}
%% \label{??:9JWAO}
%% \label{??:KCY55}
%% \label{??:47XSX}
%% \label{??:IRR24}
%% \label{??:60ISN}
%% \label{??:C01FZ}
%% \label{??:2CJY7}
%% \label{??:YP7NM}
%% \label{??:6T5VY}

% vim: set syntax=tex tabstop=2 shiftwidth=2 expandtab spell spelllang=pl:
