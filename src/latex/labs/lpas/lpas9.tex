\documentclass[paper=a4,DIV=12]{lpas}

\usepackage[polish]{babel}
\usepackage[T1]{fontenc}
\usepackage[utf8]{inputenc}
\usepackage{courier}
\usepackage{tgtermes,newtxtext,newtxmath}
\usepackage[]{hyperref}
\usepackage{natbib}

%%\everymath{\displaystyle}

\usepackage{bm}
\usepackage{amsmath,amsfonts}
\usepackage{mathtools}
\usepackage[pdftex]{graphicx}
\usepackage[titletoc,title]{appendix}
\usepackage{subcaption}
\usepackage{listings}
\usepackage{float}
\usepackage{tabularx}
\usepackage{tikz}
\usepackage{circuitikz}
\usetikzlibrary{arrows.meta,3d,patterns,calc,shapes.geometric}

\newcommand{\brm}[1]{\bm{\mathrm{#1}}}
\renewcommand{\arraystretch}{1.2}
\newcolumntype{L}[1]{>{\raggedright\arraybackslash}p{#1}}
\newcolumntype{C}[1]{>{\centering\arraybackslash}p{#1}}
\newcolumntype{R}[1]{>{\raggedleft\arraybackslash}p{#1}}

% commath provides \od, but the package is not available on my travis-ci setup
\newcommand{\od}[2]{\frac{\mathrm{d}#1}{\mathrm{d}#2}}
\newcommand{\odn}[3]{\frac{\mathrm{d}^{#1}#2}{\mathrm{d}{#3}^{#1}}}
\newcommand{\tod}[2]{\tfrac{\mathrm{d}#1}{\mathrm{d}#2}}
\newcommand{\todn}[3]{\tfrac{\mathrm{d}^{#1}#2}{\mathrm{d}{#3}^{#1}}}
% gensymb provides \degree command, but whole package for just one symbol?
\newcommand{\degree}{^{\circ}}

\lstset{%
  basicstyle=\footnotesize\ttfamily\selectfont,
  language=Matlab,
  inputencoding=utf8,
  extendedchars=true,
  frame=trBL
}

\newfloat{lstfloat}{htbp}{lop}
\floatname{lstfloat}{Listing}

\setcitestyle{numbers,square,comma}

\begin{document}


\serietitle{\\{\ }\\{\ }\\Laboratorium pomiarów, automatyki i~sterowania~I}

\title{\Large{Ćwiczenie nr 9}}

\subtitle{\huge{Badanie charakterystyk częstotliwościowych podstawowych członów automatyki}}

\author{\\Paweł Tomulik\\ Zakład Teorii Maszyn i Robotów\\ ITLiMS, MEIL, PW}
\date{}
\maketitle
\thispagestyle{empty}

\pagebreak
\tableofcontents
\pagebreak

\begin{abstract}
\section{Streszczenie}
\noindent Identyfikacja modeli układów dynamicznych ma istotne znaczenie
  w~projektowaniu układów sterowania. Odpowiednia metoda pomiarowa umożliwia
  rozpoznanie badanego obiektu i~określenie parametrów modelu reprezentującego
  ten obiekt w~projekcie. Przykładowo, dla podstawowych członów liniowych,
  można zdjąć charakterystyki częstotliwościowe przy użyciu odpowiednich metod
  pomiaru, a~następnie, na ich podstawie, rozpoznać typ elementu liniowego
  i~oszacować jego parametry. W~ten sposób powstaje model rzeczywistego
  obiektu, który może być uwzględniony przy doborze nastaw regulatora bądź
  w~dostrajaniu algorytmu sterowania tym obiektem.

  W~ramach wykładu, studenci poznają tzw. ,,podstawowe człony automatyki'' --
  elementy liniowe używane w~syntezie regulatorów czy w~modelowaniu prostych
  obiektów sterowania. Studenci poznają formuły opisujące transmitancje
  operatorowe i~widmowe oraz kształty teoretycznych charakterystyk
  amplitudowo-fazowych i~częstotliwościowych tychże członów.

  Niniejsze ćwiczenie jest praktycznym uzupełnieniem informacji z~wykładu
  i~umożliwia studentom zetknięcie się z~rzeczywistą realizacją poznanych na
  wykładzie modeli. Demonstruje również jedną z~metod wykreślania
  charakterystyk częstotliwościowych tych elementów, polegającą na pomiarze
  parametrów odpowiedzi {\em ustalonej} na wymuszenia sinusoidalne, oraz
  przykładową metodą określania parametrów tych modeli z~danych pomiarowych.
\end{abstract}

\section{Wprowadzenie}
\label{sec:AA9C2}

W~celu przygotowania się do uczestnictwa w~zajęciach laboratoryjnych, student
powinien zapoznać się w~pierwszej kolejności z~modelowym opisem eksperymentu
przedstawionym w~dodatku~\ref{sec:OD8DY} oraz planem eksperymentu
przedstawionym w~dodatku~\ref{sec:0DR40}.

\subsection{Cel ćwiczenia}
\label{sec:7HOCC}

Ćwiczenie ma na celu zapoznanie studentów z~jedną z~metod zdejmowania
charakterystyk częstotliwościowych wybranych ,,podstawowych członów
automatyki'' (układów LUS\footnote{LUS - Liniowy Układ Stacjonarny, określenie
odnosi się do układów liniowych, których parametry nie zmieniają się w~czasie.
Często też zwane układami LTI -- Linear Time-Invariant.}) oraz sposobem
zastosowania uzyskanych charakterystyk do identyfikacji modeli liniowych.
Metoda polega na pomiarze składowych {\em odpowiedzi ustalonej} na zadane
wymuszenia sinusoidalne o~różnych częstotliwościach.


\subsection{Badane układy}
\label{sec:ITWYS}

Stanowisko laboratoryjne umożliwia badanie jednego trzech członów liniowych,
zrealizowanych w~postaci elektronicznych obwodów RC/RLC (R -- rezystancja,
L -- indukcyjność, C -- pojemność):
%%%%%%%%%%%%%%%%%%%%%%%%%%%%%%%%%%%%%%%%%%%%%%%%%%%%%%%%%%%%%%%%%%%%%%%%%%%%%%
\begin{itemize}
  \item człony pierwszego rzędu (czwórniki RC) -- opisane w~dodatku~\ref{sec:LDB7J}:
    \begin{itemize}
      \item inercyjny -- opisany w~dodatku~\ref{sec:A9C2X},
      \item różniczkujący rzeczywisty -- opisany w~dodatku~\ref{sec:TYRDA}.
    \end{itemize}
  \item człon drugiego rzędu oscylacyjny lub inercyjny (czwórnik RLC opisany w~dodatku~\ref{sec:TEOBF}).
\end{itemize}
%%%%%%%%%%%%%%%%%%%%%%%%%%%%%%%%%%%%%%%%%%%%%%%%%%%%%%%%%%%%%%%%%%%%%%%%%%%%%%
W~ostatnim przypadku, człon oscylacyjny bądź inercyjny drugiego rzędu uzyskuje
się w~zależności od ustawionych przez użytkownika parametrów R, L, C.
W~praktyce nie wiadomo z~góry który z~członów uzyskano dopóki nie przeprowadzi
się eksperymentu bądź odpowiednich obliczeń z~użyciem nastawionych wartości R,
L, C (urządzenie laboratoryjne samo z~siebie nie daje takiej informacji).

\section{Przebieg ćwiczenia}
\label{sec:0DR40}

Procedurę opisaną w~niniejszym rozdziale należy przeprowadzić dwukrotnie -- raz
dla elementu I-go rzędu i~raz dla elementu II-go rzędu. Wyjątkiem jest
kalibracja opisana w~podrozdziale~\ref{sec:VDTSB}, którą wystarczy
przeprowadzić raz, na początku zajęć.

\subsection{Przygotowanie arkusza kalkulacyjnego}
\label{sec:S5JXT}

Studentom udostępniono arkusz kalkulacyjny \texttt{lpas9-template.ods},
spełniający rolę protokołu pomiarowego. Przystosowany jest on
do~przeprowadzenia jednego pełnego eksperymentu. W~przypadku badania następnego
typu członu w~ramach jednych zajęć, należy rozpocząć pracę na nowej kopii.

Arkusz posiada trzy zakładki
%%%%%%%%%%%%%%%%%%%%%%%%%%%%%%%%%%%%%%%%%%%%%%%%%%%%%%%%%%%%%%%%%%%%%%%%%%%%%%
\begin{itemize}
  \item \texttt{POM} -- tu należy wpisywać wyniki kalibracji oraz pomiarów
    wstępnych,
  \item \texttt{IRz} -- używany, gdy badany element jest członem I-go rzędu,
  \item \texttt{IIRz} -- używany, gdy badany element jest członu II-go rzędu.
\end{itemize}
%%%%%%%%%%%%%%%%%%%%%%%%%%%%%%%%%%%%%%%%%%%%%%%%%%%%%%%%%%%%%%%%%%%%%%%%%%%%%%

\subsection{Kalibracja}
\label{sec:VDTSB}

Kalibracja polega na wyzerowaniu urządzenia oraz ustaleniu współczynnika skali
(skalowanie).

\subsubsection{Zerowanie}
\label{sec:6OYTI}

Urządzenie pomiarowe należy wyzerować tak, aby przy braku sygnału oba
woltomierze wskazywały wartość możliwie najbliższą zeru (pokrętła \texttt{SET
ZERO R} i~\texttt{SET ZERO Q} przy odłączonym sygnale).

\subsubsection{Skalowanie}
\label{sec:XW8AC}

Skalowanie przeprowadza się dla ustalonej amplitudy sygnału. Po~przeprowadzeniu
skalowania \textbf{nie zmieniać amplitudy sygnału}. Z~powodu ograniczonej
wydajności prądowej generatora, \textbf{nie ustawiać amplitudy powyżej $2.5V$}
(proponuje się ustawienie amplitudy $\approx 2V$).
%%%%%%%%%%%%%%%%%%%%%%%%%%%%%%%%%%%%%%%%%%%%%%%%%%%%%%%%%%%%%%%%%%%%%%%%%%%%%%
\begin{enumerate}
  \item Ustawić amplitudę generowanego sygnału $\approx 2V$ a~częstotliwość
    z~zakresu $50\dots100\text{ Hz}$.
  \item Połączyć bezpośrednio wyjście generatora z~wejściem urządzenia
    pomiarowego.
  \item Wpisać napięcie $P_{mV}$ wskazywane przez woltomierz do~arkusza
    (\texttt{POM} $\rightarrow$ \texttt{Kalibracja} $\rightarrow \alpha$).
\end{enumerate}
%%%%%%%%%%%%%%%%%%%%%%%%%%%%%%%%%%%%%%%%%%%%%%%%%%%%%%%%%%%%%%%%%%%%%%%%%%%%%%

Skalowanie polega na ustaleniu współczynnika skali $\alpha$ umożliwiającego
przeliczanie napięć $P_{mV}$, $Q_{mV}$, wskazywanych przez woltomierze, na
wartości bezwymiarowe $P$, $Q$
%%%%%%%%%%%%%%%%%%%%%%%%%%%%%%%%%%%%%%%%%%%%%%%%%%%%%%%%%%%%%%%%%%%%%%%%%%%%%%
\begin{equation}
  \begin{aligned}
    & P = \frac{P_{mV}}{\alpha}, && Q = \frac{Q_{mV}}{\alpha}. &&
  \end{aligned}
  \label{eq:IP4D4}
\end{equation}
%%%%%%%%%%%%%%%%%%%%%%%%%%%%%%%%%%%%%%%%%%%%%%%%%%%%%%%%%%%%%%%%%%%%%%%%%%%%%%
Współczynnik $\alpha$ ustala się w~powyższej metodzie wg pomysłu
%%%%%%%%%%%%%%%%%%%%%%%%%%%%%%%%%%%%%%%%%%%%%%%%%%%%%%%%%%%%%%%%%%%%%%%%%%%%%%
\begin{equation}
  \alpha = \left.P_{mV}(\omega)\right|_{G(s) = 1},
  \label{eq:57L58}
\end{equation}
%%%%%%%%%%%%%%%%%%%%%%%%%%%%%%%%%%%%%%%%%%%%%%%%%%%%%%%%%%%%%%%%%%%%%%%%%%%%%%
gdzie $\left.P_{mV}(\omega)\right|_{G(s) = 1}$ jest wartością wskazywaną przez
woltomierz przy sygnale z~generatora wprowadzonym bezpośrednio na wejście
urządzenia pomiarowego (kabel $\implies G(s) = 1$). Częstotliwość sygnału
użytego przy skalowaniu, teoretycznie dowolna, powinna w~praktyce być dobrana
rozsądnie. Zbyt niska może powodować falowanie wskazań woltomierzy. Zbyt wysoka
może uwidocznić wpływ indukcyjności i~pojemności pasożytniczych. Proponuje się
użycie częstotliwości z~zakresu $50\dots100\text{Hz}$.

\subsection{Wstępny pomiar}
\label{sec:NU1XH}

Wstępny pomiar ma na celu uzyskanie zarysu charakterystyk oraz rozpoznanie typu
badanego elementu.
%%%%%%%%%%%%%%%%%%%%%%%%%%%%%%%%%%%%%%%%%%%%%%%%%%%%%%%%%%%%%%%%%%%%%%%%%%%%%%
\begin{enumerate}
  \item Włączyć badany układ w~tor pomiarowy.
  \item Wykonać serię pomiarów $P_{mV}$, $Q_{mV}$ przy różnych
    częstotliwościach sygnału (tabela~\ref{tab:D3Y3O}).
  \item Wyniki wpisywać do arkusza (\texttt{POM} $\rightarrow$ \texttt{Wstępny pomiar}).
\end{enumerate}
%%%%%%%%%%%%%%%%%%%%%%%%%%%%%%%%%%%%%%%%%%%%%%%%%%%%%%%%%%%%%%%%%%%%%%%%%%%%%%
W~zakładce \texttt{POM} znajdują się początkowo puste wykresy zarysu
charakterystyk, które będą się pojawiały w~trakcie wpisywania zmierzonych
napięć $P_{mV}, Q_{mV}$.

\subsection{Rozpoznanie typu członu}
\label{sec:NVX7M}

Po przeprowadzeniu wstępnego pomiaru należy rozpoznać typ badanego elementu na
podstawie cech szczególnych zauważonych na zarysach charakterystyk
(patrz dodatek~\ref{sec:AP5MQ}). Jeśli jest to człon I-go rzędu, dalsze
protokołowanie należy prowadzić w~zakładce \texttt{IRz}. Jeśli jest to człon
II-go rzędu, to przejść do zakładki \texttt{IIRz}.

\subsection{Człon I-go rzędu}
\label{sec:3Q2SV}

Należy przejść do zakładki \texttt{IRz} i~ustawić typ rozpoznanego członu
(\texttt{Identyfikacja} $\rightarrow$ \texttt{Typ[I/R]}):
%%%%%%%%%%%%%%%%%%%%%%%%%%%%%%%%%%%%%%%%%%%%%%%%%%%%%%%%%%%%%%%%%%%%%%%%%%%%%%
\begin{itemize}
  \item \texttt{I} -- dla członu inercyjnego,
  \item \texttt{R} -- dla członu różniczkującego rzeczywistego.
\end{itemize}
%%%%%%%%%%%%%%%%%%%%%%%%%%%%%%%%%%%%%%%%%%%%%%%%%%%%%%%%%%%%%%%%%%%%%%%%%%%%%%

\subsubsection{Poszukiwanie $f_s$ metodą bisekcji}
\label{sec:WM5KT}

Metodą bisekcji ustalić częstotliwość graniczną $f_s$:
%%%%%%%%%%%%%%%%%%%%%%%%%%%%%%%%%%%%%%%%%%%%%%%%%%%%%%%%%%%%%%%%%%%%%%%%%%%%%%
\begin{enumerate}
  \item W~zakładce \texttt{POM} odnaleźć punkty pomiarowe $L$ i~$R$
    (rysunek~\ref{fig:UIR99}) sąsiadujące z~poszukiwanym punktem granicznym.
    Oznaczmy roboczo $L = (f_L, P_L, Q_L)$ i $R = (f_R, P_R, Q_R)$.
  \item Wpisać współrzędne $(f_L, P_L, Q_L)$ i $(f_R, P_R, Q_R)$ z~zakładki
    \texttt{POM} w~odpowiednie komórki w~zakładce \texttt{IRz}:
    \texttt{Bisekcja} $\rightarrow$ \{\texttt{L}$/$\texttt{R}\} $\rightarrow$
    (\texttt{f}, \texttt{PmV}, \texttt{QmV})  (wiersz \texttt{n=0}).
  \item W~sekcji \texttt{Bisekcja} $\rightarrow$ \texttt{M}, w~kolumnie
    \texttt{f}, pojawi się nowa częstotliwość pomiarowa (środkowa).
  \item Na generatorze ustawić nową częstotliwość środkową \texttt{f}, zmierzyć
    napięcia $(P_{mV}, Q_{mV})$, wartości wpisać w~odpowiednie komórki w~sekcji
    \texttt{Bisekcja} $\rightarrow$ \texttt{M} w~tym samym wierszu.
  \item W~następnym wierszu pojawi się częstotliwość \texttt{f} dla następnej
    iteracji. Jeśli nie różni się istotnie od obecnej, zakończyć algorytm.
    W~przeciwnym razie, przejść do następnego wiersza i~kontynuować wykonanie
    algorytmu od punktu 4 powyżej.
\end{enumerate}
%%%%%%%%%%%%%%%%%%%%%%%%%%%%%%%%%%%%%%%%%%%%%%%%%%%%%%%%%%%%%%%%%%%%%%%%%%%%%%

Algorytm zwykle zbiega się w~kilku iteracjach. Częstotliwość graniczną
odnalezioną powyższym algorytmem należy wpisać w~komórkę $\texttt{f}_\texttt{s}$ w~sekcji
\texttt{Identyfikacja}.

\subsubsection{Dodatkowe pomiary -- uzupełnienie charakterystyk}
\label{sec:09H54}

Po uzupełnieniu pola $\texttt{f}_\texttt{s}$ w~sekcji \texttt{Identyfikacja}
w~zakładce \texttt{IRz} pojawią się częstotliwości $\texttt{f}_\texttt{i}$
dla pomiarów uzupełniających (sekcja \texttt{Dodatkowy pomiar}). Wartości
generowane są wg~pomysłu opisanego w~dodatku~\ref{sec:2C7WK}
(wzór~\eqref{eq:9R1WL}). Należy przeprowadzić pomiary $(P_{mV}, Q_{mV})$ dla
tych częstotliwości, wpisując wyniki w~kolumnach $\texttt{P}_\texttt{mV}$,
$\texttt{Q}_\texttt{mV}$. Proponuje się rozpocząć pomiary od~wiersza
\texttt{i=0} (w~połowie tabeli) i~kontynuować dla sąsiednich wierszy oddalając
się w~górę i~w~dół tabeli.


\subsection{Człon II-go rzędu}
\label{sec:D53ZC}

Opracowanie danych pomiarowych należy prowadzić w~zakładce \texttt{IIRz}.

\subsubsection{Poszukiwanie $f_0$ metodą bisekcji}
\label{sec:WM5KT}

Metodą bisekcji ustalić częstotliwość graniczną $f_0$:
%%%%%%%%%%%%%%%%%%%%%%%%%%%%%%%%%%%%%%%%%%%%%%%%%%%%%%%%%%%%%%%%%%%%%%%%%%%%%%
\begin{enumerate}
  \item W~zakładce \texttt{POM} odnaleźć punkty pomiarowe $L$ i~$R$
    (rysunek~\ref{fig:WC8TU}) sąsiadujące z~poszukiwanym punktem granicznym.
    Oznaczmy roboczo $L = (f_L, P_L, Q_L)$ i $R = (f_R, P_R, Q_R)$.
  \item Wpisać współrzędne $(f_L, P_L)$ i $(f_R, P_R)$ z~zakładki
    \texttt{POM} w~odpowiednie komórki w~zakładce \texttt{IIRz}:
    \texttt{Bisekcja} $\rightarrow$ \{\texttt{L}$/$\texttt{R}\} $\rightarrow$
    (\texttt{f}, \texttt{PmV})  (wiersz \texttt{n=0}).
  \item W~sekcji \texttt{Bisekcja} $\rightarrow$ \texttt{M}, w~kolumnie
    \texttt{f}, pojawi się nowa częstotliwość pomiarowa (środkowa).
  \item Na generatorze ustawić nową częstotliwość środkową \texttt{f}, zmierzyć
    napięcie $P_{mV}$, wartość wpisać w~odpowiednią komórkę w~sekcji
    \texttt{Bisekcja} $\rightarrow$ \texttt{M} w~tym samym wierszu.
  \item W~następnym wierszu pojawi się częstotliwość \texttt{f} dla następnej
    iteracji. Jeśli nie różni się istotnie od obecnej, zanotować wskazywaną
    wartość $Q_{mV}$ i~zakończyć algorytm. W~przeciwnym razie, przejść do
    następnego wiersza i~kontynuować wykonanie algorytmu od punktu 4 powyżej.
\end{enumerate}
%%%%%%%%%%%%%%%%%%%%%%%%%%%%%%%%%%%%%%%%%%%%%%%%%%%%%%%%%%%%%%%%%%%%%%%%%%%%%%

Częstotliwość $f_0$ odnalezioną powyższym algorytmem należy wpisać
w~komórkę $\texttt{f}_\texttt{0}$ w~sekcji \texttt{"o - obliczenia wg
($\omega_0$, A($\omega_0$))"}.

\subsubsection{Wyznaczenie współczynnika tłumienia $\zeta$}
\label{eq:A20W9}

W~komórce \texttt{A($\omega_0$)} sekcji \texttt{"o - obliczenia wg ($\omega_0$,
A($\omega_0$))"} należy wpisać wartość obliczoną jako
%%%%%%%%%%%%%%%%%%%%%%%%%%%%%%%%%%%%%%%%%%%%%%%%%%%%%%%%%%%%%%%%%%%%%%%%%%%%%%
\begin{equation}
  A(\omega_0) = \frac{|Q_{mV}(\omega_0)|}{\alpha},
  \label{eq:A7ZD6}
\end{equation}
%%%%%%%%%%%%%%%%%%%%%%%%%%%%%%%%%%%%%%%%%%%%%%%%%%%%%%%%%%%%%%%%%%%%%%%%%%%%%%
gdzie $\alpha$ jest współczynnikiem skali opisanym
w~podrozdziale~\ref{sec:XW8AC}, a~$Q_{mV}(\omega_0)$ napięciem zanotowanym
w~ostatniej iteracji algorytmu bisekcji~\ref{sec:WM5KT}


\subsubsection{Dodatkowe pomiary -- uzupełnienie charakterystyk}
\label{sec:09H54}

Po uzupełnieniu pól $\texttt{f}_\texttt{s}$ i $\zeta$ w~sekcji
\texttt{Identyfikacja} w~zakładce \texttt{IIRz} pojawią się częstotliwości
$\texttt{f}_\texttt{i}$ dla pomiarów uzupełniających (sekcja \texttt{Dodatkowy
pomiar}). Wartości generowane są wg~pomysłu opisanego w~dodatku~\ref{sec:X9JYP}
(wzór~\eqref{eq:1GXRP}). Należy przeprowadzić pomiary $(P_{mV}, Q_{mV})$ dla
tych częstotliwości, wpisując wyniki w~kolumnach $\texttt{P}_\texttt{mV}$,
$\texttt{Q}_\texttt{mV}$. Proponuje się rozpocząć pomiary od~wiersza
\texttt{i=0} (w~połowie tabeli) i~kontynuować dla sąsiednich wierszy oddalając
się w~górę i~w~dół tabeli.

\subsubsection{Rezonans}
\label{sec:0NZQS}

Jeśli na charakterystyce amplitudowej $A(\omega)$ widoczny jest szczyt
rezonansowy, poleca się dodatkowo wyznaczyć $\omega_0$ i~$\zeta$ zgodnie
sugestią zawartą w~dodatku~\ref{sec:U6GVM}, tj. na podstawie wartości
szczytowych:

\begin{enumerate}
  \item W~sekcji \texttt{Dodatkowy pomiar} zakładki \texttt{IIRz} odnaleźć
    szczytową (najwyższą) wartość amplitudy $A_r$ (kolumna \texttt{A($\omega_i$)})
    i~odpowiadającą jej częstotliwość $f_r$ (kolumna $\texttt{f}_{\texttt{i}}$).
  \item Wartości $f_r$ i $A_r$ wpisać do odpowiednich komórek w~sekcji
    \texttt{"r - obliczenia wg ($\omega_r$, A($\omega_r$))"}.
\end{enumerate}
Pulsacja drgań własnych $\omega_0$ i~współczynnik tłumienia $\zeta$ wyliczą się
po~uzupełnieniu powyższych pól.

\section{Przygotowanie sprawozdania}
\label{sec:TFLLK}

Sprawozdanie powinno zawierać, dla każdego z~badanych na zajęciach członów:
%%%%%%%%%%%%%%%%%%%%%%%%%%%%%%%%%%%%%%%%%%%%%%%%%%%%%%%%%%%%%%%%%%%%%%%%%%%%%%
\begin{enumerate}
  \item Wykresy charakterystyk z~pomiaru wstępnego (takie jak w~zakładce
    \texttt{POM}, oś częstotliwości w~Hz).
  \item Informację jaki typ członu rozpoznano na podstawie charakterystyk, z~uzasadnieniem.
  \item Wyniki identyfikacji parametrycznej metodą bisekcji ($f_s$ lub $(f_0, \zeta)$ -- zależnie od przypadku).
  \item Znormalizowane wykresy charakterystyk Bodego oraz charakterystykę
    Nyquista, następująco\footnote{Wykresy tego typu są dostępne w~odpowiedniej
    zakładce \texttt{IRz}$/$\texttt{IIRz} opracowanego na zajęciach arkusza.}
    \begin{itemize}
      \item na każdym wykresie linią ciągłą wyrysowana charakterystyka
        modelowa, tj. wygenerowana ze wzorów na $G(j\omega)$, z~użyciem
        parametrów ustalonych metodą identyfikacji (z~bisekcji),
      \item na tym samym wykresie, w~postaci chmury punktów, naniesione
        wartości pochodzące z~pomiarów (pomiary wstępny i~dodatkowy),
      \item w~przypadku członu II-go rzędu, również charakterystyka
        nielogarytmiczna $A(\omega)$.
    \end{itemize}
    Oś częstotliwości bezwymiarowa, skala logarytmiczna (jak na wykresach
    z~zakładek \texttt{IRz}$/$\texttt{IIRz}).
  \item W~przypadku członu oscylacyjnego z~rezonansem, wyniki obliczeń
    $\omega_0$ i~$\zeta$ z~podrozdziału~\ref{sec:0NZQS}, łącznie z~wartością
    wskaźnika uwarunkowania $\text{cond}_{\zeta}$.
  \item Wyliczone wartości teoretyczne parametrów modelu ($f_s$ lub
    $(f_0,\zeta)$ -- zależnie od przypadku) na podstawie podanych przez
    prowadzącego ustawień R, L, C. Porównanie z~wartościami uzyskanymi na
    drodze identyfikacji parametrycznej.
  \item Wnioski krytyczne. Odnieść się do zauważonych nieprawidłowości, podać
    ich przyczyny. Skomentować rozbieżności między modelem teoretycznym
    a~rezultatem identyfikacji.
\end{enumerate}
%%%%%%%%%%%%%%%%%%%%%%%%%%%%%%%%%%%%%%%%%%%%%%%%%%%%%%%%%%%%%%%%%%%%%%%%%%%%%%

\clearpage

\begin{appendices}
  \section{Opis modelowy eksperymentu}
  \label{sec:OD8DY}

  Niniejszy dodatek wprowadza modele teoretyczne układów badanych podczas
  ćwiczenia laboratoryjnego. Właściwości członów liniowych występujących na
  stanowisku laboratoryjnym opisano poprzez podanie transmitancji oraz
  wyrysowanie znormalizowanych charakterystyk amplitudowo-fazowych
  i~częstotliwościowych. Wskazano pewne cechy szczególne charakterystyk, które
  wykorzystuje się do identyfikacji układów wg metodyki przedstawionej
  w~głównej części instrukcji.

  \subsection{Wstęp teoretyczny}
  \label{sec:AIMJR}

  Badany układ traktujemy jako ,,czarną skrzynkę'' posiadającą wejście i~wyjście.
  Jeśli na wejście podamy zmienny w~czasie sygnał $u(t)$, to na wyjściu
  zaobserwujemy sygnał $y(t)$.
  %%%%%%%%%%%%%%%%%%%%%%%%%%%%%%%%%%%%%%%%%%%%%%%%%%%%%%%%%%%%%%%%%%%%%%%%%%%%%%
  \begin{figure}[htbp]
    \centering
    \begin{tikzpicture}[scale=1.0]
      \node[draw,inner sep=1em] (G) {$G\left(s\right)$};
      \draw[{Stealth}-] (G.west) -- ++(-1cm, 0cm) node[left] {$U\left(s\right)$};
      \draw[-{Stealth}] (G.east) -- ++( 1cm, 0cm) node[right] {$Y\left(s\right)$};
    \end{tikzpicture}
    \caption{Element badany, jego transmitancja $G(s)$, sygnały.}
    \label{fig:6TEOB}
  \end{figure}
  %%%%%%%%%%%%%%%%%%%%%%%%%%%%%%%%%%%%%%%%%%%%%%%%%%%%%%%%%%%%%%%%%%%%%%%%%%%%%%
  W~rozważaniach teoretycznych posługujemy się zwyczajowo reprezentacją
  $U = \mathscr{L}\{u\}$ i~$Y = \mathscr{L}\{y\}$ powyższych sygnałów
  w~dziedzinie operatorowej (transformata Laplace'a sygnałów $u$ i $y$).
  W~ogólnych rozważaniach używa się też pojęcia transmitancji operatorowej
  (funkcji przejścia) zdefiniowanej jako
  %%%%%%%%%%%%%%%%%%%%%%%%%%%%%%%%%%%%%%%%%%%%%%%%%%%%%%%%%%%%%%%%%%%%%%%%%%%%%%
  \begin{equation}
    G\left(s\right) = \frac{Y(s)}{U(s)}, \;\;
    s = \left( \sigma + j \omega \right) \in S \subset \mathbb{C}.
    \label{eq:34IJQ}
  \end{equation}
  %%%%%%%%%%%%%%%%%%%%%%%%%%%%%%%%%%%%%%%%%%%%%%%%%%%%%%%%%%%%%%%%%%%%%%%%%%%%%%
  Symbole $U\left(s\right)$ i~$Y\left(s\right)$ reprezentują odpowiednio sygnały
  wejściowy $u(t)$ i~wyjściowy $y(t)$ wyrażone w~dziedzinie
  operatorowej (Laplace'a), a funkcja przejścia $G(s)$ wiąże ze sobą te sygnały --
  odzwierciedla zmianę sygnału w~wyniku jego przejścia przez opisywany element.
  Analiza przy użyciu transmitancji operatorowej {\em obejmuje jedynie odpowiedź
  ustaloną} na zadane wymuszenie, w teorii ma więc zastosowanie do opisu układów
  liniowych poddanych wymuszeniom przy {\em zerowych warunkach początkowych}
  (zerowe warunki początkowe gwarantują niewystępowanie składowej przejściowej
  odpowiedzi).

  Do opisu układów LUS\footnote{LUS - Liniowy Układ Stacjonarny, określenie
  odnosi się do układów liniowych, których parametry nie zmieniają się
  w~czasie. Często też zwane układami LTI -- Linear Time-Invariant.} ,
  poddanych wymuszeniom sinusoidalnym, stosuje się zwyczajowo {\em transmitancję
  widmową}, która (technicznie) powstaje przez podstawienie $s = j\omega$
  w~definicji $G(s)$
  %%%%%%%%%%%%%%%%%%%%%%%%%%%%%%%%%%%%%%%%%%%%%%%%%%%%%%%%%%%%%%%%%%%%%%%%%%%%%%
  \begin{equation}
    G\left(j \omega\right) = G\left(s\right)|_{s = j \omega},
    \;\; \omega \in \mathbb{R}.
    \label{eq:WEMIV}
  \end{equation}
  %%%%%%%%%%%%%%%%%%%%%%%%%%%%%%%%%%%%%%%%%%%%%%%%%%%%%%%%%%%%%%%%%%%%%%%%%%%%%%
  Znając wartości $G(j\omega) \in \mathbb{C}$ dla wszystkich $\omega$
  w~interesującym przedziale, można wykreślić charakterystykę amplitudowo-fazową
  (Nyquista) nanosząc bezpośrednio na płaszczyznę zespoloną punkty
  %%%%%%%%%%%%%%%%%%%%%%%%%%%%%%%%%%%%%%%%%%%%%%%%%%%%%%%%%%%%%%%%%%%%%%%%%%%%%%
  \begin{equation}
    G(j\omega) = P(\omega) +j Q(\omega) \in \mathbb{C}.
    \label{eq:Z6A8G}
  \end{equation}
  %%%%%%%%%%%%%%%%%%%%%%%%%%%%%%%%%%%%%%%%%%%%%%%%%%%%%%%%%%%%%%%%%%%%%%%%%%%%%%
  Tę samą informację można wyrazić w~postaci charakterystyk częstotliwościowych
  (Bodego), gdzie na oddzielnych wykresach przedstawiamy moduł $M(\omega)$ w~dB
  i~przesunięcie fazowe~$\varphi(\omega)$
  %%%%%%%%%%%%%%%%%%%%%%%%%%%%%%%%%%%%%%%%%%%%%%%%%%%%%%%%%%%%%%%%%%%%%%%%%%%%%%
  \begin{equation}
    \begin{aligned}
      &
      M(\omega) = 20 \log \left|G(j\omega)\right|,
      &&
      \varphi(\omega) = \text{arg}{\left(G(j\omega)\right)}
      &
    \end{aligned}
    \label{eq:YT6UA}
  \end{equation}
  %%%%%%%%%%%%%%%%%%%%%%%%%%%%%%%%%%%%%%%%%%%%%%%%%%%%%%%%%%%%%%%%%%%%%%%%%%%%%%
  w~funkcji pulsacji $\omega$. Oczywiście, sam moduł transmitancji $|G(j\omega)|$
  wyznaczany jest jako
  %%%%%%%%%%%%%%%%%%%%%%%%%%%%%%%%%%%%%%%%%%%%%%%%%%%%%%%%%%%%%%%%%%%%%%%%%%%%%%
  \begin{equation}
    A(\omega) = |G(j\omega)| = \sqrt{P^2(\omega) + Q^2(\omega)},
    \label{eq:S278F}
  \end{equation}
  %%%%%%%%%%%%%%%%%%%%%%%%%%%%%%%%%%%%%%%%%%%%%%%%%%%%%%%%%%%%%%%%%%%%%%%%%%%%%%
  a~przesunięcie fazowe~$\varphi(\omega)$ można wyliczyć, używając funkcji
  $\text{atan2}(y,x)$
  %%%%%%%%%%%%%%%%%%%%%%%%%%%%%%%%%%%%%%%%%%%%%%%%%%%%%%%%%%%%%%%%%%%%%%%%%%%%%%
  \begin{equation}
    \varphi(\omega) = \frac{180\degree}{\pi}\text{atan2}(Q(\omega), P(\omega)).
    \label{eq:XDQ8Q}
  \end{equation}
  %%%%%%%%%%%%%%%%%%%%%%%%%%%%%%%%%%%%%%%%%%%%%%%%%%%%%%%%%%%%%%%%%%%%%%%%%%%%%%

  \subsection{Człony pierwszego rzędu}
  \label{sec:LDB7J}

  Członem liniowym pierwszego rzędu nazywamy układ opisany liniowym równaniem
  różniczkowym zwyczajnym pierwszego rzędu. Stanowisko laboratoryjne wyposażone
  jest w~dwa typy tych członów -- inercyjny i~różniczkujący rzeczywisty. Ich
  modele opisano w~podrozdziałach \ref{sec:A9C2X} i~\ref{sec:TYRDA}.
  W~niniejszym opracowaniu, model członu liniowego I-go rzędu posiada dokładnie
  jeden parametr\footnote{W~ogólności, model członu I-go rzędu jest określony
  dwoma parametrami (np. stała czasowa $T$ i~wzmocnienie $k$). W~niniejszym
  opracowaniu zakłada się jednak $k=1$, co~jest cechą specyficzną czwórników RC
  użytych w~ćwiczeniu.}. Identyfikacja modelu sprowadza się więc do~wyznaczenia
  jednego tylko parametru (np. pulsacji granicznej $\omega_s$).

  Omawiane tutaj człony liniowe I-go rzędu mają pewne cechy charakterystyczne:
  \begin{enumerate}
    \item Wielomian charakterystyczny ma postać $H(s) = Ts+1$ i~ma zawsze
      dokładnie jeden pierwiastek $s_0 = -1/T \in \mathbb{R}$ (biegun
      transmitancji operatorowej). Związana jest z~nim tzw pulsacja graniczna
  %%%%%%%%%%%%%%%%%%%%%%%%%%%%%%%%%%%%%%%%%%%%%%%%%%%%%%%%%%%%%%%%%%%%%%%%%%%%%%
      \begin{equation}
        \omega_s = \frac{1}{T}.
        \label{eq:3UCS1}
      \end{equation}
  %%%%%%%%%%%%%%%%%%%%%%%%%%%%%%%%%%%%%%%%%%%%%%%%%%%%%%%%%%%%%%%%%%%%%%%%%%%%%%
    \item Charakterystyka $M(\omega)$ zbiega się z~asymptotą o~nachyleniu
      $-20\text{ dB/dek}$ (człon inercyjny) lub $+20\text{ dB/dek}$ (człon
      różniczkujący rzeczywisty) przechodzącą przez punkt
      $(\omega_s,0\text{dB})$.
    \item Przy pulsacji granicznej $\omega_s$ występuje tzw. 3-decybelowy spadek
      modułu\footnote{Dokładna wartość to $M(\omega_s) = 20 \log{(\sqrt{2}/2)}
      \approx -3.0103\text{ dB}$.}: $M(\omega_s) \approx -3\text{ dB}$,
      a~przesunięcie fazowe przyjmuje wartość $\varphi(\omega_s) = -45\degree$
      (człon inercyjny) lub $\varphi(\omega_s) = +45\degree$ (człon
      różniczkujący rzeczywisty) -- rysunki~\ref{fig:V0CUU} i~\ref{fig:YZX1R}.
  \end{enumerate}
  Są to cechy niezmienne, umożliwiające rozpoznanie członu liniowego I-go rzędu
  i~określenie parametru $\omega_s$ poprzez analizę charakterystyk
  doświadczalnych.

  \subsubsection{Człon inercyjny pierwszego rzędu}
  \label{sec:A9C2X}

  Człon inercyjny I-go rzędu zrealizowany jest w~postaci czwórnika RC
  wg~schematu z~rysunku~\ref{fig:VHF3V}.
  %%%%%%%%%%%%%%%%%%%%%%%%%%%%%%%%%%%%%%%%%%%%%%%%%%%%%%%%%%%%%%%%%%%%%%%%%%%%%%
  \begin{figure}[H]
    \begin{center}
      \begin{circuitikz}[european]
        \draw (0,1)
          to[short, o-, i=$i$] (1,1)
          to[R, l=$R$] (2,1)
          to[short] (3,1)
          to[C, i=$i$, l=$C$] (3,-1)
          to[short, -o] (0,-1)
        ;

        \node at (0,0) {$u(t)$}
        ;

        \draw(3,-1)
          to[short,*-o] (5,-1)
        ;

        \node at (5,0) {$y(t)$}
        ;

        \draw (5,1)
          to[short,o-*] (3,1)
        ;

        \node at (8, 0) {$\begin{aligned}
          & i(t) = C \od{y(t)}{t} & \\
          & R i(t) + y(t) = u(t) &
        \end{aligned}$};
      \end{circuitikz}
    \end{center}
    \caption{Czwórnik RC jako człon inercyjny pierwszego rzędu.}
    \label{fig:VHF3V}
  \end{figure}
  %%%%%%%%%%%%%%%%%%%%%%%%%%%%%%%%%%%%%%%%%%%%%%%%%%%%%%%%%%%%%%%%%%%%%%%%%%%%%%
  Sygnałem wejściowym $u(t)$ jest napięcie mierzone na zaciskach wejściowych (po
  lewej) a~za sygnał wyjściowy $y(t)$ obrano napięcie na zaciskach wyjściowych
  (po prawej). Równanie różniczkowe \eqref{eq:U35FU}
  %%%%%%%%%%%%%%%%%%%%%%%%%%%%%%%%%%%%%%%%%%%%%%%%%%%%%%%%%%%%%%%%%%%%%%%%%%%%%%
  \begin{equation}
    R C \od{y(t)}{t} + y(t) = u(t)
    \label{eq:U35FU}
  \end{equation}
  %%%%%%%%%%%%%%%%%%%%%%%%%%%%%%%%%%%%%%%%%%%%%%%%%%%%%%%%%%%%%%%%%%%%%%%%%%%%%%
  wyraża bilans napięć i~prądów chwilowych. Odpowiadająca mu transmitancja
  operatorowa to
  %%%%%%%%%%%%%%%%%%%%%%%%%%%%%%%%%%%%%%%%%%%%%%%%%%%%%%%%%%%%%%%%%%%%%%%%%%%%%%
  \begin{equation}
    \begin{aligned}
      &
      G(s) = \frac{1}{1 + Ts},
      & &
      T = R C,
      &
    \end{aligned}
    \label{eq:92W6D}
  \end{equation}
  %%%%%%%%%%%%%%%%%%%%%%%%%%%%%%%%%%%%%%%%%%%%%%%%%%%%%%%%%%%%%%%%%%%%%%%%%%%%%%
  gdzie $R$ jest rezystancją opornika wyrażoną w~$\Omega$ (Ohm), $C$
  pojemnością kondensatora w~$F$ (Farad), zaś~$T$ -- stałą czasową układu.
  Przez podstawienie $s=j\omega$ do~\eqref{eq:92W6D} otrzymujemy transmitancję
  widmową
  %%%%%%%%%%%%%%%%%%%%%%%%%%%%%%%%%%%%%%%%%%%%%%%%%%%%%%%%%%%%%%%%%%%%%%%%%%%%%%
  \begin{equation}
      G(j\omega)
      = \frac{1}{1 + j T\omega}
      = \underbrace{\frac{1}{1 + (T\omega)^2}}_{P(\omega)}
      + j \underbrace{\frac{-T\omega}{1 + (T\omega)^2}}_{Q(\omega)}.
    \label{eq:11SSB}
  \end{equation}
  %%%%%%%%%%%%%%%%%%%%%%%%%%%%%%%%%%%%%%%%%%%%%%%%%%%%%%%%%%%%%%%%%%%%%%%%%%%%%%

  Charakterystykę amplitudowo-fazową (Nyquista) członu inercyjnego przedstawiono
  na rysunku~\ref{fig:5Y0UB}.
  %%%%%%%%%%%%%%%%%%%%%%%%%%%%%%%%%%%%%%%%%%%%%%%%%%%%%%%%%%%%%%%%%%%%%%%%%%%%%%
  \begin{figure}[H]
    \centering
    \input{lpas9/inert/nyquist.tex}
    \caption{Charakterystyka amplitudowo-fazowa dla członu inercyjnego 1-go rzędu.}
    \label{fig:5Y0UB}
  \end{figure}
  %%%%%%%%%%%%%%%%%%%%%%%%%%%%%%%%%%%%%%%%%%%%%%%%%%%%%%%%%%%%%%%%%%%%%%%%%%%%%%

  Charakterystyki częstotliwościowe (Bodego) zamieszczono na rysunku~\ref{fig:V0CUU}.
  %%%%%%%%%%%%%%%%%%%%%%%%%%%%%%%%%%%%%%%%%%%%%%%%%%%%%%%%%%%%%%%%%%%%%%%%%%%%%%
  \begin{figure}[H]
    \centering
    \input{lpas9/inert/bode.tex}
    \caption{Charakterystyki Bodego dla członu inercyjnego 1-go rzędu.}
    \label{fig:V0CUU}
  \end{figure}
  %%%%%%%%%%%%%%%%%%%%%%%%%%%%%%%%%%%%%%%%%%%%%%%%%%%%%%%%%%%%%%%%%%%%%%%%%%%%%%

  \subsubsection{Człon różniczkujący rzeczywisty}
  \label{sec:TYRDA}

  Człon różniczkujący rzeczywisty zrealizowany jest w~postaci czwórnika RC
  z~rysunku~\ref{fig:Z5EXB}.
  %%%%%%%%%%%%%%%%%%%%%%%%%%%%%%%%%%%%%%%%%%%%%%%%%%%%%%%%%%%%%%%%%%%%%%%%%%%%%%
  \begin{figure}[H]
    \begin{center}
      \begin{circuitikz}[european]
        \draw (0,1)
          to[short, o-, i=$i$] (1,1)
          to[C, l=$C$] (2,1)
          to[short] (3,1)
          to[R, l=$R$, i=$i$] (3,-1)
          to[short, -o] (0,-1)
        ;

        \node at (0,0) {$u(t)$}
        ;

        \draw(3,-1)
          to[short,*-o] (5,-1)
        ;

        \node at (5,0) {$y(t)$}
        ;

        \draw (5,1)
          to[short,o-*] (3,1)
        ;

        \node at (9,0) {$\begin{aligned}
          & i(t) = C \od{(u(t) - y(t))}{t} & \\
          & y(t) = R i(t) &
        \end{aligned}$};
      \end{circuitikz}
    \end{center}
    \caption{Czwórnik RC jako człon różniczkujący rzeczywisty}
    \label{fig:Z5EXB}
  \end{figure}
  %%%%%%%%%%%%%%%%%%%%%%%%%%%%%%%%%%%%%%%%%%%%%%%%%%%%%%%%%%%%%%%%%%%%%%%%%%%%%%

  Równanie różniczkowe~\eqref{eq:HXCTX}
  %%%%%%%%%%%%%%%%%%%%%%%%%%%%%%%%%%%%%%%%%%%%%%%%%%%%%%%%%%%%%%%%%%%%%%%%%%%%%%
  \begin{equation}
    R C \od{y(t)}{t} + y(t) = R C \od{u(t)}{t}
    \label{eq:HXCTX}
  \end{equation}
  %%%%%%%%%%%%%%%%%%%%%%%%%%%%%%%%%%%%%%%%%%%%%%%%%%%%%%%%%%%%%%%%%%%%%%%%%%%%%%
  wyraża bilans napięć i~prądów chwilowych. Odpowiadająca mu transmitancja
  operatorowa to
  %%%%%%%%%%%%%%%%%%%%%%%%%%%%%%%%%%%%%%%%%%%%%%%%%%%%%%%%%%%%%%%%%%%%%%%%%%%%%%
  \begin{equation}
    \begin{aligned}
      &
      G(s) = \frac{Ts}{1 + Ts},
      & &
      T = R C,
      &
    \end{aligned}
    \label{eq:1PZGG}
  \end{equation}
  %%%%%%%%%%%%%%%%%%%%%%%%%%%%%%%%%%%%%%%%%%%%%%%%%%%%%%%%%%%%%%%%%%%%%%%%%%%%%%
  gdzie $R$ jest rezystancją opornika wyrażoną w~$\Omega$ (Ohm), $C$
  pojemnością kondensatora w~$F$ (Farad), zaś~$T$ -- stałą czasową układu.
  Przez podstawienie $s=j\omega$ do~\eqref{eq:1PZGG} otrzymujemy transmitancję
  widmową
  %%%%%%%%%%%%%%%%%%%%%%%%%%%%%%%%%%%%%%%%%%%%%%%%%%%%%%%%%%%%%%%%%%%%%%%%%%%%%%
  \begin{equation}
      G(j\omega)
      = \frac{jT\omega}{1 + j T\omega}
      = \underbrace{\frac{(T\omega)^2}{1 + (T\omega)^2}}_{P(\omega)}
      + j \underbrace{\frac{T\omega}{1 + (T\omega)^2}}_{Q(\omega)}.
    \label{eq:9KF4R}
  \end{equation}
  %%%%%%%%%%%%%%%%%%%%%%%%%%%%%%%%%%%%%%%%%%%%%%%%%%%%%%%%%%%%%%%%%%%%%%%%%%%%%%

  Charakterystykę amplitudowo-fazową członu różniczkującego
  rzeczywistego przedstawia rysunek~\ref{fig:4HNG2}.
  %%%%%%%%%%%%%%%%%%%%%%%%%%%%%%%%%%%%%%%%%%%%%%%%%%%%%%%%%%%%%%%%%%%%%%%%%%%%%%
  \begin{figure}[H]
    \centering
    \input{lpas9/deriv/nyquist.tex}
    \caption{Charakterystyka amplitudowo-fazowa dla członu różniczkującego rzeczywistego.}
    \label{fig:4HNG2}
  \end{figure}
  %%%%%%%%%%%%%%%%%%%%%%%%%%%%%%%%%%%%%%%%%%%%%%%%%%%%%%%%%%%%%%%%%%%%%%%%%%%%%%

  Charakterystyki częstotliwościowe (Bodego) zamieszczono na rysunku~\ref{fig:YZX1R}.
  %%%%%%%%%%%%%%%%%%%%%%%%%%%%%%%%%%%%%%%%%%%%%%%%%%%%%%%%%%%%%%%%%%%%%%%%%%%%%%
  \begin{figure}[H]
    \centering
    \input{lpas9/deriv/bode.tex}
    \caption{Charakterystyki Bodego dla członu różniczkującego rzeczywistego.}
    \label{fig:YZX1R}
  \end{figure}
  %%%%%%%%%%%%%%%%%%%%%%%%%%%%%%%%%%%%%%%%%%%%%%%%%%%%%%%%%%%%%%%%%%%%%%%%%%%%%%

  \subsection{Człon drugiego rzędu}
  \label{sec:TEOBF}

  Członem liniowym drugiego rzędu nazywamy układ opisany liniowym równaniem
  różniczkowym drugiego rzędu. Stanowisko laboratoryjne wyposażone jest
  w~układ, który, zachowuje się jak człon oscylacyjny bądź inercyjny II-go
  rzędu (zależnie od ustawień R, L, C). Model badanego członu II-go rzędu zdefiniowany
  jest przez dwa parametry\footnote{Model członu II-go rzędu jest w~ogólności
  określony trzema parametrami (np. stałe czasowe $T_1$, $T_2$ i~wzmocnienie
  $k$). W~niniejszym opracowaniu zakłada się jednak $k=1$, co jest cechą
  specyficzną badanego obwodu RLC.}. Do pełnej identyfikacji wystarczy więc
  oznaczenie dwu parametrów (np. pulsacji granicznej $\omega_0$ i~tłumienia
  $\zeta$).

  Człon drugiego rzędu zrealizowany jest w~postaci elektronicznego czwórnika RLC
  (rysunek~\ref{fig:FKPAJ}).
  %%%%%%%%%%%%%%%%%%%%%%%%%%%%%%%%%%%%%%%%%%%%%%%%%%%%%%%%%%%%%%%%%%%%%%%%%%%%%%
  \begin{figure}[H]
    \begin{center}
      \begin{circuitikz}[american voltages]
        \draw (0,1)
          to[short, o-, i=$i$] (1,1)
          to[L, l=$L$] (2,1)
          to[short] (3,1)
          to[european resistor, l=$R$] (4,1)
          to[short] (5,1)
          to[C, l=$C$, i=$i$] (5,-1)
          to[short, -o] (0,-1)
        ;

        \node at (0,0) {$u(t)$}
        ;

        \draw(5,-1)
          to[short,*-o] (7,-1)
        ;

        \node at (7,0) {$y(t)$}
        ;

        \draw (7,1)
          to[short,o-*] (5,1)
        ;

        \node at (11,0) {$\begin{aligned}
          & i(t) = C \od{y(t)}{t} & \\
          & L \od{i(t)}{t} + R i(t)  + y(t) = u(t) & \\
        \end{aligned}$};
      \end{circuitikz}
    \end{center}
    \caption{Czwórnik RLC jako człon oscylacyjny/inercyjny drugiego rzędu.}
    \label{fig:FKPAJ}
  \end{figure}
  %%%%%%%%%%%%%%%%%%%%%%%%%%%%%%%%%%%%%%%%%%%%%%%%%%%%%%%%%%%%%%%%%%%%%%%%%%%%%%

  Równanie różniczkowe~\eqref{eq:BYA03}
  %%%%%%%%%%%%%%%%%%%%%%%%%%%%%%%%%%%%%%%%%%%%%%%%%%%%%%%%%%%%%%%%%%%%%%%%%%%%%%
  \begin{equation}
    L C \odn{2}{y(t)}{t} + R C \od{y(t)}{t} + y(t) = u(t)
    \label{eq:BYA03}
  \end{equation}
  %%%%%%%%%%%%%%%%%%%%%%%%%%%%%%%%%%%%%%%%%%%%%%%%%%%%%%%%%%%%%%%%%%%%%%%%%%%%%%
  wyraża bilans napięć i~prądów chwilowych. Odpowiadająca mu transmitancja
  operatorowa to
  %%%%%%%%%%%%%%%%%%%%%%%%%%%%%%%%%%%%%%%%%%%%%%%%%%%%%%%%%%%%%%%%%%%%%%%%%%%%%%
  \begin{equation}
    G(s) = \frac{1}{T_2^2 s^2 + T_1 s + 1},
    \;\; T_2 = \sqrt{LC}, \;\; T_1 = RC,
    \label{eq:QWXWZ}
  \end{equation}
  %%%%%%%%%%%%%%%%%%%%%%%%%%%%%%%%%%%%%%%%%%%%%%%%%%%%%%%%%%%%%%%%%%%%%%%%%%%%%%
  gdzie $R$ jest rezystancją opornika wyrażoną w $\Omega$~(Ohm), $L$
  indukcyjnością cewki wyrażoną w~$H$~(Henr), $C$ pojemnością kondensatora
  wyrażoną w~$F$~(Farad).

  Zamiast $T_1$ i $T_2$, jako parametrów modelu członu II-go rzędu można użyć
  %%%%%%%%%%%%%%%%%%%%%%%%%%%%%%%%%%%%%%%%%%%%%%%%%%%%%%%%%%%%%%%%%%%%%%%%%%%%%%
  \begin{equation}
    \begin{aligned}
      &
      \omega_0 = \frac{1}{T_2} = \frac{1}{\sqrt{LC}},
      &&
      \zeta = \frac{T_1}{2 T_2} = \frac{R\sqrt{LC}}{2L},
      &
    \end{aligned}
  \end{equation}
  %%%%%%%%%%%%%%%%%%%%%%%%%%%%%%%%%%%%%%%%%%%%%%%%%%%%%%%%%%%%%%%%%%%%%%%%%%%%%%
  gdzie $\omega_0$ jest pulsacją graniczną, a~$\zeta$ bezwymiarowym
  współczynnikiem tłumienia. Przy tak obranych parametrach, transmitancja
  operatorowa może być wyrażona jako
  %%%%%%%%%%%%%%%%%%%%%%%%%%%%%%%%%%%%%%%%%%%%%%%%%%%%%%%%%%%%%%%%%%%%%%%%%%%%%%
  \begin{equation}
    G(s) = \frac{\omega_0^2}{s^2 + 2\zeta\omega_0 s + \omega_0^2}.
    \label{eq:7FANL}
  \end{equation}
  %%%%%%%%%%%%%%%%%%%%%%%%%%%%%%%%%%%%%%%%%%%%%%%%%%%%%%%%%%%%%%%%%%%%%%%%%%%%%%

  Przez podstawienie $s = j\omega$ do \eqref{eq:QWXWZ}
  otrzymujemy transmitancję widmową:
  %%%%%%%%%%%%%%%%%%%%%%%%%%%%%%%%%%%%%%%%%%%%%%%%%%%%%%%%%%%%%%%%%%%%%%%%%%%%%%
  \begin{equation}
    G(j\omega) = \frac{1}{T_2^2 (j \omega)^2 + T_1 j\omega + 1}
    = \underbrace{\frac{1 - (T_2\omega)^2}{\left(1-(T_2\omega)^2\right)^2 + (T_1\omega)^2}}_{P(\omega)}
    + j \underbrace{\frac{-T_1\omega}{\left(1-(T_2\omega)^2\right)^2 + (T_1\omega)^2}}_{Q(\omega)}.
    \label{eq:LS6G7}
  \end{equation}
  %%%%%%%%%%%%%%%%%%%%%%%%%%%%%%%%%%%%%%%%%%%%%%%%%%%%%%%%%%%%%%%%%%%%%%%%%%%%%%

  Moduł transmitancji, przy pulsacji granicznej $\omega = \omega_0 = 1/T_2$
  przyjmuje wartość
  %%%%%%%%%%%%%%%%%%%%%%%%%%%%%%%%%%%%%%%%%%%%%%%%%%%%%%%%%%%%%%%%%%%%%%%%%%%%%%
  \begin{equation}
    A(\omega_0) = |G(j\omega_0)|
                = \sqrt{0^2 + \left(\frac{-T_1/T_2}{(T_1/T_2)^2}\right)^2}
                = \frac{T_2}{T_1} = \frac{1}{2 \zeta}.
    \label{eq:2E47P}
  \end{equation}
  %%%%%%%%%%%%%%%%%%%%%%%%%%%%%%%%%%%%%%%%%%%%%%%%%%%%%%%%%%%%%%%%%%%%%%%%%%%%%%
  a~przesunięcie fazowe $\varphi(\omega_0) = -90\degree$. Jednocześnie
  %%%%%%%%%%%%%%%%%%%%%%%%%%%%%%%%%%%%%%%%%%%%%%%%%%%%%%%%%%%%%%%%%%%%%%%%%%%%%%
  \begin{equation}
    \begin{aligned}
      & P(\omega_0) = 0, && A(\omega_0) = \left|Q(\omega_0)\right|. &
    \end{aligned}
    \label{eq:XGWVE}
  \end{equation}
  %%%%%%%%%%%%%%%%%%%%%%%%%%%%%%%%%%%%%%%%%%%%%%%%%%%%%%%%%%%%%%%%%%%%%%%%%%%%%%

  Gdy $\zeta < 1$, transmitancja $G(s)$ ma bieguny zespolone -- układ
  z~rys.~\ref{fig:FKPAJ} jest wtedy członem oscylacyjnym i~$\omega_0$ jest
  pulsacją drgań własnych\footnote{Nie należy mylić pulsacji drgań
  własnych $\omega_0$ z~pulsacja rezonansową $\omega_r$.
  } układu. Dla~$\zeta>1$ transmitancja $G(s)$ ma dwa bieguny {\em rzeczywiste}
  i~człon przestaje być oscylacyjnym stając się członem inercyjnym II-go
  rzędu\footnote{Układ inercyjny II-go rzędu jest tożsamy z~szeregowym
  połączeniem dwóch układów inercyjnych I-go rzędu (o~stałych czasowych $T_3$
  i~$T_4$).}:
  %%%%%%%%%%%%%%%%%%%%%%%%%%%%%%%%%%%%%%%%%%%%%%%%%%%%%%%%%%%%%%%%%%%%%%%%%%%%%%
  \begin{equation}
    \begin{aligned}
      &
      G(s) = \frac{1}{(1 + T_3 s)(1 + T_4 s)}
      &&
      (\text{gdy } \zeta > 1),
      &
    \end{aligned}
  \end{equation}
  %%%%%%%%%%%%%%%%%%%%%%%%%%%%%%%%%%%%%%%%%%%%%%%%%%%%%%%%%%%%%%%%%%%%%%%%%%%%%%
  gdzie
  %%%%%%%%%%%%%%%%%%%%%%%%%%%%%%%%%%%%%%%%%%%%%%%%%%%%%%%%%%%%%%%%%%%%%%%%%%%%%%
  \begin{equation}
    \begin{aligned}
      &
      T_3 = \frac{T_1}{2} + \sqrt{\frac{T_1^2}{4} - T_2^2},
      &&
      T_4 = \frac{T_1}{2} - \sqrt{\frac{T_1^2}{4} - T_2^2}.
      &
    \end{aligned}
  \end{equation}
  %%%%%%%%%%%%%%%%%%%%%%%%%%%%%%%%%%%%%%%%%%%%%%%%%%%%%%%%%%%%%%%%%%%%%%%%%%%%%%

  Charakterystyki amplitudowo-fazowe (Nyquista) omawianego układu, przy różnych
  wartościach współczynnika tłumienia $\zeta$, przedstawiono
  na~rysunku~\ref{fig:3KG02}.
  %%%%%%%%%%%%%%%%%%%%%%%%%%%%%%%%%%%%%%%%%%%%%%%%%%%%%%%%%%%%%%%%%%%%%%%%%%%%%%
  \begin{figure}[H]
    \centering
    \input{lpas9/oscil/nyquist.tex}
    \caption{Charakterystyka amplitudowo-fazowa dla członu oscylacyjnego.}
    \label{fig:3KG02}
  \end{figure}
  %%%%%%%%%%%%%%%%%%%%%%%%%%%%%%%%%%%%%%%%%%%%%%%%%%%%%%%%%%%%%%%%%%%%%%%%%%%%%%

  Charakterystyki częstotliwościowe (Bodego) zamieszczono na rysunku~\ref{fig:O26C4}.
  %%%%%%%%%%%%%%%%%%%%%%%%%%%%%%%%%%%%%%%%%%%%%%%%%%%%%%%%%%%%%%%%%%%%%%%%%%%%%%
  \begin{figure}[H]
    \centering
    \input{lpas9/oscil/bode.tex}
    \caption{Charakterystyki Bodego dla członu II-go rzędu.}
    \label{fig:O26C4}
  \end{figure}
  %%%%%%%%%%%%%%%%%%%%%%%%%%%%%%%%%%%%%%%%%%%%%%%%%%%%%%%%%%%%%%%%%%%%%%%%%%%%%%
  W~przypadku członu II-go rzędu warto wyrysować również $A(\omega)$, ponieważ
  na wykresie modułu logarytmicznego $M(\omega)$ niektóre cechy (jak szczyt
  rezonansowy) są słabo widoczne.
  %%%%%%%%%%%%%%%%%%%%%%%%%%%%%%%%%%%%%%%%%%%%%%%%%%%%%%%%%%%%%%%%%%%%%%%%%%%%%%
  \begin{figure}[H]
    \centering
    \input{lpas9/oscil/aw.tex}
    \caption{Charakterystyka amplitudowa $A(\omega)$ dla członu II-go rzędu.}
    \label{eq:9LBEW}
  \end{figure}
  %%%%%%%%%%%%%%%%%%%%%%%%%%%%%%%%%%%%%%%%%%%%%%%%%%%%%%%%%%%%%%%%%%%%%%%%%%%%%%

  Asymptota, do której zbiega wykres $M(\omega)$ przy $\omega >> \omega_0$, ma
  dla członów II-go rzędu nachylenie $-40 \text{dB/dek}$. Można po tym
  rozpoznać, że mamy do czynienia z~członem II-go rzędu. Prosta ta przechodzi
  przez punkt $(\omega_0, 0\text{dB})$. W~tym samym punkcie, przesunięcie
  fazowe wynosi $\varphi(\omega_0) = -90\degree$. Z~charakterystyki
  amplitudowo-fazowej, jak również z~wykresu $A(\omega)$, można pośrednio
  odczytać współczynnik tłumienia, mianowicie
  $\zeta=1/(2A(\omega_0))=-1/(2Q(\omega_0))$. Dodatkowo, jeśli na wykresie
  $A(\omega)$ widoczne jest ekstremum (szczyt rezonansowy), to wiadomo, że mamy
  do czynienia z~członem oscylacyjnym rezonującym ($\zeta < \sqrt{2}/2$).
  W~przeciwnym przypadku może to być człon oscylacyjny bez rezonansu
  ($\sqrt{2}/2 \le \zeta$ < 1) lub człon inercyjny II-go rzędu ($\zeta \ge 1$).

  \subsubsection{Rezonans}
  \label{sec:8N4ZW}

  Jeśli bezwymiarowy współczynnik tłumienia $\zeta < 1$, to mamy do czynienia
  z~członem oscylacyjnym. Dodatkowo, przy {\em wystarczająco niskim tłumieniu}
  ($\zeta < \sqrt{2}/{2}$), na~wykresie modułu $A(\omega)$ można zauważyć
  ekstremum zwane {\em szczytem rezonansowym}. Jeśli występuje rezonans, to
  współrzędne $(\omega_r, A_r)$ szczytu rezonansowego wynikają z~pulsacji drgań
  własnych $\omega_0$ i~współczynnika tłumienia $\zeta$:
  %%%%%%%%%%%%%%%%%%%%%%%%%%%%%%%%%%%%%%%%%%%%%%%%%%%%%%%%%%%%%%%%%%%%%%%%%%%%%%
  \begin{equation}
    \begin{aligned}
      &
      \omega_r = \omega_0 \sqrt{1 - 2 \zeta^2},
      &&
      A_r = \frac{1}{2 \zeta \sqrt{1 - \zeta^2}}.
      &
    \end{aligned}
    \label{eq:EMIVT}
  \end{equation}
  %%%%%%%%%%%%%%%%%%%%%%%%%%%%%%%%%%%%%%%%%%%%%%%%%%%%%%%%%%%%%%%%%%%%%%%%%%%%%%
  Rezonans występuje tylko, gdy $\zeta < \frac{\sqrt{2}}{2} \approx 0.70711$.
  Dla~większych tłumień zachodzi
  %%%%%%%%%%%%%%%%%%%%%%%%%%%%%%%%%%%%%%%%%%%%%%%%%%%%%%%%%%%%%%%%%%%%%%%%%%%%%%
  \begin{equation}
    \zeta > \frac{\sqrt{2}}{2} \implies 1 - 2\zeta^2 < 0 \implies \omega_r \notin \mathbb{R},
  \end{equation}
  %%%%%%%%%%%%%%%%%%%%%%%%%%%%%%%%%%%%%%%%%%%%%%%%%%%%%%%%%%%%%%%%%%%%%%%%%%%%%%
  co oznacza, że szczyt rezonansowy nie istnieje. Człony II-go rzędu o~tłumieniu
  $\frac{\sqrt{2}}{2} \le \zeta < 1$ są więc członami oscylacyjnymi, które nie
  wpadają w~rezonans.

  Jeśli na wyznaczonej w~ramach eksperymentu charakterystyce $A(\omega)$ szczyt
  rezonansowy jest widoczny wystarczająco wyraźnie, to można próbować odczytać
  z~wykresu współrzędne $(\omega_r, A_r)$ punktu szczytowego, a~następnie
  wyliczyć współczynnik tłumienia~$\zeta(A_r)$
  %%%%%%%%%%%%%%%%%%%%%%%%%%%%%%%%%%%%%%%%%%%%%%%%%%%%%%%%%%%%%%%%%%%%%%%%%%%%%%
  \begin{equation}
    \zeta(A_r) = \sqrt{\frac{1}{2}\left(1 - \sqrt{1 - \frac{1}{A_r^2}}\right)}.
    \label{eq:SZENE}
  \end{equation}
  %%%%%%%%%%%%%%%%%%%%%%%%%%%%%%%%%%%%%%%%%%%%%%%%%%%%%%%%%%%%%%%%%%%%%%%%%%%%%%

  Należy podkreślić, że w~zastosowaniach praktycznych wyznaczanie $\zeta(A_r)$
  ma~sens jedynie w~pewnym zakresie wartości $A_r$.
  %%
  Jako miarę liczbową wrażliwości metody obliczania $\zeta(A_r)$ na błąd $A_r$
  możemy obrać wskaźnik uwarunkowania funkcji~$\zeta(A_r)$, który definiuje się
  zwyczajowo jako
  %%%%%%%%%%%%%%%%%%%%%%%%%%%%%%%%%%%%%%%%%%%%%%%%%%%%%%%%%%%%%%%%%%%%%%%%%%%%%%
  \begin{equation}
    \begin{aligned}
      &
      \text{cond}_{\zeta}(A_r)
        = \left|\frac{\zeta'}{\zeta} A_r\right|
        = \left|\frac{1-\zeta^2}{2\zeta^2-1}\right|,
      &&
      \text{bo pochodna }\zeta'(A_r)
        = \frac{1}{A_r'(\zeta)}
        = \frac{2 \zeta^2 (1-\zeta^2)^{\frac{3}{2}}}{2 \zeta^2 - 1}.
      &
    \end{aligned}
    \label{eq:HY93S}
  \end{equation}
  %%%%%%%%%%%%%%%%%%%%%%%%%%%%%%%%%%%%%%%%%%%%%%%%%%%%%%%%%%%%%%%%%%%%%%%%%%%%%%
  Duże wartości $\text{cond}_{\zeta}$ świadczą o~dużej wrażliwości na błąd.
  Przykładowa wartość $\text{cond}_{\zeta}(A_r) = 2$ oznacza, że $1\%$ błąd
  względny w~$A_r$ powoduje $2\%$ błąd względny w~wyliczonej $\zeta(A_r)$.

  Wykres funkcji $\zeta(A_r)$ i~współczynnika uwarunkowania
  $\text{cond}_{\zeta}(A_r)$ przedstawiono na~rysunku~\ref{fig:5O32M}. Można
  wykazać, że dla $A_r$ niewiele większych od jedności mamy
  %%%%%%%%%%%%%%%%%%%%%%%%%%%%%%%%%%%%%%%%%%%%%%%%%%%%%%%%%%%%%%%%%%%%%%%%%%%%%%
  \begin{equation}
    \lim_{A_r \to 1^{+}} \text{cond}_{\zeta}(A_r) = \infty.
    \label{eq:77ILN}
  \end{equation}
  %%%%%%%%%%%%%%%%%%%%%%%%%%%%%%%%%%%%%%%%%%%%%%%%%%%%%%%%%%%%%%%%%%%%%%%%%%%%%%
  Metoda nie ma więc zastosowania dla $A_r$ bliskich jedności. Dla $A_r > 1.1$
  uzyskuje się już akceptowalne wartości wskaźnika $\text{cond}_{\zeta}(A_r) < 1.7$.
  %%%%%%%%%%%%%%%%%%%%%%%%%%%%%%%%%%%%%%%%%%%%%%%%%%%%%%%%%%%%%%%%%%%%%%%%%%%%%%
  \begin{figure}[H]
    \centering
    \input{lpas9/oscil/zar.tex}
    \caption{Bezwymiarowy współczynnik tłumienia $\zeta$ w~funkcji amplitudy
             szczytowej $A_r$.}
    \label{fig:5O32M}
  \end{figure}
  %%%%%%%%%%%%%%%%%%%%%%%%%%%%%%%%%%%%%%%%%%%%%%%%%%%%%%%%%%%%%%%%%%%%%%%%%%%%%%

  Mając wyznaczony współczynnik tłumienia $\zeta$ oraz odczytaną z~wykresu
  pulsację rezonansową $\omega_r$ można wyliczyć pulsację drgań własnych
  $\omega_0$
  %%%%%%%%%%%%%%%%%%%%%%%%%%%%%%%%%%%%%%%%%%%%%%%%%%%%%%%%%%%%%%%%%%%%%%%%%%%%%%
  \begin{equation}
    \omega_0 = \frac{\omega_r}{\sqrt{1 - 2\zeta^2}}.
    \label{eq:GIWGJ}
  \end{equation}
  %%%%%%%%%%%%%%%%%%%%%%%%%%%%%%%%%%%%%%%%%%%%%%%%%%%%%%%%%%%%%%%%%%%%%%%%%%%%%%

  \subsection{Metoda pomiarowa}
  \label{sec:FTYRD}

  Użyta metoda pomiarowa jest prosta w swojej istocie. Na wejście badanego
  elementu zadaje się sygnał sinusoidalny o~określonej amplitudzie $|U|$
  i~pulsacji $\omega$ i~mierzy się w~specjalny sposób parametry sygnału
  wyjściowego $Y$ (rys.~\ref{fig:6TEOB}). Eksperyment powtarza się dla różnych
  pulsacji $\omega$. Do przeprowadzenia eksperymentu używane jest urządzenie
  (woltomierz fazoczuły) umożliwiające pomiar wartości $P_{mV}(\omega)$ i
  $Q_{mV}(\omega)$ proporcjonalnych odpowiednio do~składowej rzeczywistej
  $P(\omega)$ i~urojonej $Q(\omega)$ transmitancji $G(j \omega) = P(\omega) + j
  Q(\omega)$ badanego układu.

  Wyznaczywszy z~eksperymentu wartości $(P(\omega_i), Q(\omega_i))$ dla serii
  punktów $\omega_i$, konstruuje się charakterystykę Nyquista nanosząc punkty
  $G(j\omega_i)$ (wzór \eqref{eq:Z6A8G}) na płaszczyznę zespoloną oraz
  charakterystyki Bodego nanosząc punkty $(\omega_i, M(\omega_i))$
  (wzór~\eqref{eq:YT6UA}) i~$(\omega_i, \varphi(\omega_i))$
  (wzór~\eqref{eq:XDQ8Q}) na odpowiednie wykresy.

  \section{Plan eksperymentu}
  \label{sec:0DR40}

  Eksperyment proponuje się przeprowadzić wg diagramu czynności przedstawionego
  na rys.~\ref{fig:ZB14M}. Na diagramie umieszczono odnośniki do podrozdziałów
  opisujących poszczególne czynności.
  %%%%%%%%%%%%%%%%%%%%%%%%%%%%%%%%%%%%%%%%%%%%%%%%%%%%%%%%%%%%%%%%%%%%%%%%%%%%%%
  \begin{figure}[H]
    \centering
    \begin{tikzpicture}[node distance=1.6cm,font=\footnotesize]
      \tikzstyle{startstop} = [
        rectangle,
        rounded corners,
        minimum width=3cm,
        minimum height=1cm,
        text centered,
        draw=black,
        fill=red!30
      ]
      \tikzstyle{io} = [
        trapezium,
        trapezium left angle=70,
        trapezium right angle=110,
        minimum width=3cm,
        minimum height=1cm,
        text centered,
        draw=black,
        fill=blue!30
      ]
      \tikzstyle{process} = [
        rectangle,
        minimum width=3cm,
        minimum height=1cm,
        text centered,
        draw=black,
        fill=orange!10
      ]
      \tikzstyle{decision} = [
        diamond,
        minimum width=3cm,
        minimum height=1cm,
        text centered,
        draw=black,
        fill=green!30
      ]
      \tikzstyle{arrow} = [thick,->,>=stealth]

      \node[process] (P1) {Wstępne pomiary $(P_i, Q_i)$ [\ref{sec:HE6KE}]};
      \node[decision, below of=P1, yshift=-1cm] (D1) {Zbocze $M(\omega_i)$};
      \node[process, left of=D1, xshift=-1.0cm, yshift=-1.5cm] (P2L) {Człon I-go rzędu [\ref{sec:4C7KI}]};
      \node[process, right of=D1, xshift=1.0cm, yshift=-1.5cm] (P2R) {Człon II-go rzędu [\ref{sec:YDUPM}]};
      %\node[decision, below of=P2L, text width=2.0cm, yshift=-1cm] (D2L) {Asymptota?};
      \node[decision, below of=P2R, text width=2.0cm, yshift=-1cm] (D2R) {Szczyt rezonansowy?};
      \node[startstop, below of=D2R, yshift=-1.5cm] (S2) {Stop};
      \node[process, right of=D2R, xshift=1.0cm, yshift=-1.5cm] (P3RR) {Rezonans [\ref{sec:U6GVM}]};

      \draw[arrow] (P1) -- (D1);
      \draw[arrow] (D1) -| node[above] {$\sim \mp 20\text{dB/dek}$} (P2L);
      \draw[arrow] (D1) -| node[above] {$\sim -40\text{dB/dek}$} (P2R);
      \draw[arrow] (P2R) -- (D2R);
      \draw[arrow] (D2R) -- node[above,fill=white] {Niewidoczny} (S2);
      \draw[arrow] (D2R) -| node[above] {Widoczny} (P3RR);
    \end{tikzpicture}
    \caption{Przebieg eksperymentu -- diagram czynności}
    \label{fig:ZB14M}
  \end{figure}
  %%%%%%%%%%%%%%%%%%%%%%%%%%%%%%%%%%%%%%%%%%%%%%%%%%%%%%%%%%%%%%%%%%%%%%%%%%%%%%
  Proponuje się przeprowadzić cały eksperyment dwukrotnie -- raz dla członu I-go
  rzędu i~raz dla członu II-go rzędu. Studenci nie powinni wiedzieć jaki typ
  elementu badają ani jakie są ustawienia $R$, $L$, $C$, aż do momentu
  samodzielnego zidentyfikowania badanego układu na podstawie charakterystyk.
  Później należy udostępnić studentom ustawienia obu eksperymentów w~celu
  weryfikacji wyników.

  \subsection{Wstępne pomiary $(P_i,Q_i)$}
  \label{sec:HE6KE}

  Wstępne pomiary umożliwiają utworzenie zarysu charakterystyk
  częstotliwościowych. Jakość uzyskanego zarysu zależy od rozkładu punktów
  pomiarowych $\{\omega_i\}$. Optymalny rozkład nie jest z~góry znany. Pomiar
  wstępny proponuje się przeprowadzić w~punktach z~tabeli~\ref{tab:D3Y3O} ($28$
  punktów pomiarowych). Aby sprostać ograniczeniom czasowym laboratorium
  \textbf{można pominąć co drugą kolumnę tabeli}
  %%%%%%%%%%%%%%%%%%%%%%%%%%%%%%%%%%%%%%%%%%%%%%%%%%%%%%%%%%%%%%%%%%%%%%%%%%%%%%
  \begin{table}[H]
    \centering
    \begin{tabular}{|r|r|r|r|r|r|r|r|r|}
      \hline
      \multicolumn{9}{|c|}{Częstotliwość $f_i\text{ [Hz]}$} \\
      \hline\hline
      10 &    13 &    17 &    22 &    28 &    36 &    46 &    60 &    77 \\
      100 &   130 &   170 &   220 &   280 &   360 &   460 &   600 &   770 \\
      1000 &  1300 &  1700 &  2200 &  2800 &  3600 &  4600 &  6000 &  7700 \\
      10000 &       &       &       &       &       &       &       & \\\hline
    \end{tabular}
    \caption{Proponowany rozkład punktów pomiarowych dla pomiaru wstępnego.}
    \label{tab:D3Y3O}
  \end{table}
  %%%%%%%%%%%%%%%%%%%%%%%%%%%%%%%%%%%%%%%%%%%%%%%%%%%%%%%%%%%%%%%%%%%%%%%%%%%%%%
  Wartości z~proponowanego szeregu są równomiernie rozłożone na skali
  logarytmicznej, tzn.
  %%%%%%%%%%%%%%%%%%%%%%%%%%%%%%%%%%%%%%%%%%%%%%%%%%%%%%%%%%%%%%%%%%%%%%%%%%%%%%
  \begin{equation}
    \log{f_{i+1}} - \log{f_i} \approx \text{ const}.
    \label{eq:I4BBA}
  \end{equation}
  %%%%%%%%%%%%%%%%%%%%%%%%%%%%%%%%%%%%%%%%%%%%%%%%%%%%%%%%%%%%%%%%%%%%%%%%%%%%%%
  Charakterystyki należy wyrysowywać na bieżąco, w~trakcie wykonywania pomiarów
  $P_{mV}(f_i)$, $Q_{mV}(f_i)$ w~kolejnych punktach pomiarowych $f_i$. Umożliwi
  to wizualną ocenę przebiegu eksperymentu.

  Rysunki~\ref{fig:P2GUQ}, \ref{fig:KSU5J} i~\ref{fig:KRYUF} przedstawiają,
  przykładowe zarysy charakterystyk uzyskanych w~wyniku wstępnego pomiaru
  czwórnika RLC.
  %%%%%%%%%%%%%%%%%%%%%%%%%%%%%%%%%%%%%%%%%%%%%%%%%%%%%%%%%%%%%%%%%%%%%%%%%%%%%%
  \begin{figure}[H]
    \centering
    \input{lpas9/rlc/rough/bode.tex}
    \caption{Wstępny pomiar -- przykładowe charakterystyki Bodego}
    \label{fig:P2GUQ}
  \end{figure}
  %%%%%%%%%%%%%%%%%%%%%%%%%%%%%%%%%%%%%%%%%%%%%%%%%%%%%%%%%%%%%%%%%%%%%%%%%%%%%%

  %%%%%%%%%%%%%%%%%%%%%%%%%%%%%%%%%%%%%%%%%%%%%%%%%%%%%%%%%%%%%%%%%%%%%%%%%%%%%%
  \begin{figure}[H]
    \centering
    \input{lpas9/rlc/rough/ampfreq.tex}
    \caption{Wstępny pomiar -- charakterystyka amplitudowa}
    \label{fig:KSU5J}
  \end{figure}
  %%%%%%%%%%%%%%%%%%%%%%%%%%%%%%%%%%%%%%%%%%%%%%%%%%%%%%%%%%%%%%%%%%%%%%%%%%%%%%

  %%%%%%%%%%%%%%%%%%%%%%%%%%%%%%%%%%%%%%%%%%%%%%%%%%%%%%%%%%%%%%%%%%%%%%%%%%%%%%
  \begin{figure}[H]
    \centering
    \input{lpas9/rlc/rough/nyquist.tex}
    \caption{Wstępny pomiar -- przykładowa charakterystyka Nyquista}
    \label{fig:KRYUF}
  \end{figure}
  %%%%%%%%%%%%%%%%%%%%%%%%%%%%%%%%%%%%%%%%%%%%%%%%%%%%%%%%%%%%%%%%%%%%%%%%%%%%%%

  Z~rysunków~\ref{fig:KSU5J} i~\ref{fig:KRYUF} widać wyraźnie, że proponowany
  szereg punktów pomiarowych jest wybrakowany w~newralgicznym miejscu --
  w~okolicy częstotliwości drgań własnych $f_0$ (nieznanej na tym etapie).
  Uzupełnienie braków odbędzie się poprzez dodatkowe pomiary, wg metodyki
  opisanej w~podrozdziałach dedykowanych poszczególnym członom liniowym
  (podrozdziały \ref{sec:4C7KI} i~\ref{sec:YDUPM}).

  \subsubsection{Rozpoznanie badanego elementu}
  \label{sec:AP5MQ}

  Na podstawie zarysu charakterystyk można zwykle ustalić, jakiego typu członem
  jest badany element:
  %%%%%%%%%%%%%%%%%%%%%%%%%%%%%%%%%%%%%%%%%%%%%%%%%%%%%%%%%%%%%%%%%%%%%%%%%%%%%%
  \begin{itemize}
    \item jeżeli stroma część charakterystyki $M(f_i)$ ma nachylenie $\approx \mp
      20\text{ dB/dek}$, to mamy do czynienia z~członem I-go rzędu -- dalsze
      postępowanie wg~wytycznych z~podrozdziału~\ref{sec:4C7KI}.
    \item jeśli nachylenie jest rzędu $-40\text{ dB/dek}$, to mamy do czynienia
      z~członem II-giego rzędu -- dalsze postępowanie wg wytycznych
      z~podrozdziału~\ref{sec:YDUPM}.
  \end{itemize}
  %%%%%%%%%%%%%%%%%%%%%%%%%%%%%%%%%%%%%%%%%%%%%%%%%%%%%%%%%%%%%%%%%%%%%%%%%%%%%%
  Nachylenie charakterystyki ocenia się wizualnie. Można też kierować się
  innymi cechami szczególnymi widocznymi na charakterystykach (np. występowanie
  szczytu rezonansowego, przecinanie $\mp 45\degree$ lub $-90\degree$ na
  charakterystyce fazowej w~miejscu gdzie $M(\omega)$ zagina się, itd.).

  \subsection{Człon I-go rzędu}
  \label{sec:4C7KI}

  Identyfikacja parametryczna członu I-go rzędu polega na znalezieniu pulsacji
  granicznej $\omega_s = 2 \pi f_s$. Poszukiwanie częstotliwości granicznej
  $f_s$ proponuje się przeprowadzić iteracyjnie, poszukując poprzez pomiar
  takiej częstotliwości $f_s$, przy której\footnote{Gdziekolwiek w~tym
  podrozdziale pojawi się symbol $\mp$, należy przyjmować ,,$-$'', gdy mamy do
  czynienia z~członem inercyjnym lub ,,$+$'', gdy mamy do czynienia z~członem
  różniczkującym rzeczywistym.}
  %%%%%%%%%%%%%%%%%%%%%%%%%%%%%%%%%%%%%%%%%%%%%%%%%%%%%%%%%%%%%%%%%%%%%%%%%%%%%%
  \begin{equation}
    P_{mV}(f_s) \approx \mp Q_{mV}(f_s)
    \label{eq:7M2JN}
  \end{equation}
  %%%%%%%%%%%%%%%%%%%%%%%%%%%%%%%%%%%%%%%%%%%%%%%%%%%%%%%%%%%%%%%%%%%%%%%%%%%%%%
  Można tego dokonać metodą bisekcji.

  \subsubsection{Poszukiwanie $f_s$ metodą bisekcji}
  \label{sec:L0JNW}

  Ze~wstępnych pomiarów obrać przybliżenie początkowe
  $(f_L, P_L, Q_L) = (f_L, P_{mV}(f_L), Q_{mV}(f_L))$
  oraz $(f_R, P_R, Q_R) = (f_R, P_{mV}(f_R), Q_{mV}(f_R))$
  na podstawie punktów~$L$ i~$R$ oznaczonych rysunku~\ref{fig:UIR99}. $L$ i~$R$
  to najbliższe punkty odpowiednio po lewej i prawej stronie punktu przecięcia
  charakterystyki fazowej $\varphi(f)$ z~rzędną $\varphi(f_s) = \mp 45\degree$.
  Następnie wyszukać $f_s$ wg algorytmu (bisekcja):

  %%%%%%%%%%%%%%%%%%%%%%%%%%%%%%%%%%%%%%%%%%%%%%%%%%%%%%%%%%%%%%%%%%%%%%%%%%%%%%
  \begin{enumerate}
    \item Jeśli $f_R$ i~$f_L$ są wystarczająco blisko siebie (np. na podziałce
      generatora nie da się ustawić częstotliwości $f_M: f_L < f_M < f_R$), to
      gotowe. Przyjąć $f_s = f_L$ lub $f_s = f_R$ (które lepsze).
    \item Obliczyć $f_M^{*} = \frac{f_L + f_R}{2}$, ustawić najbliższą dostępną
      częstotliwość $f_M$ i~zmierzyć \[P_M = P_{mV}(f_M),\;\; Q_M = Q_{mV}(f_M).\]
    \item Dokonać bisekcji:
      \begin{itemize}
        \item jeśli $Q_M > \mp P_M$, podstawić $(f_L, P_L, Q_L) := (f_M, P_M, Q_M)$,
        \item jeśli $Q_M < \mp P_M$, podstawić $(f_R, P_R, Q_R) := (f_M, P_M, Q_M)$.
      \end{itemize}
      Przejść do punktu 1.
  \end{enumerate}
  %%%%%%%%%%%%%%%%%%%%%%%%%%%%%%%%%%%%%%%%%%%%%%%%%%%%%%%%%%%%%%%%%%%%%%%%%%%%%%
  Powyższy algorytm został zaimplementowany w~arkuszu obliczeniowym używanym
  w~eksperymencie.

  %%%%%%%%%%%%%%%%%%%%%%%%%%%%%%%%%%%%%%%%%%%%%%%%%%%%%%%%%%%%%%%%%%%%%%%%%%%%%%
  \begin{figure}[H]
    \centering
    \input{lpas9/rc/rough/bode.tex}
    \caption{Zarys charakterystyk Bodego (wstępny pomiar) dla czwórnika RC.}
    \label{fig:UIR99}
  \end{figure}
  %%%%%%%%%%%%%%%%%%%%%%%%%%%%%%%%%%%%%%%%%%%%%%%%%%%%%%%%%%%%%%%%%%%%%%%%%%%%%%

  \subsubsection{Dodatkowe pomiary -- wypełnienie charakterystyki}
  \label{sec:2C7WK}

  Proponuje się przeprowadzenie dodatkowych pomiarów, w~celu uzupełnienia ew. luk
  na charakterystykach uzyskanych z~pomiaru wstępnego. Przedstawiona metoda
  polega na doborze punktów pomiarowych $f_i$ równomiernie rozłożonych na
  charakterystyce amplitudowo-fazowej Nyquista, tj. takich, że:
  %%%%%%%%%%%%%%%%%%%%%%%%%%%%%%%%%%%%%%%%%%%%%%%%%%%%%%%%%%%%%%%%%%%%%%%%%%%%%%
  \begin{equation}
    \varphi(f_{i+1}) - \varphi(f_i) \approx \text{ const}.
    \label{eq:BMOQL}
  \end{equation}
  %%%%%%%%%%%%%%%%%%%%%%%%%%%%%%%%%%%%%%%%%%%%%%%%%%%%%%%%%%%%%%%%%%%%%%%%%%%%%%
  W~tym celu obieramy szereg równomiernie rozłożonych wartości $\{\varphi_i\}$,
  które zamierzamy uzyskać na drodze pomiaru. Wśród $\{\varphi_i\}$ powinna się
  znaleźć wartość $\mp 45\degree$, pamiętając, że odpowiada ona
  częstotliwości granicznej~$f_s$. Proponowany szereg wartości $\{\varphi_i\}$, to
  %%%%%%%%%%%%%%%%%%%%%%%%%%%%%%%%%%%%%%%%%%%%%%%%%%%%%%%%%%%%%%%%%%%%%%%%%%%%%%%%
  \begin{equation}
    \begin{aligned}
      & \varphi_i = \mp 45\degree - i \cdot 1.8\degree, && i=\{0, -1,+1,-2,+2,\dots,-24,+24\}. &
    \end{aligned}
    \label{eq:0ZBDF}
  \end{equation}
  %%%%%%%%%%%%%%%%%%%%%%%%%%%%%%%%%%%%%%%%%%%%%%%%%%%%%%%%%%%%%%%%%%%%%%%%%%%%%%%%
  Częstotliwości $f_i$ odpowiadające obranym wartościom $\varphi_i$ można wyliczyć
  z~zależności
  %%%%%%%%%%%%%%%%%%%%%%%%%%%%%%%%%%%%%%%%%%%%%%%%%%%%%%%%%%%%%%%%%%%%%%%%%%%%%%%%
  \begin{equation}
    \frac{f_i}{f_s} = \begin{cases}
      - \tan{\varphi_i} & \text{człon inercyjny}, \\
      (\tan{\varphi_i})^{-1} & \text{człon różniczkujący rzeczywisty}.
    \end{cases}
    \label{eq:9R1WL}
  \end{equation}
  %%%%%%%%%%%%%%%%%%%%%%%%%%%%%%%%%%%%%%%%%%%%%%%%%%%%%%%%%%%%%%%%%%%%%%%%%%%%%%%%
  Formuły \eqref{eq:9R1WL} zaimplementowane są w~arkuszu obliczeniowym używanym
  w~eksperymencie.

  \subsection{Człon II-go rzędu}
  \label{sec:YDUPM}

  Identyfikacja parametryczna członu II-go rzędu polega na określeniu pulsacji
  granicznej $\omega_0 = 2 \pi f_0$ oraz bezwymiarowego współczynnika tłumienia
  $\zeta$. Poszukiwanie częstotliwości granicznej proponuje się przeprowadzić
  iteracyjnie, poszukując poprzez pomiar takiej częstotliwości $f_0$, przy
  której
  %%%%%%%%%%%%%%%%%%%%%%%%%%%%%%%%%%%%%%%%%%%%%%%%%%%%%%%%%%%%%%%%%%%%%%%%%%%%%%
  \begin{equation}
    P_{mV}(f_s) \approx 0.
    \label{eq:8AO8B}
  \end{equation}
  %%%%%%%%%%%%%%%%%%%%%%%%%%%%%%%%%%%%%%%%%%%%%%%%%%%%%%%%%%%%%%%%%%%%%%%%%%%%%%
  Można tego dokonać metodą bisekcji.

  \subsubsection{Poszukiwanie $f_0$ metodą bisekcji}
  \label{sec:HVY4X}

  Ze~wstępnych pomiarów wynotować współrzędne $(f_L, P_L) = (f_L, P_{mV}(f_L))$
  oraz $(f_R, P_R) = (f_R, P_{mV}(f_R))$ punktów~$L$ i~$R$ oznaczonych
  rysunku~\ref{fig:WC8TU}. $L$ i~$R$ to najbliższe punkty odpowiednio po lewej
  i prawej stronie punktu przecięcia charakterystyki fazowej $\varphi(f)$
  z~rzędną $\varphi(f_0) = - 90\degree$.

  %%%%%%%%%%%%%%%%%%%%%%%%%%%%%%%%%%%%%%%%%%%%%%%%%%%%%%%%%%%%%%%%%%%%%%%%%%%%%%
  \begin{figure}[H]
    \centering
    \input{lpas9/rlc/rough/bode.tex}
    \caption{Zarys charakterystyk Bodego (wstępny pomiar) dla czwórnika RLC.}
    \label{fig:WC8TU}
  \end{figure}
  %%%%%%%%%%%%%%%%%%%%%%%%%%%%%%%%%%%%%%%%%%%%%%%%%%%%%%%%%%%%%%%%%%%%%%%%%%%%%%

  Następnie wyszukać $f_0$ wg algorytmu (bisekcja):
  %%%%%%%%%%%%%%%%%%%%%%%%%%%%%%%%%%%%%%%%%%%%%%%%%%%%%%%%%%%%%%%%%%%%%%%%%%%%%%
  \begin{enumerate}
    \item Jeśli $f_R$ i~$f_L$ są wystarczająco blisko siebie (np. na podziałce
      generatora nie da się ustawić częstotliwości $f_M: f_L < f_M < f_R$), to
      gotowe. Przyjąć $f_0 = f_L$ lub $f_0 = f_R$ (które lepsze).
    \item Obliczyć $f_M^{*} = \frac{f_L + f_R}{2}$, ustawić najbliższą dostępną
      częstotliwość $f_M$ i~zmierzyć $P_M = P_{mV}(f_M)$.
    \item Dokonać bisekcji:
      \begin{itemize}
        \item jeśli $P_M > 0$, podstawić $(f_L, P_L) := (f_M, P_M)$,
        \item jeśli $P_M < 0$, podstawić $(f_R, P_R) := (f_M, P_M)$.
      \end{itemize}
      Przejść do punktu 1.
  \end{enumerate}
  %%%%%%%%%%%%%%%%%%%%%%%%%%%%%%%%%%%%%%%%%%%%%%%%%%%%%%%%%%%%%%%%%%%%%%%%%%%%%%
  Powyższy algorytm został zaimplementowany w~arkuszu obliczeniowym używanym
  w~eksperymencie.

  \subsubsection{Określenie współczynnika tłumienia $\zeta$}
  \label{sec:L7DED}

  Współczynnik tłumienia $\zeta$ proponuje się określić na podstawie
  zależności~\eqref{eq:2E47P}, \eqref{eq:XGWVE}, jako
  %%%%%%%%%%%%%%%%%%%%%%%%%%%%%%%%%%%%%%%%%%%%%%%%%%%%%%%%%%%%%%%%%%%%%%%%%%%%%%
  \begin{equation}
    \zeta = \frac{1}{2 |Q(\omega_0)|},
    \label{eq:G616R}
  \end{equation}
  %%%%%%%%%%%%%%%%%%%%%%%%%%%%%%%%%%%%%%%%%%%%%%%%%%%%%%%%%%%%%%%%%%%%%%%%%%%%%%
  gdzie $Q(\omega_0)$ jest składową urojoną transmitancji zmierzoną przy
  częstotliwości $f_0$ (podrozdział \ref{sec:HVY4X}).

  \subsubsection{Dodatkowe pomiary -- wypełnienie charakterystyki}
  \label{sec:X9JYP}

  Proponuje się przeprowadzenie dodatkowych pomiarów, w~celu uzupełnienia ew. luk
  na charakterystykach uzyskanych z~pomiaru wstępnego. Przedstawiona metoda
  polega na doborze punktów pomiarowych $f_i$ równomiernie rozłożonych na
  charakterystyce amplitudowo-fazowej Nyquista, tj. takich, że:
  %%%%%%%%%%%%%%%%%%%%%%%%%%%%%%%%%%%%%%%%%%%%%%%%%%%%%%%%%%%%%%%%%%%%%%%%%%%%%%
  \begin{equation}
    \begin{aligned}
      & \varphi_i = -90\degree - i \cdot 3.6\degree, && i=\{0, -1,+1,-2,+2,\dots,-24,+24\}. &
    \end{aligned}
  \end{equation}
  %%%%%%%%%%%%%%%%%%%%%%%%%%%%%%%%%%%%%%%%%%%%%%%%%%%%%%%%%%%%%%%%%%%%%%%%%%%%%%
  Częstotliwości $f_i$ odpowiadające obranym wartościom $\varphi_i$ można
  wyliczyć z~zależności
  %%%%%%%%%%%%%%%%%%%%%%%%%%%%%%%%%%%%%%%%%%%%%%%%%%%%%%%%%%%%%%%%%%%%%%%%%%%%%%
  \begin{equation}
    \frac{f_i}{f_0} = \frac{
      \zeta\cos{\varphi_i} - \sqrt{\zeta^2\cos^2{\varphi_i} + \sin^2{\varphi_i}}
    }{
      \sin{\varphi_i}
    }
    \label{eq:1GXRP}
  \end{equation}
  %%%%%%%%%%%%%%%%%%%%%%%%%%%%%%%%%%%%%%%%%%%%%%%%%%%%%%%%%%%%%%%%%%%%%%%%%%%%%%
  Formuła~\eqref{eq:1GXRP} zaimplementowana jest w~arkuszu obliczeniowym
  używanym w~eksperymencie.

  \subsection{Rezonans}
  \label{sec:U6GVM}

  Jeżeli na charakterystyce amplitudowej $A(\omega)$ widoczny jest szczyt
  rezonansowy, to można dodatkowo odczytać z~doświadczalnej charakterystyki
  $A(\omega)$ współrzędne $(\omega_r, A_r)$ punktu wierzchołkowego. Do tego
  celu należy posługiwać się charakterystykami uwzględniającymi dodatkowe
  pomiary (podrozdział~\ref{sec:X9JYP}). Następnie, dla porównania, obliczyć
  wartości $\zeta$ i~$\omega_0$ wg~wzorów \eqref{eq:SZENE}, \eqref{eq:GIWGJ}.
  Niezbędne formuły są zaimplementowane w~arkuszu obliczeniowym używanym
  w~eksperymencie.

\end{appendices}

%% \bibliographystyle{unsrtnat}
%% \bibliography{lpas}

\end{document}

%\label{?:X6E21}
%\label{?:TJ2HQ}

% vim: set syntax=tex tabstop=2 shiftwidth=2 expandtab spell spelllang=pl:
