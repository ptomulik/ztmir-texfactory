\documentclass[paper=a4,DIV=12]{lpas}

\usepackage[polish]{babel}
\usepackage[T1]{fontenc}
\usepackage[utf8]{inputenc}
\usepackage{courier}
\usepackage{tgtermes,newtxtext,newtxmath}
\usepackage[]{hyperref}
\usepackage{natbib}

%%\everymath{\displaystyle}

\usepackage{bm}
\usepackage{amsmath,amsfonts}
\usepackage{mathtools}
\usepackage{graphicx}
\usepackage[titletoc,title]{appendix}
\usepackage{subcaption}
\usepackage{listings}
\usepackage{float}
\usepackage{tabularx}
\usepackage{tikz}
\usepackage{circuitikz}
\usetikzlibrary{arrows.meta,3d,patterns,calc,shapes.geometric}

\newcommand{\brm}[1]{\bm{\mathrm{#1}}}
\renewcommand{\arraystretch}{1.2}
\newcolumntype{L}[1]{>{\raggedright\arraybackslash}p{#1}}
\newcolumntype{C}[1]{>{\centering\arraybackslash}p{#1}}
\newcolumntype{R}[1]{>{\raggedleft\arraybackslash}p{#1}}

% commath provides \od, but the package is not available on my travis-ci setup
\newcommand{\od}[2]{\frac{\mathrm{d}#1}{\mathrm{d}#2}}
\newcommand{\odn}[3]{\frac{\mathrm{d}^{#1}#2}{\mathrm{d}{#3}^{#1}}}
\newcommand{\tod}[2]{\tfrac{\mathrm{d}#1}{\mathrm{d}#2}}
\newcommand{\todn}[3]{\tfrac{\mathrm{d}^{#1}#2}{\mathrm{d}{#3}^{#1}}}
% gensymb provides \degree command, but whole package for just one symbol?
\newcommand{\degree}{^{\circ}}

\lstset{%
  basicstyle=\footnotesize\ttfamily\selectfont,
  language=Matlab,
  inputencoding=utf8,
  extendedchars=true,
  frame=trBL
}

\newfloat{lstfloat}{htbp}{lop}
\floatname{lstfloat}{Listing}

\setcitestyle{numbers,square,comma}

\begin{document}


\serietitle{\\{\ }\\{\ }\\Laboratorium pomiarów, automatyki i~sterowania~I}

\title{\Large{Repetytorium}}

\subtitle{\huge{Modele układów liniowych, transmitancja operatorowa i~widmowa}}

\author{\\Paweł Tomulik\\ Zakład Teorii Maszyn i Robotów\\ ITLiMS, MEIL, PW}
\date{}
\maketitle
\thispagestyle{empty}

\pagebreak
\tableofcontents
\pagebreak

\begin{abstract}
\section{Streszczenie}
  \noindent Niniejszy dokument zawiera przypomnienie podstawowych informacji
  dotyczących rozwiązania zagadnienia początkowego z~równaniem różniczkowym
  liniowym o~stałych współczynnikach. W~ramach niniejszego ,,repetytorium''
  przywołane są pojęcia odpowiedzi ustalonej i~nieustalonej (przejściowej),
  stabilności asymptotycznej, oraz objaśnione są konsekwencje istnienia
  pierwiastków wielokrotnych równania charakterystycznego. W~rozdziale
  \ref{sec:2N6W7} pojawi się również pojęcie transmitancji operatorowej
  i~widmowej w~kontekście rozwiązania równania różniczkowego przedstawionego
  w~tym rozdziale.
\end{abstract}


\section{Rozwiązanie równania różniczkowego zwyczajnego o~stałych
         współczynnikach}
\label{sec:BVHF3}


Matematyczny model układu LUS\footnote{LUS - Liniowy Układ Stacjonarny,
określenie odnosi się do układów liniowych, których parametry nie zmieniają
się w~czasie, często też zwane układami LTI -- Linear Time-Invariant.} można
zapisać w~postaci równania różniczkowego zwyczajnego (RRZ) rzędu $N$
o~stałych współczynnikach rzeczywistych $a_k$, $b_k$
%%%%%%%%%%%%%%%%%%%%%%%%%%%%%%%%%%%%%%%%%%%%%%%%%%%%%%%%%%%%%%%%%%%%%%%%%%%%%%
\begin{equation}
  \sum_{k = 0}^{N} a_k \odn{k}{y}{t} = \sum_{k = 0}^{M} b_k \odn{k}{u}{t},
  \;\; a_k, b_k \in \mathbb{R}, \;\; a_N \neq 0, \;\; N \ge M.
  \label{eq:ZV86M}
\end{equation}
%%%%%%%%%%%%%%%%%%%%%%%%%%%%%%%%%%%%%%%%%%%%%%%%%%%%%%%%%%%%%%%%%%%%%%%%%%%%%%
Prawa strona równania \eqref{eq:ZV86M}, którą oznaczymy jako $r(t)$:
%%%%%%%%%%%%%%%%%%%%%%%%%%%%%%%%%%%%%%%%%%%%%%%%%%%%%%%%%%%%%%%%%%%%%%%%%%%%%%
\begin{equation}
  r(t) = \sum_{k = 0}^{M} b_k \odn{k}{u}{t},
  \label{eq:D11SS}
\end{equation}
%%%%%%%%%%%%%%%%%%%%%%%%%%%%%%%%%%%%%%%%%%%%%%%%%%%%%%%%%%%%%%%%%%%%%%%%%%%%%%
jest wymuszeniem wynikającym jednoznacznie z~przyłożonego sygnału wejściowego
$u(t)$. Równanie~\eqref{eq:ZV86M} możemy więc zapisać, używając
oznaczenia~$r(t)$, jako:
%%%%%%%%%%%%%%%%%%%%%%%%%%%%%%%%%%%%%%%%%%%%%%%%%%%%%%%%%%%%%%%%%%%%%%%%%%%%%%
\begin{equation}
  \sum_{k = 0}^{N} a_k \odn{k}{y}{t} = r(t).
  \label{eq:41PZG}
\end{equation}
%%%%%%%%%%%%%%%%%%%%%%%%%%%%%%%%%%%%%%%%%%%%%%%%%%%%%%%%%%%%%%%%%%%%%%%%%%%%%%
Rozwiązanie $y(t)$ równania~\eqref{eq:ZV86M}, przy zadanych warunkach
początkowych~\eqref{eq:U5Y0U}
%%%%%%%%%%%%%%%%%%%%%%%%%%%%%%%%%%%%%%%%%%%%%%%%%%%%%%%%%%%%%%%%%%%%%%%%%%%%%%
\begin{equation}
  \left.\odn{i}{y}{t}\right|_{t=0} = y^{(i)}_0, \;\; i = 0, \dots, N-1,
  \label{eq:U5Y0U}
\end{equation}
%%%%%%%%%%%%%%%%%%%%%%%%%%%%%%%%%%%%%%%%%%%%%%%%%%%%%%%%%%%%%%%%%%%%%%%%%%%%%%
konstruuje się zwyczajowo jako superpozycję dwu składowych
%%%%%%%%%%%%%%%%%%%%%%%%%%%%%%%%%%%%%%%%%%%%%%%%%%%%%%%%%%%%%%%%%%%%%%%%%%%%%%
\begin{equation}
  y(t) = y_P(t) + y_U(t).
  \label{eq:V92W6}
\end{equation}
%%%%%%%%%%%%%%%%%%%%%%%%%%%%%%%%%%%%%%%%%%%%%%%%%%%%%%%%%%%%%%%%%%%%%%%%%%%%%%
Składowe te, to rozwiązanie $y_P(t)$ równania jednorodnego \eqref{eq:BHP9M},
uzyskanego przez przyjęcie $r(t) \equiv 0$
%%%%%%%%%%%%%%%%%%%%%%%%%%%%%%%%%%%%%%%%%%%%%%%%%%%%%%%%%%%%%%%%%%%%%%%%%%%%%%
\begin{equation}
  \sum_{k = 0}^{N} a_k \odn{k}{y}{t} = 0
  \label{eq:BHP9M}
\end{equation}
%%%%%%%%%%%%%%%%%%%%%%%%%%%%%%%%%%%%%%%%%%%%%%%%%%%%%%%%%%%%%%%%%%%%%%%%%%%%%%
i~rozwiązanie $y_U(t)$ równania niejednorodnego~\eqref{eq:ZV86M}. Rozwiązanie
$y_P(t)$, zwane całką ogólną równania jednorodnego, wynika jedynie z~warunków
początkowych i~ma charakter przejściowy (dla układów stabilnych zanika
z~czasem) -- wykazano to w~rozdziale \ref{sec:ZFKPA}. Rozwiązanie $y_U(t)$,
zwane całką szczególną równania niejednorodnego, wynika z~przyłożonego
wymuszenia i~ma charakter ustalony (np. funkcja okresowa o~ustalonych
parametrach) -- wykazano to w~rozdziale~\ref{sec:JITWY}.

\subsection{Rozwiązanie równania jednorodnego -- odpowiedź przejściowa}
\label{sec:ZFKPA}

%%  Zapiszmy równanie jednorodne \eqref{eq:BHP9M} używając notacji macierzowej
%%  %%%%%%%%%%%%%%%%%%%%%%%%%%%%%%%%%%%%%%%%%%%%%%%%%%%%%%%%%%%%%%%%%%%%%%%%%%%%%%
%%  \begin{equation}
%%    \mathbf{a}^T \mathbf{y}(t) = 0,
%%    \label{eq:1BYA0}
%%  \end{equation}
%%  %%%%%%%%%%%%%%%%%%%%%%%%%%%%%%%%%%%%%%%%%%%%%%%%%%%%%%%%%%%%%%%%%%%%%%%%%%%%%%
%%  gdzie
%%  %%%%%%%%%%%%%%%%%%%%%%%%%%%%%%%%%%%%%%%%%%%%%%%%%%%%%%%%%%%%%%%%%%%%%%%%%%%%%%
%%  \begin{equation}
%%    \begin{aligned}
%%      &
%%      \mathbf{a} = \begin{bmatrix}
%%        a_0 & a_1 & \dots & a_N
%%      \end{bmatrix}^T,
%%      & &
%%      \mathbf{y}(t) = \begin{bmatrix}
%%        y(t) & \od{y}{t}(t) & \dots & \odn{N}{y}{t}(t)
%%      \end{bmatrix}^T.
%%      &
%%    \end{aligned}
%%    \label{eq:C3UCS}
%%  \end{equation}
%%  %%%%%%%%%%%%%%%%%%%%%%%%%%%%%%%%%%%%%%%%%%%%%%%%%%%%%%%%%%%%%%%%%%%%%%%%%%%%%%
%%
%%  Lewa strona równania \eqref{eq:1BYA0} jest iloczynem skalarnym wektorów
%%  $\mathbf{a}$ i~$\mathbf{y}$ należących do $N+1$-wymiarowych przestrzeni.
%%  Samo równanie należy więc interpretować jako wymóg ortogonalności wektorów
%%  $\mathbf{a}$ i~$\mathbf{y}$ dla każdej ustalonej wartości $t$. Ogólnym
%%  rozwiązaniem równania~\eqref{eq:1BYA0} będzie więc cała $N$-wymiarowa
%%  podprzestrzeń oryginalnej przestrzeni $N+1$-wymiarowej złożona ze wszystkich
%%  wektorów~$\mathbf{y}$ ortogonalnych do~$\mathbf{a}$. Taką podprzestrzeń
%%  zwykło się definiować podając $N$ liniowo niezależnych wektorów
%%  rozpinających. Przykładowo, dla przestrzeni 3-wymiarowej, podprzestrzeń
%%  wektorów ortogonalnych do wektora~$\mathbf{a}$ jest jednoznacznie określona
%%  przez dowolne dwa liniowo niezależne wektory $\mathbf{y}_1$, $\mathbf{y}_2$,
%%  z~których każdy jest ortogonalny do $\mathbf{a}$. Podprzestrzeń {\em
%%  wszystkich} wektorów ortogonalnych do $\mathbf{a}$ (rozwiązanie ogólne) jest
%%  wtedy zdefiniowana jako $\mathbf{y} = c_1 \mathbf{y}_1 + c_2 \mathbf{y}_2$,
%%  z~dowolnymi $c_1, c_2 \in \mathbb{R}$ (płaszczyzna prostopadła do $\mathbf{a}$).

%%  W~świetle powyższego wnioskujemy, że

Ogólne rozwiązanie $y_P(t)$ równania~\eqref{eq:BHP9M} jest określone przez
dowolny zbiór $\left\lbrace y_1, y_2, \dots, y_N \right\rbrace$ {\em liniowo
niezależnych} funkcji, z~których każda indywidualnie spełnia
\eqref{eq:BHP9M}. Zbiór ten jest tzw. {\em fundamentalnym zbiorem rozwiązań}
równania jednorodnego~\eqref{eq:BHP9M}. Rozwiązanie ogólne zapisuje się jako
kombinację liniową funkcji ze zbioru fundamentalnego
%%%%%%%%%%%%%%%%%%%%%%%%%%%%%%%%%%%%%%%%%%%%%%%%%%%%%%%%%%%%%%%%%%%%%%%%%%%%%%
\begin{equation}
  y_P(t) = \sum_{i = 1}^{N} c_i y_i(t),
  \;\; c_i \in \mathbb{R},
  \label{eq:M9LBE}
\end{equation}
%%%%%%%%%%%%%%%%%%%%%%%%%%%%%%%%%%%%%%%%%%%%%%%%%%%%%%%%%%%%%%%%%%%%%%%%%%%%%%

Rozwiązywanie równania jednorodnego zwyczajowo rozpoczyna się właśnie
poszukiwaniem zbioru fundamentalnego rozwiązań. Tradycyjnym narzędziem,
służącym do badania liniowej niezależności funkcji jest wyznacznik macierzy
Wrońskiego (Wrońskian). Macierz Wrońskiego, zdefiniowana jest jako macierz
$N \times N$
%%%%%%%%%%%%%%%%%%%%%%%%%%%%%%%%%%%%%%%%%%%%%%%%%%%%%%%%%%%%%%%%%%%%%%%%%%%%%%
\begin{equation}
  \mathbf{W}(t) = \begin{bmatrix}
    f_1               & f_2               & \cdots  & f_N               \\
    \od{f_1}{t}       & \od{f_2}{t}       & \cdots  & \od{f_N}{t}       \\
    \vdots            & \ddots            &         &                   \\
    \odn{N-1}{f_1}{t} & \odn{N-1}{f_2}{t} & \cdots  & \odn{N-1}{f_N}{t}
  \end{bmatrix}
  \label{eq:P191Z}
\end{equation}
%%%%%%%%%%%%%%%%%%%%%%%%%%%%%%%%%%%%%%%%%%%%%%%%%%%%%%%%%%%%%%%%%%%%%%%%%%%%%%
i~ma tę własność, że jej wyznacznik $\det{\mathbf{W}(t)} \equiv 0$ jeśli $f_1, \dots,
f_N$ są liniowo zależne. W~przeciwnym razie $\det{\mathbf{W}(t)} \neq 0$
w~jakimś punkcie $t$.

\subsubsection{Konstruowanie zbioru fundamentalnego rozwiązań}
\label{sec:UV185}

Zbiór fundamentalny rozwiązań równania~\eqref{eq:BHP9M} konstruuje się
przyjmując roboczo
%%%%%%%%%%%%%%%%%%%%%%%%%%%%%%%%%%%%%%%%%%%%%%%%%%%%%%%%%%%%%%%%%%%%%%%%%%%%%%
\begin{equation}
  y_i(t) = e^{\lambda_i t}, \; \lambda_i \in \mathbb{C}.
  \label{eq:JSZEN}
\end{equation}
%%%%%%%%%%%%%%%%%%%%%%%%%%%%%%%%%%%%%%%%%%%%%%%%%%%%%%%%%%%%%%%%%%%%%%%%%%%%%%
Aby każda z~funkcji $y_i$ była rozwiązaniem równania jednorodnego~\eqref{eq:BHP9M},
musi zachodzić
%%%%%%%%%%%%%%%%%%%%%%%%%%%%%%%%%%%%%%%%%%%%%%%%%%%%%%%%%%%%%%%%%%%%%%%%%%%%%%
\begin{equation}
  \left(\sum_{k = 0}^{N} a_k \lambda_i^k \right) e^{\lambda_i t} = 0,
  \;\; i = 1, \dots, N.
  \label{eq:RV0CU}
\end{equation}
%%%%%%%%%%%%%%%%%%%%%%%%%%%%%%%%%%%%%%%%%%%%%%%%%%%%%%%%%%%%%%%%%%%%%%%%%%%%%%
Ponieważ $e^{\lambda_i t} \neq 0$, to równanie~\eqref{eq:RV0CU} spełnione jest
wtedy i~tylko wtedy, gdy
%%%%%%%%%%%%%%%%%%%%%%%%%%%%%%%%%%%%%%%%%%%%%%%%%%%%%%%%%%%%%%%%%%%%%%%%%%%%%%
\begin{equation}
  H(\lambda_i) \coloneqq \sum_{k = 0}^{N} a_k \lambda_i^k = 0.
  \label{eq:2YZX1}
\end{equation}
%%%%%%%%%%%%%%%%%%%%%%%%%%%%%%%%%%%%%%%%%%%%%%%%%%%%%%%%%%%%%%%%%%%%%%%%%%%%%%
$H\left(\lambda\right)$ jest {\em wielomianem charakterystycznym}
równania~\eqref{eq:BHP9M} a~\eqref{eq:2YZX1} nazywamy {\em równaniem
charakterystycznym}. Pierwiastki $\lambda_i: H\left(\lambda_i\right) = 0$
wielomianu~$H$ generują rozwiązania o~postaci~$e^{\lambda_i t}$.

\subsubsection{Pierwiastki wielokrotne a~niezależność liniowa zbioru fundamentalnego}
\label{sec:XU35F}

Wymaganą cechą zbioru fundamentalnego jest liniowa niezależność jego elementów.
Narzędziem, którego można użyć do sprawdzania liniowej niezależności funkcji,
jest wyznacznik macierzy Wrońskiego. Jeśli w~zbiorze $\left\lbrace y_i
\right\rbrace$ istnieją funkcje liniowo zależne, to Wrońskian
$\det{\left(\mathbf{W}\right)}$ jest tożsamościowo równy zeru
($\det{\left(\mathbf{W}(t)\right)} \equiv 0, t \in \mathbb{R}$).

W~kontekście przyjętej postaci rozwiązania~\eqref{eq:JSZEN}, macierz
Wrońskiego będzie dana jako
%%%%%%%%%%%%%%%%%%%%%%%%%%%%%%%%%%%%%%%%%%%%%%%%%%%%%%%%%%%%%%%%%%%%%%%%%%%%%%
\begin{equation}
  \mathbf{W}(t) = \begin{bmatrix}
    e^{\lambda_1 t}                 & e^{\lambda_2 t}                 & \dots   & e^{\lambda_N t} \\
    \lambda_1 e^{\lambda_1 t}       & \lambda_2 e^{\lambda_2 t}       & \dots   & \lambda_N e^{\lambda_N t} \\
    \vdots         &                & \ddots                          & \vdots  &       \\
    \lambda_1^{N-1} e^{\lambda_1 t} & \lambda_2^{N-1} e^{\lambda_2 t} & \dots   & \lambda_N^{N-1} e^{\lambda_N t}
  \end{bmatrix} = \mathbf{V} \cdot \mathbf{E}(t),
  \label{eq:SGIWG}
\end{equation}
%%%%%%%%%%%%%%%%%%%%%%%%%%%%%%%%%%%%%%%%%%%%%%%%%%%%%%%%%%%%%%%%%%%%%%%%%%%%%%
gdzie
%%%%%%%%%%%%%%%%%%%%%%%%%%%%%%%%%%%%%%%%%%%%%%%%%%%%%%%%%%%%%%%%%%%%%%%%%%%%%%
\begin{equation}
  \begin{aligned}
    & \mathbf{V} = \begin{bmatrix}
      1               & 1               & \dots   & 1           \\
      \lambda_1       & \lambda_2       & \dots   & \lambda_N   \\
      \vdots          &                 & \ddots  & \vdots  &   \\
      \lambda_1^{N-1} & \lambda_2^{N-1} & \dots   & \lambda_N^{N-1}
    \end{bmatrix},
    &
    &
    \mathbf{E}(t) = \begin{bmatrix}
      e^{\lambda_1 t} & 0               & \dots  & 0      \\
      0               & e^{\lambda_2 t} & \dots  & 0      \\
      \vdots          &                 & \ddots & \vdots \\
      0               & \dots           & 0      & e^{\lambda_N t}
    \end{bmatrix}
    &
  \end{aligned}.
  \label{eq:NHY93}
\end{equation}
%%%%%%%%%%%%%%%%%%%%%%%%%%%%%%%%%%%%%%%%%%%%%%%%%%%%%%%%%%%%%%%%%%%%%%%%%%%%%%
Wyznacznik macierzy $\mathbf{W}$ jest więc iloczynem
%%%%%%%%%%%%%%%%%%%%%%%%%%%%%%%%%%%%%%%%%%%%%%%%%%%%%%%%%%%%%%%%%%%%%%%%%%%%%%
\begin{equation}
  \det{\left(\mathbf{W}(t)\right)} = \det{\left(\mathbf{V}\right)} \cdot \det{\left(\mathbf{E}(t)\right)}.
  \label{eq:SHE6K}
\end{equation}
%%%%%%%%%%%%%%%%%%%%%%%%%%%%%%%%%%%%%%%%%%%%%%%%%%%%%%%%%%%%%%%%%%%%%%%%%%%%%%

Ponieważ $\det{\left(\mathbf{E}(t)\right)} \neq 0, \;\; t \in \mathbb{R}$,
to istnienie liniowej zależności w~zbiorze funkcji
$\left\lbrace e^{\lambda_i t} \right\rbrace$
pociąga za sobą zerowanie się wyznacznika $\det{\left(\mathbf{V}\right)}$.

Macierz $\mathbf{V}$ ma specjalną strukturę. Jest to tzw. macierz Vandermonda,
dla której
%%%%%%%%%%%%%%%%%%%%%%%%%%%%%%%%%%%%%%%%%%%%%%%%%%%%%%%%%%%%%%%%%%%%%%%%%%%%%%
\begin{equation}
  \det{\left(\mathbf{V}\right)} = \prod_{
    \substack{k = 2 \dots N-1 \\ i = 1 \dots k-1}
  }{\left(\lambda_k - \lambda_i\right)}.
  \label{eq:KOD8D}
\end{equation}
%%%%%%%%%%%%%%%%%%%%%%%%%%%%%%%%%%%%%%%%%%%%%%%%%%%%%%%%%%%%%%%%%%%%%%%%%%%%%%
Wyznacznik $\det{\left(\mathbf{V}\right)}$ zeruje się więc wtedy i~tylko
wtedy, gdy występują pierwiastki wielokrotne wielomianu charakterystycznego
$H(\lambda)$. W~takich przypadkach konieczne jest specjalne postępowanie.

Jeśli $\alpha$ jest pierwiastkiem {\em rzeczywistym} o~krotności $k$,
tzn. $H(\lambda) = (\lambda - \alpha)^k q(\lambda)$, gdzie
$q(\alpha) \neq 0$, to $k$ liniowo niezależnych rozwiązań odpowiadających
temu pierwiastkowi obiera się jako
%%%%%%%%%%%%%%%%%%%%%%%%%%%%%%%%%%%%%%%%%%%%%%%%%%%%%%%%%%%%%%%%%%%%%%%%%%%%%%
\begin{equation}
  e^{\alpha t}, \; t e^{\alpha t},\; \dots,\; t^{k-1} e^{\alpha t}
  \label{eq:Y77IL}
\end{equation}
%%%%%%%%%%%%%%%%%%%%%%%%%%%%%%%%%%%%%%%%%%%%%%%%%%%%%%%%%%%%%%%%%%%%%%%%%%%%%%
Jeśli natomiast $\alpha \pm j \beta$ są sprzężonymi pierwiastkami {\em
zespolonymi}, każdy o~krotności $k$, tzn.
%%%%%%%%%%%%%%%%%%%%%%%%%%%%%%%%%%%%%%%%%%%%%%%%%%%%%%%%%%%%%%%%%%%%%%%%%%%%%%
\begin{equation}
  H(\lambda) = (\lambda - \lambda_1)^k (\lambda - \overline{\lambda}_1)^k p(\lambda),
  \label{eq:P0DR4}
\end{equation}
%%%%%%%%%%%%%%%%%%%%%%%%%%%%%%%%%%%%%%%%%%%%%%%%%%%%%%%%%%%%%%%%%%%%%%%%%%%%%%
gdzie $\lambda_1 = \alpha + j \beta$ i~$p(\lambda_1) \neq 0$, $p(\overline{\lambda}_1) \neq 0$,
to $2k$ liniowo niezależnych rozwiązań odpowiadających tym pierwiastkom
obiera się jako
%%%%%%%%%%%%%%%%%%%%%%%%%%%%%%%%%%%%%%%%%%%%%%%%%%%%%%%%%%%%%%%%%%%%%%%%%%%%%%
\begin{equation}
  \begin{aligned}
    & e^{\alpha t} \cos{\beta t}, \;
    & t e^{\alpha t} \cos{\beta t}, \;
    & \dots, \;
    & t^{k-1} e^{\alpha t} \cos{\beta t}, \;
    \\
    & & \text{oraz} & &
    \\
    & e^{\alpha t} \sin{\beta t}, \;
    & t e^{\alpha t} \sin{\beta t}, \;
    & \dots, \;
    & t^{k-1} e^{\alpha t} \sin{\beta t}. \;
  \end{aligned}
  \label{eq:0TFLL}
\end{equation}
%%%%%%%%%%%%%%%%%%%%%%%%%%%%%%%%%%%%%%%%%%%%%%%%%%%%%%%%%%%%%%%%%%%%%%%%%%%%%%

\subsubsection{Przejściowy charakter rozwiązania równania jednorodnego}
\label{sec:V185P}

Jeśli przyjmiemy $\lambda_i = \sigma_i + j \omega_i$, gdzie $\sigma_i,
\omega_i \in \mathbb{R}$ i~$j = \sqrt{-1}$, to, w~myśl
równania~\eqref{eq:M9LBE}, rozwiązanie $y_P$ można zapisać
jako\footnote{Zakładamy, że wielomian charakterystyczny ma tylko pierwiastki
jednokrotne.}
%%%%%%%%%%%%%%%%%%%%%%%%%%%%%%%%%%%%%%%%%%%%%%%%%%%%%%%%%%%%%%%%%%%%%%%%%%%%%%
\begin{equation}
  y_P(t) = \sum_{i = 1}^{N} c_i e^{\sigma_i t} e^{j \omega_i t}.
  \label{eq:GS278}
\end{equation}
%%%%%%%%%%%%%%%%%%%%%%%%%%%%%%%%%%%%%%%%%%%%%%%%%%%%%%%%%%%%%%%%%%%%%%%%%%%%%%
Założywszy, że spełnione są warunki stabilności asymptotycznej układu:
$\mathrm{Re}\left({\lambda_i}\right) = \sigma_i < 0, \; i = 1 \dots N$,
zauważymy, że rozwiązanie $y_P(t)$ z~czasem zanika, tzn.
%%%%%%%%%%%%%%%%%%%%%%%%%%%%%%%%%%%%%%%%%%%%%%%%%%%%%%%%%%%%%%%%%%%%%%%%%%%%%%
\begin{equation}
  \forall_{i = 1 \dots N}{\mathrm{Re}\left(\lambda_i\right) < 0}
  \implies
  \lim_{t \to \infty}{y_P(t)} = 0,
  \label{eq:AZ6A8}
\end{equation}
%%%%%%%%%%%%%%%%%%%%%%%%%%%%%%%%%%%%%%%%%%%%%%%%%%%%%%%%%%%%%%%%%%%%%%%%%%%%%%
ponieważ wszystkie czynniki $e^{\sigma_i t}$ dążą do zera. Rozwiązanie $y_P$
jest zatem składową przejściową odpowiedzi układu (na zadane warunki
początkowe).

\subsection{Rozwiązanie równania niejednorodnego -- odpowiedź ustalona}
\label{sec:JITWY}

Rozwiązania szczególnego $y_U$ równania niejednorodnego można poszukiwać
metodą przewidywań pośród funkcji odpowiadających zadanemu wymuszeniu $r(t)$.
Jeśli przyjmiemy sygnał wejściowy o~przebiegu sinusoidalnym
%%%%%%%%%%%%%%%%%%%%%%%%%%%%%%%%%%%%%%%%%%%%%%%%%%%%%%%%%%%%%%%%%%%%%%%%%%%%%%
\begin{equation}
  u(t) = \left(\cos{\omega t} + j \sin{\omega t} \right)
                  = e^{j \omega t},
  \label{eq:HP9M8}
\end{equation}
%%%%%%%%%%%%%%%%%%%%%%%%%%%%%%%%%%%%%%%%%%%%%%%%%%%%%%%%%%%%%%%%%%%%%%%%%%%%%%
to, zgodnie z~\eqref{eq:D11SS}, wymuszenie $r(t)$ jest następujące
%%%%%%%%%%%%%%%%%%%%%%%%%%%%%%%%%%%%%%%%%%%%%%%%%%%%%%%%%%%%%%%%%%%%%%%%%%%%%%
\begin{equation}
  r(t)
  = \left( \sum_{k=0}^{N}{b_k \left(j\omega\right)^k} \right) e^{j \omega t}.
  \label{eq:BXDQ8}
\end{equation}
%%%%%%%%%%%%%%%%%%%%%%%%%%%%%%%%%%%%%%%%%%%%%%%%%%%%%%%%%%%%%%%%%%%%%%%%%%%%%%
Dla takiego wymuszenia przewidujemy rozwiązanie $y_U$ równania
niejednorodnego w~postaci
%%%%%%%%%%%%%%%%%%%%%%%%%%%%%%%%%%%%%%%%%%%%%%%%%%%%%%%%%%%%%%%%%%%%%%%%%%%%%%
\begin{equation}
  y_U(t) = \tilde{G} e^{j \omega t},
  \;\; \tilde{G} \in \mathbb{C}.
  \label{eq:QYT6U}
\end{equation}
%%%%%%%%%%%%%%%%%%%%%%%%%%%%%%%%%%%%%%%%%%%%%%%%%%%%%%%%%%%%%%%%%%%%%%%%%%%%%%
Zauważmy od razu, że przyjmując taką postać rozwiązania, mamy jednocześnie
%%%%%%%%%%%%%%%%%%%%%%%%%%%%%%%%%%%%%%%%%%%%%%%%%%%%%%%%%%%%%%%%%%%%%%%%%%%%%%
\begin{equation}
  y_U(t) = \tilde{G} \, u(t),
  \;\; \tilde{G} \in \mathbb{C}.
  \label{eq:W4GQ5}
\end{equation}
%%%%%%%%%%%%%%%%%%%%%%%%%%%%%%%%%%%%%%%%%%%%%%%%%%%%%%%%%%%%%%%%%%%%%%%%%%%%%%
Symbol $\tilde{G}$ jest liczbą zespoloną wyrażającą amplitudę i~przesunięcie
fazowe rozwiązania $y_U$ względem sygnału wejściowego $u$. Wstawiwszy
\eqref{eq:QYT6U} i~\eqref{eq:BXDQ8} do równania
niejednorodnego~\eqref{eq:41PZG} otrzymamy
%%%%%%%%%%%%%%%%%%%%%%%%%%%%%%%%%%%%%%%%%%%%%%%%%%%%%%%%%%%%%%%%%%%%%%%%%%%%%%
\begin{equation}
  \left(\sum_{k = 0}^{N} a_k \left({j \omega}\right)^{k}\right) \tilde{G} e^{j \omega t}
  =
  \left(\sum_{k = 0}^{M} b_k \left({j \omega}\right)^{k}\right) e^{j \omega t},
  \label{eq:GZ5EX}
\end{equation}
%%%%%%%%%%%%%%%%%%%%%%%%%%%%%%%%%%%%%%%%%%%%%%%%%%%%%%%%%%%%%%%%%%%%%%%%%%%%%%
co po prostych przekształceniach daje
%%%%%%%%%%%%%%%%%%%%%%%%%%%%%%%%%%%%%%%%%%%%%%%%%%%%%%%%%%%%%%%%%%%%%%%%%%%%%%
\begin{equation}
  \tilde{G}\left(\omega\right) = \frac{
    \sum_{k = 0}^{M} b_k \left({j \omega}\right)^{k}
  } {
    \sum_{k = 0}^{N} a_k \left({j \omega}\right)^{k}
  }.
  \label{eq:T8N4Z}
\end{equation}
%%%%%%%%%%%%%%%%%%%%%%%%%%%%%%%%%%%%%%%%%%%%%%%%%%%%%%%%%%%%%%%%%%%%%%%%%%%%%%

Funkcja $\tilde{G}\left(\omega\right)$ przyjmuje wartości zespolone, które
można wyrazić jako
%%%%%%%%%%%%%%%%%%%%%%%%%%%%%%%%%%%%%%%%%%%%%%%%%%%%%%%%%%%%%%%%%%%%%%%%%%%%%%
\begin{equation}
  \tilde{G}\left(\omega\right) = P\left(\omega\right) + j Q\left(\omega\right),
  \;\; P\left(\omega\right), Q\left(\omega\right) \in \mathbb{R}
  \label{eq:9B5N7}
\end{equation}
%%%%%%%%%%%%%%%%%%%%%%%%%%%%%%%%%%%%%%%%%%%%%%%%%%%%%%%%%%%%%%%%%%%%%%%%%%%%%%
lub
%%%%%%%%%%%%%%%%%%%%%%%%%%%%%%%%%%%%%%%%%%%%%%%%%%%%%%%%%%%%%%%%%%%%%%%%%%%%%%
\begin{equation}
  \tilde{G}\left(\omega\right) = A\left(\omega\right) e^{j \varphi\left(\omega\right)},
  \;\; A\left(\omega\right) \in \mathbb{R}, \varphi\left(\omega\right) \in \mathbb{R}
  \label{eq:3LDB7}
\end{equation}
%%%%%%%%%%%%%%%%%%%%%%%%%%%%%%%%%%%%%%%%%%%%%%%%%%%%%%%%%%%%%%%%%%%%%%%%%%%%%%
i~wtedy
%%%%%%%%%%%%%%%%%%%%%%%%%%%%%%%%%%%%%%%%%%%%%%%%%%%%%%%%%%%%%%%%%%%%%%%%%%%%%%
\begin{equation}
  y_U(t)
  = P\left(\omega\right) \tilde{U} e^{j \omega t}
  + j Q\left(\omega\right) \tilde{U} e^{j \omega t}
  = A\left(\omega\right) \cdot \tilde{U} e^{j \omega t + \varphi\left(\omega\right)}.
  \label{eq:4IJQ6}
\end{equation}
%%%%%%%%%%%%%%%%%%%%%%%%%%%%%%%%%%%%%%%%%%%%%%%%%%%%%%%%%%%%%%%%%%%%%%%%%%%%%%
Jak widać, odpowiedź $y_U$ jest falą sinusoidalną o~tej samej pulsacji
$\omega$ co sygnał wejściowy $u(t) = \tilde{U} e^{j \omega t}$. Sygnały
różnią się jedynie amplitudą i~są względem siebie przesunięte w~fazie.

\subsection{Wyznaczenie stałych w~rozwiązaniu równania jednorodnego}
\label{sec:JHXCT}

Na tym etapie możemy wyznaczyć współczynniki $c_i$ pojawiające się
w~składowej przejściowej $y_P$ (wzór \eqref{eq:M9LBE}). Wartości
współczynników wynikają z~warunków początkowych~\eqref{eq:U5Y0U}, które to
warunki, po uwzględnieniu postaci rozwiązania~\eqref{eq:V92W6}, przyjmują
postać
%%%%%%%%%%%%%%%%%%%%%%%%%%%%%%%%%%%%%%%%%%%%%%%%%%%%%%%%%%%%%%%%%%%%%%%%%%%%%%
\begin{equation}
  \sum_{k = 1}^{N} \lambda_k^{i} c_k  = y_0^{(i)} - \tilde{G}\left(\omega\right) u_0^{(i)},
  \;\; i = 0, \dots, N-1,
  \label{eq:FVRCD}
\end{equation}
%%%%%%%%%%%%%%%%%%%%%%%%%%%%%%%%%%%%%%%%%%%%%%%%%%%%%%%%%%%%%%%%%%%%%%%%%%%%%%
gdzie $u_0^{(i)} = (j \omega)^i \tilde{U}$ jest wartością początkową
$i$-tej pochodnej sygnału wejściowego $u(t)$.

Układ $N$ równań liniowych~\eqref{eq:FVRCD} można zapisać w~postaci
macierzowej $\mathbf{V} \mathbf{c} = \mathbf{d}$
%%%%%%%%%%%%%%%%%%%%%%%%%%%%%%%%%%%%%%%%%%%%%%%%%%%%%%%%%%%%%%%%%%%%%%%%%%%%%%
\begin{equation}
  \underbrace{\begin{bmatrix}
    1               & 1               & 1               & \dots & 1               \\
    \lambda_1       & \lambda_2       & \lambda_3       & \dots & \lambda_N       \\
    \lambda_1^2     & \lambda_2^2     & \lambda_3^2     & \dots & \lambda_N^2     \\
    \vdots          &                 &                 & \ddots& \vdots          \\
    \lambda_1^{N-1} & \lambda_2^{N-1} & \lambda_3^{N-1} & \dots & \lambda_N^{N-1} \\
  \end{bmatrix}}_{\mathbf{V}}
  \underbrace{\begin{bmatrix}
    c_1 \\ c_2 \\ c_3 \\ \vdots \\ c_N
  \end{bmatrix}}_{\mathbf{c}}
  =
  \underbrace{\begin{bmatrix}
    d_1 \\ d_2 \\ d_3 \\ \vdots \\ d_N
  \end{bmatrix}}_{\mathbf{d}},
\end{equation}
%%%%%%%%%%%%%%%%%%%%%%%%%%%%%%%%%%%%%%%%%%%%%%%%%%%%%%%%%%%%%%%%%%%%%%%%%%%%%%
gdzie $d_i = y_0^{(i)} - \tilde{G}\left(\omega\right) u_0^{(i)}$. W~myśl
wcześniejszych rozważań, jeśli nie występują wielokrotne pierwiastki równania
charakterystycznego, to macierz $\mathbf{V}$ jest odwracalna. Oznacza to, że
współczynniki $c_i$ można wyznaczyć jednoznacznie. Odpowiedź przejściowa
$y_P$ jest więc jednoznacznie określona przez warunki początkowe (w~tym
wartości początkowe wymuszenia $u$ i~jego pochodnych) i~nie zależy od
sygnału wejściowego $u(t)$ dla $t > 0$.

\section{Transmitancja operatorowa i~widmowa}
\label{sec:2N6W7}

Transformata Laplace'a jest narzędziem matematycznym, które z~założenia ma
ułatwić rozwiązywanie równań różniczkowych. W~przypadku RRZ o~stałych
współczynnikach, transformata Laplace'a przekształca równania różniczkowe do
postaci równań algebraicznych.

W~wyniku zastosowania transformaty Laplace'a do obu stron równania
różniczkowego \eqref{eq:ZV86M} otrzymujemy następujące równanie algebraiczne
%%%%%%%%%%%%%%%%%%%%%%%%%%%%%%%%%%%%%%%%%%%%%%%%%%%%%%%%%%%%%%%%%%%%%%%%%%%%%%
\begin{equation}
  \sum_{k = 0}^{N} a_k \left( s^k Y(s) - \sum_{i = 0}^{k-1} s^{k-1-i} \left.\odn{i}{y}{t}\right|_{t=0}\right)
  =
  \sum_{k = 0}^{M} b_k \left( s^k U(s) - \sum_{i = 0}^{k-1} s^{k-1-i} \left.\odn{i}{u}{t}\right|_{t=0}\right),
  \label{eq:8P191}
\end{equation}
%%%%%%%%%%%%%%%%%%%%%%%%%%%%%%%%%%%%%%%%%%%%%%%%%%%%%%%%%%%%%%%%%%%%%%%%%%%%%%
a~po przekształceniach, rozwiązanie równania w~dziedzinie Laplace'a
%%%%%%%%%%%%%%%%%%%%%%%%%%%%%%%%%%%%%%%%%%%%%%%%%%%%%%%%%%%%%%%%%%%%%%%%%%%%%%
\begin{equation}
  Y\left(s\right) = \underbrace{\frac{
    \sum_{k = 0}^{M}{b_k s^k}
  }{
    \sum_{k = 0}^{N}{a_k s^k}
  } \cdot U\left(s\right)}_{Y_U(s)}
  +
  \underbrace{\frac{
    \sum_{k = 0}^{N}{ a_k \sum_{i = 0}^{k-1} s^{k-1-i} \left.\odn{i}{y}{t}\right|_{t=0} }
    -
    \sum_{k = 0}^{M}{ b_k \sum_{i = 0}^{k-1} s^{k-1-i} \left.\odn{i}{u}{t}\right|_{t=0} }
  }{
    \sum_{k = 0}^{N}{a_k s^k}
  }}_{Y_P(s)}.
  \label{eq:4QWXW}
\end{equation}
%%%%%%%%%%%%%%%%%%%%%%%%%%%%%%%%%%%%%%%%%%%%%%%%%%%%%%%%%%%%%%%%%%%%%%%%%%%%%%
Jeśli zaniedbamy odpowiedź przejściową $Y_P\left(s\right)$, która zależy
tylko od warunków początkowych i~dąży z~czasem do zera (stabilność
asymptotyczna), możemy zapisać
%%%%%%%%%%%%%%%%%%%%%%%%%%%%%%%%%%%%%%%%%%%%%%%%%%%%%%%%%%%%%%%%%%%%%%%%%%%%%%
\begin{equation}
  Y\left(s\right) = \underbrace{\frac{
    \sum_{k = 0}^{M}{b_k s^k}
  }{
    \sum_{k = 0}^{N}{a_k s^k}
  }}_{G(s)} \cdot U\left(s\right).
  \label{eq:2O26C}
\end{equation}
%%%%%%%%%%%%%%%%%%%%%%%%%%%%%%%%%%%%%%%%%%%%%%%%%%%%%%%%%%%%%%%%%%%%%%%%%%%%%%
Należy zauważyć, że odpowiedź przejściowa $Y_P(s)$ nie występuje w~ogóle
jeśli wszystkie warunki początkowe są zerowe (zarówno
$\left.\odn{i}{y}{t}\right|_{t=0},\;i = 0, \dots, N$ jak
i~$\left.\odn{i}{u}{t}\right|_{t=0},\;i = 0, \dots, M$).
Możemy zatem przyjąć, że transmitancja operatorowa $G(s)$ ma bezpośrednie
zastosowanie do opisu układów w~procesach z~zerowymi warunkami początkowymi.

Transmitancja operatorowa $G\left(s\right)$ umożliwia wyznaczenie odpowiedzi
ustalonej $Y_U(s)$ układu na zadane wymuszenie $U(s)$. Dla układu LUS
transmitancja jest funkcją wymierną (iloraz wielomianów)
%%%%%%%%%%%%%%%%%%%%%%%%%%%%%%%%%%%%%%%%%%%%%%%%%%%%%%%%%%%%%%%%%%%%%%%%%%%%%%
\begin{equation}
  G\left(s\right) = \frac{
    \sum_{k = 0}^{M}{b_k s^k}
  }{
    \sum_{k = 0}^{N}{a_k s^k}
  }
  \label{eq:X3KG0}
\end{equation}
%%%%%%%%%%%%%%%%%%%%%%%%%%%%%%%%%%%%%%%%%%%%%%%%%%%%%%%%%%%%%%%%%%%%%%%%%%%%%%
Zauważymy natychmiast, że $G(j\omega)$ jest tożsame z~$\tilde{G}(\omega)$ --
funkcją, która pojawiła się już we~wzorze~\eqref{eq:T8N4Z}.

%% \bibliographystyle{unsrtnat}
%% \bibliography{lpas}

\end{document}

%\label{?:X6E21}
%\label{?:TJ2HQ}
%\label{?:0NZQS}
%\label{?:A20W9}
%\label{?:7M2JN}
%\label{?:KSU5J}
%\label{?:KRYUF}
%\label{?:P2GUQ}

% vim: set syntax=tex tabstop=2 shiftwidth=2 expandtab spell spelllang=pl:
