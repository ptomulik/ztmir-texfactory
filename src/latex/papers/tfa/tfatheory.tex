\documentclass[paper=a4,DIV=12]{tfa}

\usepackage[polish]{babel}
\usepackage[T1]{fontenc}
\usepackage[utf8]{inputenc}
\usepackage{courier}
\usepackage{tgtermes,newtxtext,newtxmath}
\usepackage[pdftex,colorlinks,allcolors=blue]{hyperref}
\usepackage{natbib}

%%\everymath{\displaystyle}

\usepackage{bm}
\usepackage{amsmath,amsfonts}
\usepackage{mathtools, nccmath}
\usepackage[pdftex]{graphicx}
\usepackage[titletoc,title]{appendix}
\usepackage{subcaption}
\usepackage{listings}
\usepackage{float}
\usepackage{tabularx}
\usepackage{tikz}
\usetikzlibrary{arrows,3d,patterns,calc}

%%\newcommand{\brm}[1]{\bm{\mathrm{#1}}}
%%\renewcommand{\arraystretch}{1.2}
%%\newcolumntype{L}[1]{>{\raggedright\arraybackslash}p{#1}}
%%\newcolumntype{C}[1]{>{\centering\arraybackslash}p{#1}}
%%\newcolumntype{R}[1]{>{\raggedleft\arraybackslash}p{#1}}
%%
%%% commath provides \od, but the package is not available on my travis-ci setup
%%\newcommand{\od}[2]{\frac{\mathrm{d}#1}{\mathrm{d}#2}}
%%\newcommand{\odn}[3]{\frac{\mathrm{d}^{#1}#2}{\mathrm{d}{#3}^{#1}}}
%%\newcommand{\tod}[2]{\tfrac{\mathrm{d}#1}{\mathrm{d}#2}}
%%\newcommand{\todn}[3]{\tfrac{\mathrm{d}^{#1}#2}{\mathrm{d}{#3}^{#1}}}
%%% gensymb provides \degree command, but whole package for just one symbol?
%%\newcommand{\degree}{^{\circ}}
%%
%%\lstset{%
%%  basicstyle=\footnotesize\ttfamily\selectfont,
%%  language=Matlab,
%%  inputencoding=utf8,
%%  extendedchars=true,
%%  frame=trBL
%%}

%%\newfloat{lstfloat}{htbp}{lop}
%%\floatname{lstfloat}{Listing}

\DeclarePairedDelimiterX{\set}[1]{\{}{\}}{\setargs{#1}}
\NewDocumentCommand{\setargs}{>{\SplitArgument{1}{;}}m}
{\setargsaux#1}
\NewDocumentCommand{\setargsaux}{mm}
{\IfNoValueTF{#2}{#1} {#1\,\delimsize|\,\mathopen{}#2}}%{#1\:;\:#2}

\setcitestyle{numbers,square,comma}

\begin{document}

\serietitle{Elektroniczny Analizator Transmitancji}
\title{Przygotowanie teoretyczne}
%\subtitle{}
\author{Paweł Tomulik}
\date{}
\maketitle

\begin{abstract}
  Niniejszy dokument przedstawia rozważania teoretyczne, mające pomóc autorowi
  w~zaprojektowaniu niskobudżetowego analizatora transmitancji. Urządzenie to
  ma umożliwić pomiar transmitancji liniowych układów elektronicznych, głównie
  filtrów RLC. Jako że autor nie posiada właściwej edukacji w~dziedzinie
  elektroniki, rozważania poniższe stanowią swojego rodzaju notatki sporządzone
  w~trakcie samo-edukacji.
\end{abstract}

\section{Wprowadzenie}
\label{sec:G1J9K}

\section{Układ pomiarowy}
\label{sec:NMQFB}

\section{Koncepcja pomiaru}
\label{sec:6C4QN}

Koncepcja pomiaru jest w zasadzie znana od prawie 100 lat. Urządzenia z
kategorii ,,transfer function analyzer'', czy wzmacniacze typu ,,lock-in
amplifier'' działają wg podobnej zasady. U podstaw działania tych urządzeń leży
proces mieszania sygnału badanego $u(t)$ z~pewnym sygnałem wzorcowym. W~tym
celu stosuje się nieliniowe elementy elektroniczne, bądź układy scalone
przeznaczone do przemiany częstotliwości. Jedną z~metod mieszania
częstotliwości jest mnożenie sygnałów. Szczególnie interesujące dla potrzeb
niniejszego projektu jest przetwarzanie sygnałów sinusoidalnych.

\subsection{Iloczyn funkcji sinusoidalnych}
\label{sec:NZPSA}

Pamiętamy, że funkcje trygonometryczne można wyrazić w~postaci wykładniczej
%%%%%%%%%%%%%%%%%%%%%%%%%%%%%%%%%%%%%%%%%%%%%%%%%%%%%%%%%%%%%%%%%%%%%%%%%%%%%%
\begin{align}
  & \cos x = \frac{1}{2} \left( e^{jx} + e^{-jx}\right), &
  & \sin x = \frac{1}{2j} \left( e^{jx} - e^{-jx}\right). &
  \label{eq:7R2OT}
\end{align}
%%%%%%%%%%%%%%%%%%%%%%%%%%%%%%%%%%%%%%%%%%%%%%%%%%%%%%%%%%%%%%%%%%%%%%%%%%%%%%

Funkcję sinusoidalną $\sin \left(\Omega t + \varphi\right)$ możemy wymnożyć
przez funkcje wzorcowe $u_I(t) = \sin \omega t$ oraz $u_Q(t) = \cos \omega t$
otrzymując ciekawy rezultat
%%%%%%%%%%%%%%%%%%%%%%%%%%%%%%%%%%%%%%%%%%%%%%%%%%%%%%%%%%%%%%%%%%%%%%%%%%%%%%
\begin{subequations}
  \label{eq:9LJZO}
  \begin{multline}
      \sin \left(\Omega t + \varphi\right) \sin \omega t
      =
      \frac{1}{2j} \left(
        e^{j(\Omega t + \varphi)} - e^{-j(\Omega t + \varphi)}
      \right)
      \frac{1}{2j} \left(
        e^{j \omega t} - e^{-j \omega t}
      \right)
      \\
      =
      - \frac{1}{2} \cdot \frac{1}{2} \left(
          e^{j((\Omega + \omega) t + \varphi)}
        - e^{j((\Omega - \omega) t + \varphi)}
        - e^{-j((\Omega - \omega) t + \varphi)}
        + e^{-j((\Omega + \omega) t + \varphi)}
      \right)
      \\
      =
      \frac{1}{2} \cdot \frac{1}{2} \left(
          e^{j((\Omega - \omega) t + \varphi)}
        + e^{-j((\Omega - \omega) t + \varphi)}
      \right)
      - \frac{1}{2} \cdot \frac{1}{2} \left(
          e^{j((\Omega + \omega) t + \varphi)}
        + e^{-j((\Omega + \omega) t + \varphi)}
      \right)
      \\
      = \frac{1}{2} \cos{((\Omega - \omega) t + \varphi)}
      - \frac{1}{2} \cos{((\Omega + \omega) t + \varphi)},
    \label{eq:VZX5G}
  \end{multline}
  \begin{multline}
      \sin \left(\Omega t + \varphi\right) \cos \omega t
      =
      \frac{1}{2j} \left(
        e^{j(\Omega t + \varphi)} - e^{-j(\Omega t + \varphi)}
      \right)
      \frac{1}{2} \left(
        e^{j \omega t} + e^{-j \omega t}
      \right)
      \\
      =
        \frac{1}{2} \cdot \frac{1}{2j} \left(
          e^{j((\Omega + \omega) t + \varphi)}
        + e^{j((\Omega - \omega) t + \varphi)}
        - e^{-j((\Omega - \omega) t + \varphi)}
        - e^{-j((\Omega + \omega) t + \varphi)}
      \right)
      \\
      =
      \frac{1}{2} \cdot \frac{1}{2j} \left(
          e^{j((\Omega - \omega) t + \varphi)}
        - e^{-j((\Omega - \omega) t + \varphi)}
      \right)
      + \frac{1}{2} \cdot \frac{1}{2j} \left(
          e^{j((\Omega + \omega) t + \varphi)}
        - e^{-j((\Omega + \omega) t + \varphi)}
      \right)
      \\
      = \frac{1}{2} \sin{((\Omega - \omega) t + \varphi)}
      + \frac{1}{2} \sin{((\Omega + \omega) t + \varphi)}.
    \label{eq:HWX3K}
  \end{multline}
\end{subequations}
%%%%%%%%%%%%%%%%%%%%%%%%%%%%%%%%%%%%%%%%%%%%%%%%%%%%%%%%%%%%%%%%%%%%%%%%%%%%%%

Z~równań \eqref{eq:9LJZO} widzimy, że w~wyniku mnożenia sygnałów
o~pulsacjach $\Omega$ i $\omega$ powstaje sygnał złożony z~dwu harmonicznych:
jednej o~pulsacji $\Omega - \omega$, drugiej o pulsacji $\Omega + \omega$.

W~szczególnym przypadku, gdy $\Omega = \omega$
%%%%%%%%%%%%%%%%%%%%%%%%%%%%%%%%%%%%%%%%%%%%%%%%%%%%%%%%%%%%%%%%%%%%%%%%%%%%%%
\begin{subequations}
  \label{eq:EE55O}
  \begin{align}
    &
    \sin \left(\omega t + \varphi\right) \sin\omega t
    =
    \frac{1}{2} \cos{\varphi} - \frac{1}{2} \cos{(2 \omega t + \varphi)},
    &
    \label{eq:U3M0L}
    \\
    &
    \sin \left(\omega t + \varphi\right) \cos \omega t
    =
    \frac{1}{2} \sin{\varphi} + \frac{1}{2} \sin{(2 \omega t + \varphi)},
    &
    \label{eq:2ENFD}
  \end{align}
\end{subequations}
%%%%%%%%%%%%%%%%%%%%%%%%%%%%%%%%%%%%%%%%%%%%%%%%%%%%%%%%%%%%%%%%%%%%%%%%%%%%%%
otrzymujemy sygnały, których składowe stałe $\frac{1}{2}\cos\varphi$,
$\frac{1}{2}\sin\varphi$ odzwierciedlają przesunięcie fazowe $\varphi$.

\subsection{Sygnały użyte w~pomiarze transmitancji $G_D$}
\label{sec:I1HGZ}

W~niniejszym projekcie zakłada się, że do układu badanego doprowadzony jest
zmienny sygnał napięciowy $u_1(t)$ o~amplitudzie $U_1$, pulsacji $\omega$
i~przesunięty w~fazie względem sygnału wzorcowego $u_I(t) = \sin{\omega t}$
o~kąt $\varphi_1$.
%%%%%%%%%%%%%%%%%%%%%%%%%%%%%%%%%%%%%%%%%%%%%%%%%%%%%%%%%%%%%%%%%%%%%%%%%%%%%%
\begin{figure}[htbp]
  \centering
  \begin{tikzpicture}[%
      scale = 1.0,%
      input/.style = {coordinate},%
      output/.style = {coordinate},%
      block/.style = {%
        draw,%
        fill=white,%
        rectangle,%
        minimum height=3em,%
        minimum width=3em%
      },%
      >=latex'%
    ]
    \node [input] (IN) {};
    \node [block, right of=IN, node distance=2cm] (GD) {$G_D$};
    \node [output, right of=GD, node distance=2cm] (OUT) {};
    \draw [->] (IN) -- node[anchor=south]{$u_1(t)$} (GD);
    \draw [->] (GD) -- node[anchor=south]{$u_2(t)$} (OUT);
  \end{tikzpicture}
  \caption{Układ badany oraz sygnały}
  \label{fig:6DR1S}
\end{figure}
%%%%%%%%%%%%%%%%%%%%%%%%%%%%%%%%%%%%%%%%%%%%%%%%%%%%%%%%%%%%%%%%%%%%%%%%%%%%%%


Badany układ jest liniowy, więc napięcie $u_2$ mierzone na
jego zaciskach wyjściowych będzie również miało postać przebiegu sinusoidalnego
o~pewnej amplitudzie $U_2$ i~będzie przesunięty w~fazie względem sygnału
wzorcowego o~$\varphi_2$. Ogólnie, zapiszemy to jako
%%%%%%%%%%%%%%%%%%%%%%%%%%%%%%%%%%%%%%%%%%%%%%%%%%%%%%%%%%%%%%%%%%%%%%%%%%%%%%
\begin{align}
  & u_i(t) = U_i \sin{(\omega t + \varphi_i)}
         = P_i \sin{(\omega t)} + Q_i \cos{(\omega t)},
  &
  & i \in \set{1, 2}, &
  \label{eq:MG9JQ}
\end{align}
%%%%%%%%%%%%%%%%%%%%%%%%%%%%%%%%%%%%%%%%%%%%%%%%%%%%%%%%%%%%%%%%%%%%%%%%%%%%%%
gdzie
%%%%%%%%%%%%%%%%%%%%%%%%%%%%%%%%%%%%%%%%%%%%%%%%%%%%%%%%%%%%%%%%%%%%%%%%%%%%%%
\begin{align}
  & P_i = U_i \cos{\varphi_i} &
  & Q_i = U_i \sin{\varphi_i}, &
  & i \in \set{1, 2}. &
  \label{eq:9UTG8}
\end{align}
%%%%%%%%%%%%%%%%%%%%%%%%%%%%%%%%%%%%%%%%%%%%%%%%%%%%%%%%%%%%%%%%%%%%%%%%%%%%%%

\subsection{Metoda pozyskania składowych transmitancji $G_D$}
\label{sec:TV9TW}

Mieszając sygnały $u_i(t)$, $i \in \set{1, 2}$ z~sygnałami wzorcowymi $u_I(t) =
\sin{\omega t}$, $u_Q(t) = \cos{\omega t}$ można zmierzyć $P_i = U_i
\cos{\varphi_i}$, $Q_i = U_i \sin {\varphi_i}$, $i \in \set{1, 2}$. Mnożąc
mianowicie~\eqref{eq:MG9JQ} przez $\sin{\omega t}$ bądź $\cos{\omega t}$
otrzymujemy, zgodnie z~\eqref{eq:EE55O},
%%%%%%%%%%%%%%%%%%%%%%%%%%%%%%%%%%%%%%%%%%%%%%%%%%%%%%%%%%%%%%%%%%%%%%%%%%%%%%
\begin{subequations}
  \begin{align}
    &  U_i \sin{(\omega t + \varphi_i)}\sin{\omega t}
      = U_i \left(
        \frac{1}{2}\cos{\varphi_i} - \frac{1}{2}\cos{(2 \omega t + \varphi_i)}
      \right)
      = \frac{1}{2} P_i -
        \frac{1}{2} U_i \cos{(2 \omega t + \varphi_i)},
    &
    \label{eq:S0V86}
    \\
    & U_i \sin{(\omega t + \varphi_i)}\cos{\omega t}
      = U_i \left(
        \frac{1}{2}\sin{\varphi_i} + \frac{1}{2}\sin{(2 \omega t + \varphi_i)}
      \right)
      = \frac{1}{2} Q_i +
        \frac{1}{2} U_i \sin{(2 \omega t + \varphi_i)},
    &
    \label{eq:IU4JA}
  \end{align}
  \label{eq:M1ICA}
\end{subequations}
%%%%%%%%%%%%%%%%%%%%%%%%%%%%%%%%%%%%%%%%%%%%%%%%%%%%%%%%%%%%%%%%%%%%%%%%%%%%%%
dla $i \in \set{1, 2}$. Wartości $P_i$, $Q_i$ można
uzyskać mierząc składową stałą sygnałów wyjściowych z~mieszacza (stosując
np. odpowiedni filtr dolnoprzepustowy).

%%%%%%%%%%%%%%%%%%%%%%%%%%%%%%%%%%%%%%%%%%%%%%%%%%%%%%%%%%%%%%%%%%%%%%%%%%%%%%
\begin{figure}[htbp]
  \centering
  \begin{tikzpicture}[%
      scale = 1.0,%
      input/.style = {coordinate},%
      output/.style = {coordinate},%
      junction/.style = {draw, fill=black, circle, inner sep=1pt},%
      block/.style = {%
        draw,%
        fill=white,%
        rectangle,%
        minimum height=3em,%
        minimum width=3em%
      },%
      mul/.style = {%
        draw,%
        circle,%
        fill=white,%
        path picture={%
          \draw [black]%
          (path picture bounding box.135) -- (path picture bounding box.315)%
          (path picture bounding box.45) -- (path picture bounding box.225);%
        }%
      },%
      filt/.style = {%
        draw,%
        fill=white,%
        rectangle,%
        minimum height=3em,%
        minimum width=3em,%
        path picture={%
          \draw [black]%
          ($(path picture bounding box.west) + (1mm,0mm)$)%
          --%
          ($(path picture bounding box.east) + (-1mm,0mm)$)%
          ($(path picture bounding box.west) + (2mm,3mm)$)%
          --%
          ($(path picture bounding box.center) + (0mm,3mm)$)%
          --%
          ($(path picture bounding box.east) + (-2mm,-3mm)$);%
        }%
      },%
      coord/.style = {coordinate},
      >=latex'%
    ]
    \node [block] (GD) {$G_D$};
    \node [input, below of=GD, node distance=2.5cm] (IN) {};
    \node [output, above of=GD, node distance=2.5cm] (OUT) {};
    \node [junction, below of=GD, node distance=2cm] (JI0) {};
    \node [junction, above of=GD, node distance=2cm] (JO0) {};
    \node [junction, right of=JI0, node distance=2cm] (JI1) {};
    \node [junction, right of=JO0, node distance=2cm] (JO1) {};
    \node [coord, below of=JI1, node distance=0.75cm] (CI1I) {};
    \node [coord, above of=JI1, node distance=0.75cm] (CI1Q) {};
    \node [coord, above of=JO1, node distance=0.75cm] (CO1I) {};
    \node [coord, below of=JO1, node distance=0.75cm] (CO1Q) {};
    \node [mul, right of=CI1I, node distance=1cm] (MII) {};
    \node [mul, right of=CI1Q, node distance=1cm] (MIQ) {};
    \node [mul, right of=CO1I, node distance=1cm] (MOI) {};
    \node [mul, right of=CO1Q, node distance=1cm] (MOQ) {};
    \node [filt, right of=MII, node distance=2cm] (FII) {};
    \node [filt, right of=MIQ, node distance=2cm] (FIQ) {};
    \node [filt, right of=MOI, node distance=2cm] (FOI) {};
    \node [filt, right of=MOQ, node distance=2cm] (FOQ) {};
    \node [coord, below of=MII, node distance=1cm] (UII) {};
    \node [coord, above of=MIQ, node distance=1cm] (UIQ) {};
    \node [coord, above of=MOI, node distance=1cm] (UOI) {};
    \node [coord, below of=MOQ, node distance=1cm] (UOQ) {};
    \node [coord, right of=FII, node distance=2cm] (CII) {};
    \node [coord, right of=FIQ, node distance=2cm] (CIQ) {};
    \node [coord, right of=FOI, node distance=2cm] (COI) {};
    \node [coord, right of=FOQ, node distance=2cm] (COQ) {};
    \draw [->] (IN) -- node[anchor=east]{$u_1(t)$} (GD);
    \draw [->] (GD) -- node[anchor=east]{$u_2(t)$} (OUT);
    \draw [->] (JI0) -- (JI1) -| (CI1I) -- (MII);
    \draw [->] (JI0) -- (JI1) -| (CI1Q) -- (MIQ);
    \draw [->] (JO0) -- (JO1) -| (CO1I) -- (MOI);
    \draw [->] (JO0) -- (JO1) -| (CO1Q) -- (MOQ);
    \draw [->] (UII) -- node [anchor=west]{$u_I(t)$} (MII);
    \draw [->] (UOI) -- node [anchor=west]{$u_I(t)$} (MOI);
    \draw [->] (UIQ) -- node [anchor=west]{$u_Q(t)$} (MIQ);
    \draw [->] (UOQ) -- node [anchor=west]{$u_Q(t)$} (MOQ);
    \draw [->] (MII) -- (FII);
    \draw [->] (MIQ) -- (FIQ);
    \draw [->] (MOI) -- (FOI);
    \draw [->] (MOQ) -- (FOQ);
    \draw [->] (FII) -- node [anchor=south]{$\frac{1}{2}P_1$} (CII);
    \draw [->] (FIQ) -- node [anchor=south]{$\frac{1}{2}Q_1$} (CIQ);
    \draw [->] (FOI) -- node [anchor=south]{$\frac{1}{2}P_2$} (COI);
    \draw [->] (FOQ) -- node [anchor=south]{$\frac{1}{2}Q_2$} (COQ);
  \end{tikzpicture}
  \caption{Wyznaczanie składowych $P_i$ i $Q_i$ reprezentujących sygnały $u_i$.}
  \label{fig:UCMH6}
\end{figure}
%%%%%%%%%%%%%%%%%%%%%%%%%%%%%%%%%%%%%%%%%%%%%%%%%%%%%%%%%%%%%%%%%%%%%%%%%%%%%%

Interesująca nas transmitancja $G_D$ badanego układu (liniowego) objawia się
poprzez zmianę amplitudy (wzmocnienie $A_D$) i~przesunięcie fazowe
(o~$\varphi_D$) sygnału wyjściowego $u_2$ w~stosunku do wejściowego $u_1$.
Możemy to wyrazić jako
%%%%%%%%%%%%%%%%%%%%%%%%%%%%%%%%%%%%%%%%%%%%%%%%%%%%%%%%%%%%%%%%%%%%%%%%%%%%%%%%
\begin{multline}
  u_2(t) = A_D U_1 \sin{((\omega t + \varphi_1) + \varphi_D)} = A_D U_1 \left[
    \cos{\varphi_D} \sin{(\omega t + \varphi_1)
  + \sin{\varphi_D} \cos{(\omega t + \varphi_1)}}
  \right] = \\ = U_1 \left[
    P_D \sin{(\omega t + \varphi_1) + Q_D \cos{(\omega t + \varphi_1)}}
  \right] = \\ = U_1 \left[
    P_D \left(
      \sin{\omega t} \cos{\varphi_1}
    + \cos{\omega t} \sin{\varphi_1}
    \right) + Q_D \left(
      \cos{\omega t} \cos{\varphi_1}
    - \sin{\omega t} \sin{\varphi_1}
    \right)
  \right] = \\ = U_1 \left[
    \left(P_D \cos{\varphi_1} - Q_D \sin{\varphi_1}\right)\sin{\omega t}
  + \left(P_D \sin{\varphi_1} + Q_D \cos{\varphi_1}\right)\cos{\omega t}
  \right] = \\ =
    \left(P_1 P_D - Q_1 Q_D\right) \sin{\omega t}
  + \left(Q_1 P_D + P_1 Q_D\right) \cos{\omega t},
  \label{eq:HV0TF}
\end{multline}
%%%%%%%%%%%%%%%%%%%%%%%%%%%%%%%%%%%%%%%%%%%%%%%%%%%%%%%%%%%%%%%%%%%%%%%%%%%%%%%%
gdzie $P_D$ oraz $Q_D$ reprezentują transmitancję $G_D = P_D + j Q_D$ badanego
układu.

Przez porównanie \eqref{eq:MG9JQ} z~\eqref{eq:HV0TF} otrzymujemy układ równań
liniowych z~niewiadomymi $P_D$ i $Q_D$
%%%%%%%%%%%%%%%%%%%%%%%%%%%%%%%%%%%%%%%%%%%%%%%%%%%%%%%%%%%%%%%%%%%%%%%%%%%%%%%%
\begin{subequations}
  \begin{align}
    & P_1 P_D - Q_1 Q_D = P_2, &
    \label{eq:CKOLT}
    \\
    & Q_1 P_D + P_1 Q_D = Q_2, &
    \label{eq:5FLMS}
  \end{align}
  \label{eq:85PRD}
\end{subequations}
%%%%%%%%%%%%%%%%%%%%%%%%%%%%%%%%%%%%%%%%%%%%%%%%%%%%%%%%%%%%%%%%%%%%%%%%%%%%%%%%
z~którego rozwiązania dostajemy
%%%%%%%%%%%%%%%%%%%%%%%%%%%%%%%%%%%%%%%%%%%%%%%%%%%%%%%%%%%%%%%%%%%%%%%%%%%%%%%%
\begin{subequations}
  \begin{align}
    & P_D = \frac{P_1 P_2 + Q_1 Q_2}{P_1^2 + Q_1^2}, &
    \label{eq:DJRBO}
    \\
    & Q_D = \frac{P_1 Q_2 - Q_1 P_2}{P_1^2 + Q_1^2}. &
    \label{eq:AR3L3}
  \end{align}
  \label{eq:B7J8O}
\end{subequations}
%%%%%%%%%%%%%%%%%%%%%%%%%%%%%%%%%%%%%%%%%%%%%%%%%%%%%%%%%%%%%%%%%%%%%%%%%%%%%%%%

Formułę \eqref{eq:B7J8O} możemy wyrazić w~nieco innej postaci
%%%%%%%%%%%%%%%%%%%%%%%%%%%%%%%%%%%%%%%%%%%%%%%%%%%%%%%%%%%%%%%%%%%%%%%%%%%%%%%%
\begin{subequations}
  \begin{align}
    & P_D = \frac{P_1}{U_1} \frac{P_2}{U_1}
          + \frac{Q_1}{U_1} \frac{Q_2}{U_1}, &
    \label{eq:NXSH5}
    \\
    & Q_D = \frac{P_1}{U_1} \frac{Q_2}{U_1}
          - \frac{Q_1}{U_1} \frac{P_2}{U_1}, &
    \label{eq:6L48H}
  \end{align}
  \label{eq:F3CUU}
\end{subequations}
%%%%%%%%%%%%%%%%%%%%%%%%%%%%%%%%%%%%%%%%%%%%%%%%%%%%%%%%%%%%%%%%%%%%%%%%%%%%%%%%
gdzie
%%%%%%%%%%%%%%%%%%%%%%%%%%%%%%%%%%%%%%%%%%%%%%%%%%%%%%%%%%%%%%%%%%%%%%%%%%%%%%%%
\begin{equation}
  U_1 = \sqrt{P_1^2 + Q_1^2}.
  \label{eq:W79B6}
\end{equation}
%%%%%%%%%%%%%%%%%%%%%%%%%%%%%%%%%%%%%%%%%%%%%%%%%%%%%%%%%%%%%%%%%%%%%%%%%%%%%%%%

Zauważmy, że sygnały przeskalowane $\frac{P_i}{U_1}$ oraz  $\frac{Q_i}{U_1}$
będą miały ograniczone amplitudy
%%%%%%%%%%%%%%%%%%%%%%%%%%%%%%%%%%%%%%%%%%%%%%%%%%%%%%%%%%%%%%%%%%%%%%%%%%%%%%%%
\begin{subequations}
  \begin{align}
    & \left|\frac{P_1}{U_1}\right| \le 1 &
    & \left|\frac{Q_1}{U_1}\right| \le 1, &
    \label{eq:4AA9D}
    \\
    & \left|\frac{P_2}{U_1}\right| \le A_D &
    & \left|\frac{Q_2}{U_1}\right| \le A_D. &
    \label{eq:TX2NW}
  \end{align}
\end{subequations}
%%%%%%%%%%%%%%%%%%%%%%%%%%%%%%%%%%%%%%%%%%%%%%%%%%%%%%%%%%%%%%%%%%%%%%%%%%%%%%%%


%%\begin{appendices}
%%\section{Kinematyka wahadła z~przegubem Cardana}
%%\label{sec:4XR18}
%%
%%W~poniższej sekcji wyprowadzono równania opisujące ruch modelu kinematycznego
%%badanego wahadła. Celem jest uzyskanie formuł, które wiązałyby ze sobą
%%przyspieszenia $\hat{\brm{a}}^{\prime}$ zmierzone przy pomocy akcelerometru
%%z~drugimi pochodnymi $\ddot{\varphi}_x$, $\ddot{\varphi}_y$ wychyleń kątowych
%%w~przegubie Cardana. Zastosowanie metod numerycznego całkowania, czy
%%różniczkowania, umożliwi nam następnie porównanie trajektorii uzyskanych
%%różnymi sposobami.
%%
%%\subsection{Oddziaływanie przyspieszenia grawitacyjnego na akcelerometry}
%%\label{sec:LIIHL}
%%
%%W~obecności pola grawitacyjnego, wartości przyspieszeń odczytanych
%%z~akcelerometru, oprócz obserwowanego przyspieszenia kinematycznego (ruchu),
%%będą zawierały jeszcze składowe od przyspieszenia grawitacyjnego. Poniższa
%%dyskusja objaśnia istotę tego zjawiska i~wprowadza technikę eliminacji
%%składowych grawitacyjnych.
%%
%%\subsubsection{Przykłady kanoniczne}
%%\label{sec:YFBVX}
%%
%%\paragraph{Nieruchomy akcelerometr.}
%%{\em Nieruchomy} akcelerometr będzie wskazywał wzdłuż wybranej osi  zerowe
%%przyspieszenie tylko wtedy, gdy ustawimy tę oś idealnie w~płaszczyźnie
%%poziomej (prostopadłej do wektora grawitacji). W~skrajnie odmiennej sytuacji,
%%tj. jeśli skierujemy oś nieruchomego akcelerometru idealnie wzdłuż wektora
%%grawitacji $\brm{g}$, wskazanie na tej osi powinno wynieść $+1g$ bądź $-1g$.
%%Przyjmijmy, że jeśli akcelerometr porusza się z~dodatnim przyspieszeniem wzdłuż
%%dowolnej swojej osi, np. $z^{\prime}$, to odczytana z~tego kanału wartość
%%przyspieszenia powinna być dodatnia. Przy takiej konwencji, oś {\em
%%nieruchomego} akcelerometru skierowana pionowo w~górę (przeciwnie do wektora
%%grawitacji), da odczyt $+1g$ (dodatni). Wynika to z~zależności
%%\eqref{eq:QDX8V} poniżej, gdzie dla nieruchomego akcelerometru należy
%%przyjąć $\brm{a} = \brm{0}$. Powyższe obserwacje można wykorzystać np.~do
%%skalowania akcelerometru, tj. do określenia współczynników jego
%%charakterystyki.
%%
%%\paragraph{Swobodne spadanie.}
%%Aby wskazanie wzdłuż pionowo ustawionej osi akcelerometru było zerowe,
%%należałoby mu pozwolić swobodnie opadać pod wpływem pola grawitacyjnego.
%%Jeśli oś (np.~$z^{\prime}$)  akcelerometru ustawimy idealnie pionowo w~górę
%%(zgodnie z~wektorem grawitacji) i~uwolnimy akcelerometr od więzów,
%%pozwalając mu swobodnie spadać, wskazanie wzdłuż tej osi będzie zerowe, tzn.
%%$\hat{a}_z =0$. Wiemy jednak, że względem nieruchomego układu współrzędnych
%%akcelerometr porusza się z~przyspieszeniem $a_z = -g$. A~więc faktyczne
%%przyspieszenie kinematyczne jest $a_z = \hat{a}_z - g$. W~ogólności
%%zapisalibyśmy to w~nieruchomym układzie współrzędnych jako
%%%%%%%%%%%%%%%%%%%%%%%%%%%%%%%%%%%%%%%%%%%%%%%%%%%%%%%%%%%%%%%%%%%%%%%%%%%%%%%%
%%\begin{equation}
%%  \brm{a} = \hat{\brm{a}} + \brm{g},
%%  \label{eq:QDX8V}
%%\end{equation}
%%%%%%%%%%%%%%%%%%%%%%%%%%%%%%%%%%%%%%%%%%%%%%%%%%%%%%%%%%%%%%%%%%%%%%%%%%%%%%%%
%%gdzie
%%%%%%%%%%%%%%%%%%%%%%%%%%%%%%%%%%%%%%%%%%%%%%%%%%%%%%%%%%%%%%%%%%%%%%%%%%%%%%%%
%%\begin{equation}
%%  \brm{g} = \begin{bmatrix} 0 & 0 & -g \end{bmatrix}^T, \;\; g \approx 9.81 \frac{m}{s^2}
%%  \label{eq:PWKKB}
%%\end{equation}
%%%%%%%%%%%%%%%%%%%%%%%%%%%%%%%%%%%%%%%%%%%%%%%%%%%%%%%%%%%%%%%%%%%%%%%%%%%%%%%%
%%jest wektorem przyspieszenia ziemskiego, oraz
%%%%%%%%%%%%%%%%%%%%%%%%%%%%%%%%%%%%%%%%%%%%%%%%%%%%%%%%%%%%%%%%%%%%%%%%%%%%%%%%
%%\begin{align}
%%  & \brm{a} = \begin{bmatrix}
%%    a_x & a_y & a_z
%%  \end{bmatrix}^T, &
%%  & \hat{\brm{a}} = \begin{bmatrix}
%%    \hat{a}_x & \hat{a}_y & \hat{a}_z
%%  \end{bmatrix}^T &
%%  \label{eq:0JPXH}
%%\end{align}
%%%%%%%%%%%%%%%%%%%%%%%%%%%%%%%%%%%%%%%%%%%%%%%%%%%%%%%%%%%%%%%%%%%%%%%%%%%%%%%%
%%są wektorami algebraicznymi przyspieszeń ,,netto'' i ,,brutto'' akcelerometru
%%wyrażonymi w~nieruchomym układzie współrzędnych.
%%
%%\subsubsection{Oddziaływanie na akcelerometr zamocowany na wahadle}
%%\label{sec:JO17Z}
%%
%%Oczywiście, fizyka oddziaływania pola grawitacyjnego na akcelerometr nie zmieni
%%się w~wyniku jego zamontowaniu na końcówce wahadła. Celem niniejszego
%%podrozdziału jest jedynie określenie składowych wektora grawitacji
%%pojawiających się na osiach akcelerometru w~zależności od~wychyleń kątowych
%%wahadła (a~więc wypracowanie odpowiednich transformacji geometrycznych).
%%Dotychczasowe nasze rozważania przeprowadzono w~nieruchomym układzie
%%współrzędnych $x,y,z$ jednak w~eksperymencie mamy do czynienia
%%z~przyspieszeniami wyrażonymi w~układzie
%%ruchomym~$x^{\prime},y^{\prime},z^{\prime}$ (osie akcelerometru są w~każdej
%%chwili równoległe do osi układu ruchomego $x^{\prime},y^{\prime},z^{\prime}$).
%%
%%Rysunek~\ref{fig:AFKVE} pokazuje w~dwóch rzutach wahadło w~położeniu określonym
%%przez kąty~$\varphi_x$, $\varphi_y$. W~prawej części rysunku pokazano rzut
%%na płaszczyznę $y-z$, w~której odbywa się obrót $\varphi_x$. Grot osi $x$
%%skierowany jest w~stronę obserwatora. W~lewej części rysunku pokazano rzut na
%%płaszczyznę~$z^{\prime\prime}, x^{\prime\prime}$, w~której odbywa się
%%obrót~$\varphi_y$ (tak jak to widzi oko obserwatora z~prawego rysunku).
%%%%%%%%%%%%%%%%%%%%%%%%%%%%%%%%%%%%%%%%%%%%%%%%%%%%%%%%%%%%%%%%%%%%%%%%%%%%%%%%
%%\begin{figure}[htbp]
%%  \centering
%%  \begin{tikzpicture}[%
%%      scale=0.85,%
%%      axis/.style={very thin,->},%
%%      rodr/.style={thick,double distance=0.25cm,cap=round},%
%%      rodr2/.style={thick,double distance=0.35cm,cap=round},%
%%      rodh/.style={thick,double distance=0.25cm,cap=rect},%
%%      rodh2/.style={thick,double distance=0.35cm,cap=rect},%
%%      zxplane/.style={canvas is zx plane at y=#1},%
%%      yxplane/.style={canvas is yx plane at z=#1},%
%%      yzplane/.style={canvas is yz plane at x=#1},%
%%  ]
%%
%%    \begin{scope}[x={(0,0)},y={(1cm,0)},z={(0,1cm)}]
%%
%%      \coordinate (r) at (0,0,-4);
%%
%%      \draw[dashdotted] (0,0,0) -- ++(0,0,-3.5);
%%      \begin{scope}[rotate around x=30]
%%        %eye
%%        \pgfmathsetmacro{\eyeSize}{1}
%%        \pgfmathsetmacro{\ex}{4}
%%        \pgfmathsetmacro{\ey}{-2.5}
%%        \pgfmathsetmacro{\eRot}{-180}
%%        \pgfmathsetmacro{\eAp}{-55}
%%        \draw[yzplane=0,rotate around={\eRot:(\ex,\ey)}] (\ex,\ey) -- ++(-.5*\eAp:\eyeSize)
%%             (\ex,\ey) -- ++(.5*\eAp:\eyeSize);
%%        \draw[yzplane=0] (\ex,\ey) ++(\eRot+\eAp:.75*\eyeSize) arc (\eRot+\eAp:\eRot-\eAp:.75*\eyeSize);
%%
%%        % IRIS
%%        \draw[yzplane=0,fill=gray] (\ex,\ey) ++(\eRot+\eAp/3:.75*\eyeSize) % start point
%%          arc (\eRot+180-\eAp:\eRot+180+\eAp:.28*\eyeSize);
%%
%%        %PUPIL, a filled arc
%%        \draw[yzplane=0,fill=black] (\ex,\ey) ++(\eRot+\eAp/3:.75*\eyeSize) % start point
%%          arc (\eRot+\eAp/3:\eRot-\eAp/3:.75*\eyeSize);
%%        %
%%        \draw[draw=lightgray,thin,dashed,->] (0,3,-2.5) -- ++ (0,-2.5,0);
%%        %
%%        \draw[rodh](0,0,-0.50)-- (0,0,-4) node[above=0.20cm,right=0.20cm] {$r$};
%%        \draw[rodh](0,-0.85,0) -- (0,0.85,0);
%%        \draw[rodh2](0,-0.75,0.1) -- (0,-0.75,-0.50) -- (0,0.75,-0.50) -- (0,0.75,0.1);
%%        \draw[thin,dashdotted] (0,0,0) -- ++ (0,0,-4);
%%        % x''-y''-z'' coordinate system
%%        \draw[dashed,axis] (0,0,0) -- ++(0,2,0) node[right] {$y^{\prime\prime},y^{\prime}$};
%%        \draw[dashed,axis] (0,0,0) -- ++(0,0,2) node[above] {$z^{\prime\prime},z^{\prime}$};
%%
%%        \begin{scope}[shift={(30:-4cm)}]
%%          \draw[thin,dashed] (0,0,-5.5) -- (0,0,3);
%%        \end{scope}
%%
%%        \begin{scope}[shift={(30:-8cm)}]
%%          \begin{scope}[rotate around z=-90]
%%            \draw[rodh] (-0.85,0,0) -- (0.85,0,0);
%%            % base
%%            \draw[rodh2](-0.75,0,0.75) -- (-0.75,0,-0.1);
%%            \draw[rodh2](0.75,0,-0.1) -- (0.75,0,0.75);
%%            \draw[zxplane=0,fill=white,postaction={pattern=north east lines}] (0.75,-1) rectangle (1.25,1);
%%            \draw[dashdotted] (0,0,0) -- ++(0,0,-3.5);
%%            % pendulum
%%            \begin{scope}[rotate around y=30]
%%              \draw[rodh](0,0,0) -- ++(0,0,{-4/cos(30)}) node[above=0.30cm,right=0.05cm] {$r$};
%%              \draw[rodr2](0,0,{-0.35/cos(30)}) -- (0,0,0);
%%              \draw[zxplane=0] (0,0,0) circle (0.125);
%%              \draw[thin,dashdotted,cap=round] (0,0,0) -- ++ (0,0,{-4/cos(30)});
%%              % x'-y'-z' coordinate system
%%              \draw[axis] (0,0,0) -- ++(2,0,0) node[right] {$x^{\prime}$};
%%              \draw[axis] (0,0,0) -- ++(0,0,2) node[above] {$z^{\prime}$};
%%            \end{scope}
%%            % angle \varphi_y
%%            \draw[axis,zxplane=0] (0,0) ++ (180:2) arc(180:210:2) node [below=0.25cm] {$\varphi_y$};
%%            % gravity
%%            \draw[thick,->,zxplane=0] (-150:{4/cos(30)}) -- ++ (-{2*cos(30)},0) node[below] {$g\cos{\varphi_x}$};
%%            \draw[thick,->,zxplane=0] (-150:{4/cos(30)}) -- ++ (-150:{2*cos(30)*cos(30)}) node[below=0.25cm,right] {$g\cos{\varphi_x}\cos{\varphi_y}$};
%%            \draw[thick,->,zxplane=0] (-150:{4/cos(30)}) -- ++ (-240:{2*cos(30)*sin(30)}) node[below=0.25cm,left] {$g\cos{\varphi_x}\sin{\varphi_y}$};
%%            % x''-y''-z'' coordinate system
%%            \draw[dashed,axis] (0,0,0) -- ++(2,0,0) node[left] {$x^{\prime\prime}$};
%%            \draw[dashed,axis] (0,0,0) -- ++(0,0,2) node[above] {$z^{\prime\prime},z$};
%%          \end{scope}
%%        \end{scope}
%%
%%      \end{scope}
%%      \draw[axis,yzplane=0] (0,0) ++ (-90:2) arc(-90:-60:2) node [midway, below] {$\varphi_x$};
%%
%%      \draw[rodr2] (0,0,0) -- (0,0,0.75);
%%      \draw[yzplane=0,fill=white,postaction={pattern=north east lines}] (-1,0.75) rectangle (1,1.25);
%%      \draw[yzplane=0] (0,0,0) circle (0.125);
%%
%%      % gravity
%%      \draw[thick,->,yzplane=0] (-60:4) -- ++ (0,-2) node[below] {$g$};
%%      \draw[thick,->,yzplane=0] (-60:4) -- ++ ( -60:{2*cos(30)}) node[below right] {$g\cos{\varphi_x}$};
%%      \draw[thick,->,yzplane=0] (-60:4) -- ++ (-150:{2*sin(30)}) node[left] {$g\sin{\varphi_x}$};
%%      % gravity helper lines
%%      \draw[thin,draw=lightgray,dashed,yzplane=0] {(-60:4)+(-150:{2*sin(30)})} -- ++ (-150:6);
%%      \draw[thin,draw=lightgray,dashed,yzplane=0] {(-60:{4+2*cos(30)})} -- ++ (-150:6);
%%
%%      % x-y-z coordinate system
%%      \draw[axis] (0,0,0) -- ++(0,2,0) node[right] {$y$};
%%      \draw[axis] (0,0,0) -- ++(0,0,2) node[above] {$z$};
%%    \end{scope}
%%  \end{tikzpicture}
%%  \caption{Oddziaływanie przyspieszenia ziemskiego na końcówkę wahadła}
%%  \label{fig:AFKVE}
%%\end{figure}
%%%%%%%%%%%%%%%%%%%%%%%%%%%%%%%%%%%%%%%%%%%%%%%%%%%%%%%%%%%%%%%%%%%%%%%%%%%%%%%%
%%
%%Zakładając, że osie akcelerometru są zgodne z~osiami układu ruchomego
%%$x^{\prime},y^{\prime},z^{\prime}$ wnioskujemy na podstawie rysunku oraz
%%wcześniejszych rozważań, że przyspieszenie kinematyczne $\brm{a}^{\prime}$
%%(,,netto'') będzie związane ze~wskazywanym przez akcelerometr przyspieszeniem
%%(,,brutto'')~$\hat{\brm{a}}^{\prime}$ w~następujący sposób
%%%%%%%%%%%%%%%%%%%%%%%%%%%%%%%%%%%%%%%%%%%%%%%%%%%%%%%%%%%%%%%%%%%%%%%%%%%%%%%%
%%\begin{equation}
%%  {\brm{a}^{\prime}} = \hat{\brm{a}}^{\prime}
%%  -
%%  g \cdot
%%  \begin{bmatrix}
%%              -\cos{\varphi_x} \sin{\varphi_y}  \\
%%                      \sin{\varphi_x}          \\
%%    \phantom{-}\cos{\varphi_x} \cos{\varphi_y}
%%  \end{bmatrix},
%%  \label{eq:EUIQU}
%%\end{equation}
%%%%%%%%%%%%%%%%%%%%%%%%%%%%%%%%%%%%%%%%%%%%%%%%%%%%%%%%%%%%%%%%%%%%%%%%%%%%%%%%
%%przy czym
%%%%%%%%%%%%%%%%%%%%%%%%%%%%%%%%%%%%%%%%%%%%%%%%%%%%%%%%%%%%%%%%%%%%%%%%%%%%%%%%
%%\begin{align}
%%  & \brm{a}^{\prime} = \begin{bmatrix}
%%    a_x^{\prime}  & a_y^{\prime}  & a_z^{\prime}
%%  \end{bmatrix}^T, &
%%  & \hat{\brm{a}}^{\prime}  = \begin{bmatrix}
%%    \hat{a}_x^{\prime}  & \hat{a}_y^{\prime}  & \hat{a}_z^{\prime}
%%  \end{bmatrix}^T &
%%  \label{eq:NL9I0}
%%\end{align}
%%%%%%%%%%%%%%%%%%%%%%%%%%%%%%%%%%%%%%%%%%%%%%%%%%%%%%%%%%%%%%%%%%%%%%%%%%%%%%%%
%%są przyspieszeniami ,,netto'' i~,,brutto'' akcelerometru wyrażonymi w~układzie
%%ruchomym $x^{\prime},y^{\prime},z^{\prime}$.
%%
%%\subsection{Związek między przyspieszeniami liniowymi i~kątowymi}
%%\label{sec:QTVP6}
%%
%%Przyspieszenie ,,netto'' $\brm{a}^{\prime}$ odzwierciedla liniowe
%%przyspieszenie kinematyczne końcówki wahadła wyrażone w~układzie ruchomym.
%%Jeśliby przyjąć, że wynikające z~pomiarów przyspieszenia $\brm{a}^{\prime}$
%%odpowiadają dokładnie faktycznym przyspieszeniom kinematycznym końcówki
%%(perfekcyjny eksperyment) to przyspieszenia kątowe $\ddot{\varphi_x}$
%%i~$\ddot{\varphi_y}$ będą wynikały wprost z~przyspieszeń $\brm{a}^{\prime}$
%%%%%%%%%%%%%%%%%%%%%%%%%%%%%%%%%%%%%%%%%%%%%%%%%%%%%%%%%%%%%%%%%%%%%%%%%%%%%%%%
%%\begin{subequations}
%%  \label{eq:9HDQE}
%%  \begin{align}
%%    \ddot{\varphi}_x &=\phantom{-}r^{-1} a_y^{\prime},
%%    \label{eq:UIQUR}
%%    \\
%%    \ddot{\varphi}_y &=         - r^{-1} a_x^{\prime},
%%    \label{eq:ZBLRS}
%%  \end{align}
%%\end{subequations}
%%%%%%%%%%%%%%%%%%%%%%%%%%%%%%%%%%%%%%%%%%%%%%%%%%%%%%%%%%%%%%%%%%%%%%%%%%%%%%%%
%%($r$ jest długością wahadła). Równania \eqref{eq:9HDQE} stanowią
%%układ równań różniczkowych zwyczajnych drugiego rzędu z~niewiadomymi funkcjami
%%$\varphi_x(t)$, $\varphi_y(t)$ oraz dobrze zdefiniowanymi prawymi stronami
%%$r^{-1}a_x^{\prime}(t,\varphi_x,\varphi_y)$, $r^{-1}a_y^{\prime}(t,\varphi_x)$
%%%%%%%%%%%%%%%%%%%%%%%%%%%%%%%%%%%%%%%%%%%%%%%%%%%%%%%%%%%%%%%%%%%%%%%%%%%%%%%%
%%\begin{subequations}
%%  \label{eq:ROGUO}
%%  \begin{align}
%%    \ddot{\varphi}_x & = \phantom{-} r^{-1} \left(
%%      \hat{a}_y^{\prime}\left(t\right) - g \sin{\varphi_x}
%%    \right),
%%    \label{eq:HRYMM}
%%    \\
%%    \ddot{\varphi}_y & = - r^{-1} \left(
%%        \hat{a}_x^{\prime}\left(t\right) + g \cos{\varphi_x} \sin{\varphi_y}
%%    \right).
%%    \label{eq:D4N2B}
%%  \end{align}
%%\end{subequations}
%%%%%%%%%%%%%%%%%%%%%%%%%%%%%%%%%%%%%%%%%%%%%%%%%%%%%%%%%%%%%%%%%%%%%%%%%%%%%%%%
%%Równania takie, po opatrzeniu odpowiednimi warunkami początkowymi, można
%%rozwiązywać np.~przy użyciu procedur całkowania numerycznego. W~naszym
%%przypadku, użycie tego typu metody do danych pomiarowych umożliwiłoby
%%wyznaczenie przybliżonych przebiegów wychyleń wahadła $\varphi_x$, $\varphi_y$
%%na podstawie przyspieszeń zebranych z~akcelerometrów.
%%
%%\subsection{Redukcja równań różniczkowych II rzędu do postaci równań I rzędu}
%%\label{sec:WKKBB}
%%
%%Aby zastosować procedurę całkowania numerycznego, problem powinien być
%%sformułowany jako zagadnienie początkowe z~układem równań różniczkowych
%%pierwszego rzędu, czyli w~postaci~\eqref{eq:L8KAD}. W~przypadku
%%równań~\eqref{eq:ROGUO} można to osiągnąć poprzez zastosowanie następującego
%%zestawu zmiennych
%%%%%%%%%%%%%%%%%%%%%%%%%%%%%%%%%%%%%%%%%%%%%%%%%%%%%%%%%%%%%%%%%%%%%%%%%%%%%%%%
%%\begin{equation}
%%  \brm{y} = \begin{bmatrix}
%%    y_1 & y_2 & y_3 & y_4
%%  \end{bmatrix}^T \coloneqq \begin{bmatrix}
%%    \varphi_x & \varphi_y & \dot{\varphi}_x & \dot{\varphi}_y
%%  \end{bmatrix}^T.
%%  \label{eq:U09S6}
%%\end{equation}
%%%%%%%%%%%%%%%%%%%%%%%%%%%%%%%%%%%%%%%%%%%%%%%%%%%%%%%%%%%%%%%%%%%%%%%%%%%%%%%%
%%Teraz, w~celu wyrażenia równań~\eqref{eq:ROGUO} w~postaci~\eqref{eq:R7M37}
%%wystarczy zdefiniować funkcję $\brm{f}$ jako
%%%%%%%%%%%%%%%%%%%%%%%%%%%%%%%%%%%%%%%%%%%%%%%%%%%%%%%%%%%%%%%%%%%%%%%%%%%%%%%%
%%\begin{equation}
%%  \brm{f}\left(t,\brm{y}\right) \coloneqq \begin{bmatrix}
%%    y_3 \\
%%    y_4 \\
%%    r^{-1} \left( {\hat{a}_y}^{\prime}\left(t\right) - g \sin{y_1} \right)  \\
%%    r^{-1} \left(-{\hat{a}_x}^{\prime}\left(t\right) - g \cos{y_1}\sin{y_2} \right)
%%  \end{bmatrix}.
%%  \label{eq:S8LTL}
%%\end{equation}
%%%%%%%%%%%%%%%%%%%%%%%%%%%%%%%%%%%%%%%%%%%%%%%%%%%%%%%%%%%%%%%%%%%%%%%%%%%%%%%%
%%Jako $\brm{y}_0$ w~warunkach początkowych \eqref{eq:5DFP4} należy oczywiście
%%użyć
%%%%%%%%%%%%%%%%%%%%%%%%%%%%%%%%%%%%%%%%%%%%%%%%%%%%%%%%%%%%%%%%%%%%%%%%%%%%%%%%
%%\begin{equation}
%%  \brm{y}_0 = \begin{bmatrix}
%%    \varphi_{x,0} & \varphi_{y,0} & \dot{\varphi}_{x,0} & \dot{\varphi}_{y,0}
%%  \end{bmatrix}^T.
%%\end{equation}
%%%%%%%%%%%%%%%%%%%%%%%%%%%%%%%%%%%%%%%%%%%%%%%%%%%%%%%%%%%%%%%%%%%%%%%%%%%%%%%%
%%
%%\section{Metody całkowania numerycznego}
%%\label{sec:VNLHT}
%%
%%Współczesna numeryka oferuje cały wachlarz metod obliczeniowych umożliwiających
%%przybliżone rozwiązywanie zagadnień różniczkowych. Poszczególne metody
%%numerycznego całkowania równań różniczkowych różnią się między sobą m.in.
%%takimi właściwościami, jak rząd dokładności, obszar stabilności, tłumienie
%%numeryczne~\cite{asher&petzold:1998:computer-methods}.
%%
%%Tradycyjne metody numerycznego całkowania równań różniczkowych zwyczajnych
%%umożliwiają rozwiązanie zagadnienia początkowego sformułowanego jako
%%%%%%%%%%%%%%%%%%%%%%%%%%%%%%%%%%%%%%%%%%%%%%%%%%%%%%%%%%%%%%%%%%%%%%%%%%%%%%%%
%%\begin{subequations}
%%\label{eq:L8KAD}
%%\begin{align}
%%  \dot{\brm{y}}\left(t\right) &= \brm{f}\left(t, \brm{y}(t)\right),
%%  \label{eq:R7M37}
%%  \\
%%  \brm{y}\left(0\right) &= \brm{y}_0 \in \mathbb{R}^n,
%%  \label{eq:5DFP4}
%%\end{align}
%%\end{subequations}
%%%%%%%%%%%%%%%%%%%%%%%%%%%%%%%%%%%%%%%%%%%%%%%%%%%%%%%%%%%%%%%%%%%%%%%%%%%%%%%%
%%gdzie $\brm{y}: \mathbb{R} \mapsto \mathbb{R}^n$ jest poszukiwaną funkcją.
%%Równanie \eqref{eq:R7M37} jest równaniem różniczkowym zwyczajnym (pierwszego
%%rzędu), zaś \eqref{eq:5DFP4} definiuje warunki początkowe dla zagadnienia.
%%
%%Klasyfikacji metod numerycznego całkowania dokonuje się wg rozmaitych
%%kryteriów. Pod względem konstrukcji schematu różnicowego, możemy wyodrębnić
%%podział na metody jedno i~wielokrokowe, a~w~każdej z~tych kategorii będą
%%występować schematy jawne i~niejawne. Przy ocenie dokładności rozwiązania
%%uzyskiwanego daną metodą zwykle mówi się o~rzędzie metody (mówi się, że metoda
%%jest rzędu $p$ jeśli błąd uzyskiwanego przybliżenia numerycznego jest $d =
%%O(h^{p+1})$, gdzie $h$ jest krokiem całkowania). Na potrzeby laboratorium
%%wprowadzimy kilka znanych metod całkowania.
%%
%%\subsection{Metody jednokrokowe}
%%\label{sec:5BIFZ}
%%
%%\subsubsection{Metody Eulera}
%%\label{sec:AN1EA}
%%
%%Znane i stosowane są dwa warianty metody Eulera -- jawna i niejawna. Metody
%%Eulera są metodami pierwszego rzędu. Różnią się między sobą obszarem
%%stabilności.
%%
%%\paragraph{Jawna metoda Eulera.}
%%W~metodzie jawnej Eulera używa się następującego schematu różnicowego do
%%przybliżenia pochodnej $\dot{\brm{y}}$ (schemat różnicowy ,,w~przód'')
%%%%%%%%%%%%%%%%%%%%%%%%%%%%%%%%%%%%%%%%%%%%%%%%%%%%%%%%%%%%%%%%%%%%%%%%%%%%%%%%
%%\begin{equation}
%%  \dot{\brm{y}}_{k-1} = \frac{\brm{y}_{k} - \brm{y}_{k-1}}{h},
%%  \label{eq:IHSVX}
%%\end{equation}
%%%%%%%%%%%%%%%%%%%%%%%%%%%%%%%%%%%%%%%%%%%%%%%%%%%%%%%%%%%%%%%%%%%%%%%%%%%%%%%%
%%gdzie $\brm{y}_{l} \approx \brm{y}\left(t_l\right)$ jest przybliżeniem wartości
%%poszukiwanej funkcji w~punkcie $t_l$ a~$\dot{\brm{y}}_{l} \approx
%%\dot{\brm{y}}(t_l)$ przybliżeniem pochodnej. W~wyniku prostego przekształcenia,
%%otrzymuje się jawną metodę Eulera
%%%%%%%%%%%%%%%%%%%%%%%%%%%%%%%%%%%%%%%%%%%%%%%%%%%%%%%%%%%%%%%%%%%%%%%%%%%%%%%%
%%\begin{equation}
%%  \brm{y}_k = \brm{y}_{k-1} + h \cdot \brm{f}_{k-1},\;\;k = 1,\dots,K,
%% \label{eq:3PVVU}
%%\end{equation}
%%%%%%%%%%%%%%%%%%%%%%%%%%%%%%%%%%%%%%%%%%%%%%%%%%%%%%%%%%%%%%%%%%%%%%%%%%%%%%%%
%%w~której $\brm{f}_l = \brm{f}\left(t_l, \brm{y}_l\right)$. Zauważmy, że dla
%%$k=1$ mamy po prawej stronie $\brm{y}_0$ znane z~warunków początkowych, więc
%%możemy od razu wyliczyć $\brm{y}_1$ i~dalej, krok po kroku, $\brm{y}_2$,
%%$\brm{y}_3$, itd..
%%
%%\paragraph{Niejawna metoda Eulera.}
%%W~niejawnej metodzie Eulera, pochodną aproksymuje się następującym schematem
%%różnicowym (wsteczna różnica skończona)
%%%%%%%%%%%%%%%%%%%%%%%%%%%%%%%%%%%%%%%%%%%%%%%%%%%%%%%%%%%%%%%%%%%%%%%%%%%%%%%%
%%\begin{equation}
%%  \dot{\brm{y}}_k = \frac{\brm{y}_{k} - \brm{y}_{k-1}}{h}.
%%  \label{eq:YNJO9}
%%\end{equation}
%%%%%%%%%%%%%%%%%%%%%%%%%%%%%%%%%%%%%%%%%%%%%%%%%%%%%%%%%%%%%%%%%%%%%%%%%%%%%%%%
%%Uzyskana metoda całkowania ma więc postać
%%%%%%%%%%%%%%%%%%%%%%%%%%%%%%%%%%%%%%%%%%%%%%%%%%%%%%%%%%%%%%%%%%%%%%%%%%%%%%%%
%%\begin{equation}
%%  \brm{y}_k = \brm{y}_{k-1} + h \cdot \brm{f}_{k},\;\;k = 1,\dots,K.
%%  \label{eq:R8JE3}
%%\end{equation}
%%%%%%%%%%%%%%%%%%%%%%%%%%%%%%%%%%%%%%%%%%%%%%%%%%%%%%%%%%%%%%%%%%%%%%%%%%%%%%%%
%%Należy zaznaczyć, że równanie \eqref{eq:R8JE3} jest nieliniowym równaniem
%%uwikłanym, ponieważ niewiadome~$\brm{y}_k$ są uwikłane w~definicji nieliniowej
%%funkcji $\brm{f}_k = \brm{f}(t_k,\brm{y}_k)$. W~każdym kroku całkowania należy
%%więc zastosować odpowiednią (iteracyjną) metodę rozwiązywania równań
%%nieliniowych (np.~metodę punktu stałego, bądź metodę Newtona).
%%
%%\subsection{Metody wielokrokowe}
%%\label{sec:QS6QT}
%%
%%Ogólny przepis na (liniową) metodę wielokrokową jest następujący
%%%%%%%%%%%%%%%%%%%%%%%%%%%%%%%%%%%%%%%%%%%%%%%%%%%%%%%%%%%%%%%%%%%%%%%%%%%%%%%%
%%\begin{equation}
%%  \sum_{j=0}^J \alpha_j \brm{y}_{k-j} = h \sum_{j=0}^J \beta_j \brm{f}_{k-j}.
%%  \label{eq:2AOXP}
%%\end{equation}
%%%%%%%%%%%%%%%%%%%%%%%%%%%%%%%%%%%%%%%%%%%%%%%%%%%%%%%%%%%%%%%%%%%%%%%%%%%%%%%%
%%Konkretne metody całkowania konstruuje się poprzez zastosowanie pewnych
%%(znanych) zestawów współczynników $\alpha_j$ i $\beta_j$. Zakłada się przy tym,
%%że $\alpha_0 \neq 0$ (w~wyśmienitej większości metod $\alpha_0 = 1$).
%%Liczba~$J$ definiuje liczbę kroków metody. Mówi ona ile kroków wstecz dana
%%metoda spogląda aby wyznaczyć bieżącą wartość $\brm{y}_k$. W~grupie metod
%%wielo-krokowych będą występować {\em metody jawne} ($\beta_0 = 0$) i~{\em
%%metody niejawne} ($\beta_0 \neq 0$).
%%
%%Przyjmując, zgodnie z~wcześniejszą uwagą, $\alpha_0 = 1$, wzór \eqref{eq:2AOXP}
%%można przekształcić do postaci
%%%%%%%%%%%%%%%%%%%%%%%%%%%%%%%%%%%%%%%%%%%%%%%%%%%%%%%%%%%%%%%%%%%%%%%%%%%%%%%%
%%\begin{equation}
%%  \brm{y}_k = \sum_{j=1}^J \alpha_j \brm{y}_{k-j} + h \sum_{j=0}^J \beta_j \brm{f}_{k-j}
%%  \label{eq:NN23A}
%%\end{equation}
%%%%%%%%%%%%%%%%%%%%%%%%%%%%%%%%%%%%%%%%%%%%%%%%%%%%%%%%%%%%%%%%%%%%%%%%%%%%%%%%
%%Dla $k>=J$ wszystkie wielkości po prawej stronie (oprócz nieznanego
%%$\brm{f}_k$) są określone, toteż można próbować rozwiązać
%%równanie~\eqref{eq:NN23A} względem $\brm{y}_k$. W~przypadku $k < J$ (np. $k=1$
%%dla metody dwu-krokowej) należy zastosować w~danym punkcie $k$ metodę niższego
%%rzędu (co najwyżej $k$-krokową). Mówi się wtedy potocznie o~użyciu metody
%%startującej naszą metodę. Zadaniem takiego ,,startera'' jest wyznaczenie $J-1$
%%wartości początkowych $\brm{y}_1, \dots \brm{y}_{J-1}$ na~początku procesu
%%całkowania.
%%
%%Dla metod jawnych ($\beta_0 = 0$) wzór \eqref{eq:NN23A} przybiera szczególną
%%postać
%%%%%%%%%%%%%%%%%%%%%%%%%%%%%%%%%%%%%%%%%%%%%%%%%%%%%%%%%%%%%%%%%%%%%%%%%%%%%%%%
%%\begin{equation}
%%  \brm{y}_k = \sum_{j=1}^J \alpha_j \brm{y}_{k-j} + h \sum_{j=1}^J \beta_j \brm{f}_{k-j}
%%  \label{eq:EQ0F7}
%%\end{equation}
%%%%%%%%%%%%%%%%%%%%%%%%%%%%%%%%%%%%%%%%%%%%%%%%%%%%%%%%%%%%%%%%%%%%%%%%%%%%%%%%
%%tzn. znika z~niego wyrażenie $\beta_0 \brm{f}_k$ i~po prawej stronie zostają
%%tylko wielkości z~poprzednich kroków.
%%
%%Do popularnych metod wielokrokowych należą metody Adamsa (jawne i~niejawne).
%%W~metodach tych $\alpha_0=1$, $\alpha_1=-1$ i~$\alpha_j = 0$ dla $j > 1$.
%%
%%\subsubsection{Jawne metody Adamsa (metody Adamsa-Bashfortha)}
%%\label{sec:JVLO6}
%%
%%Wszystkie metody Adamsa-Bashfortha działają wg poniższej formuły liniowej
%%%%%%%%%%%%%%%%%%%%%%%%%%%%%%%%%%%%%%%%%%%%%%%%%%%%%%%%%%%%%%%%%%%%%%%%%%%%%%%%
%%\begin{equation}
%%  \brm{y}_k = \brm{y}_{k-1} + h \sum_{j=1}^J \beta_j \brm{f}_{k-j}.
%%  \label{eq:DUQAC}
%%\end{equation}
%%%%%%%%%%%%%%%%%%%%%%%%%%%%%%%%%%%%%%%%%%%%%%%%%%%%%%%%%%%%%%%%%%%%%%%%%%%%%%%%
%%Istnieje cała rodzina metod Adamsa-Bashforta różniących się rzędem metody.
%%Najczęściej stosowane są metody rzędu $p$ od $1$ do $6$, dla których
%%współczynniki $\beta_j$ przedstawiono w~tablicy~\ref{tab:OWS53}.
%%%%%%%%%%%%%%%%%%%%%%%%%%%%%%%%%%%%%%%%%%%%%%%%%%%%%%%%%%%%%%%%%%%%%%%%%%%%%%%%
%%\begin{table}[htbp]
%%  \caption{Metody Adamsa-Bashfortha dla $p=1,\dots6$}
%%  \label{tab:OWS53}
%%  \centering
%%  \begin{tabular}{|c|c|r|r|r|r|r|r|r|}
%%    \hline
%%    $p$ & $J$ & $j \rightarrow$ &    1 &    2 &    3 &    4 & 5    & 6    \\ \hline
%%     1  &  1  &   $\beta_j$     &    1 &      &      &      &      &      \\
%%     2  &  2  &   $2\beta_j$    &    3 &   -1 &      &      &      &      \\
%%     3  &  3  &  $12\beta_j$    &   23 &  -16 &    5 &      &      &      \\
%%     4  &  4  &  $24\beta_j$    &   55 &  -59 &   37 &   -9 &      &      \\
%%     5  &  5  & $720\beta_j$    & 1901 &-2774 & 2616 & 1274 &  251 &      \\
%%     6  &  6  &$1440\beta_j$    & 4277 &-7923 & 9982 &-7298 & 2877 & -475 \\ \hline
%%  \end{tabular}
%%\end{table}
%%%%%%%%%%%%%%%%%%%%%%%%%%%%%%%%%%%%%%%%%%%%%%%%%%%%%%%%%%%%%%%%%%%%%%%%%%%%%%%%
%%Przykładowo, zastosowanie współczynników dla $p=3$ generuje metodę 3-krokową
%%($J=3$) jak poniżej
%%%%%%%%%%%%%%%%%%%%%%%%%%%%%%%%%%%%%%%%%%%%%%%%%%%%%%%%%%%%%%%%%%%%%%%%%%%%%%%%
%%\begin{equation}
%%  \brm{y}_{k} =  \brm{y}_{k-1} + \frac{h}{12} \left(23\,\brm{f}_{k-1} - 16\,\brm{f}_{k-2} + 5\,\brm{f}_{k-3} \right).
%%  \label{eq:IYF96}
%%\end{equation}
%%%%%%%%%%%%%%%%%%%%%%%%%%%%%%%%%%%%%%%%%%%%%%%%%%%%%%%%%%%%%%%%%%%%%%%%%%%%%%%%
%%Należy zauważyć, że metoda Adamsa-Bashfortha pierwszego rzędu jest w~zasadzie
%%metodą jednokrokową i~jest tożsama z~jawną metodą Eulera.
%%
%%
%%\subsubsection{Niejawne metody Adamsa (metody Adamsa-Moultona)}
%%\label{sec:FYDDW}
%%
%%Niejawne metody Adamsa-Moultona mają postać ogólną
%%%%%%%%%%%%%%%%%%%%%%%%%%%%%%%%%%%%%%%%%%%%%%%%%%%%%%%%%%%%%%%%%%%%%%%%%%%%%%%%
%%\begin{equation}
%%  \brm{y}_k = \brm{y}_{k-1} + h \sum_{j=0}^J \beta_j \brm{f}_{k-j}.
%%  \label{eq:YCGME}
%%\end{equation}
%%%%%%%%%%%%%%%%%%%%%%%%%%%%%%%%%%%%%%%%%%%%%%%%%%%%%%%%%%%%%%%%%%%%%%%%%%%%%%%%
%%Podobnie jak w~przypadku jawnych metod Adamsa, dla każdego $p=1,2,\dots,6$
%%opracowano unikalny zestaw współczynników $\beta_j$ definiujący metodę
%%Adamsa-Moultona rzędu $p$. Współczynniki te zebrano w~tablicy~\ref{tab:06769}.
%%%%%%%%%%%%%%%%%%%%%%%%%%%%%%%%%%%%%%%%%%%%%%%%%%%%%%%%%%%%%%%%%%%%%%%%%%%%%%%%
%%\begin{table}[htbp]
%%  \caption{Metody Adamsa-Moultona dla $p=1,\dots6$}
%%  \label{tab:06769}
%%  \centering
%%  \begin{tabular}{|c|c|r|r|r|r|r|r|r|}
%%    \hline
%%    $p$ & $J$ & $j \rightarrow$ &    0 &    1 &    2 &    3 &    4 &    5 \\ \hline
%%     1  &  1  &   $\beta_j$     &    1 &      &      &      &      &      \\
%%     2  &  1  &   $2\beta_j$    &    1 &    1 &      &      &      &      \\
%%     3  &  2  &  $12\beta_j$    &    5 &    8 &   -1 &      &      &      \\
%%     4  &  3  &  $24\beta_j$    &    9 &   19 &   -5 &    1 &      &      \\
%%     5  &  4  & $720\beta_j$    &  251 &  646 & -264 &  106 &  -19 &      \\
%%     6  &  5  &$1440\beta_j$    &  475 & 1427 & -798 &  482 & -173 &   27 \\ \hline
%%  \end{tabular}
%%\end{table}
%%%%%%%%%%%%%%%%%%%%%%%%%%%%%%%%%%%%%%%%%%%%%%%%%%%%%%%%%%%%%%%%%%%%%%%%%%%%%%%%
%%Przykładowo, metodę Adamsa-Moultona rzędu $p=3$ zapiszemy, zgodnie z~tablicą,
%%jako
%%%%%%%%%%%%%%%%%%%%%%%%%%%%%%%%%%%%%%%%%%%%%%%%%%%%%%%%%%%%%%%%%%%%%%%%%%%%%%%%
%%\begin{equation}
%%  \brm{y}_k = \brm{y}_{k-1} + \frac{h}{12}\left(
%%    5 \brm{f}_k + 8 \brm{f}_{k-1} - 1 \brm{f}_{k-2}
%%  \right)
%%\end{equation}
%%%%%%%%%%%%%%%%%%%%%%%%%%%%%%%%%%%%%%%%%%%%%%%%%%%%%%%%%%%%%%%%%%%%%%%%%%%%%%%%
%%
%%Należy zauważyć, że istnieją dwie różne jedno-krokowe metody Adamsa-Moultona,
%%jedna rzędu $1$, druga rzędu $2$. Ponadto dostrzeżemy, że
%%%%%%%%%%%%%%%%%%%%%%%%%%%%%%%%%%%%%%%%%%%%%%%%%%%%%%%%%%%%%%%%%%%%%%%%%%%%%%%%
%%\begin{itemize}
%%  \item wariant $J=1$ z~$\beta_1 = 0$ (metoda $p=1$) jest w~istocie niejawną
%%        metodą Eulera,
%%  \item wariant $J=1$ z~$\beta_1 \neq 0$ (metoda $p=2$) jest znana skądinąd jako
%%        niejawna metoda trapezów
%%        %%%%%%%%%%%%%%%%%%%%%%%%%%%%%%%%%%%%%%%%%%%%%%%%%%%%%%%%%%%%%%%%%%%%%%
%%        \begin{equation}
%%          \brm{y}_k = \brm{y}_{k-1} + \frac{h}{2} \left(\brm{f}_{k} + \brm{f}_{k-1}\right).
%%          \label{eq:GY62E}
%%        \end{equation}
%%        %%%%%%%%%%%%%%%%%%%%%%%%%%%%%%%%%%%%%%%%%%%%%%%%%%%%%%%%%%%%%%%%%%%%%%
%%\end{itemize}
%%%%%%%%%%%%%%%%%%%%%%%%%%%%%%%%%%%%%%%%%%%%%%%%%%%%%%%%%%%%%%%%%%%%%%%%%%%%%%%%
%%
%%
%%\end{appendices}

\bibliographystyle{unsrtnat}
\bibliography{tfa}

\end{document}

%%\label{??:VSKYR}
%%\label{??:LU2QP}
%%\label{??:H7Q5E}
%%\label{??:OBNDG}
%%\label{??:YQOKX}
%%\label{??:VAQ7B}
%%\label{??:8GIS7}
%%\label{??:8FMI2}
%%\label{??:DQ4IE}
%%\label{??:6SYZY}
%%\label{??:1RMRC}
%%\label{??:LKWY0}

% vim: set syntax=tex tabstop=2 shiftwidth=2 expandtab spell spelllang=pl:
